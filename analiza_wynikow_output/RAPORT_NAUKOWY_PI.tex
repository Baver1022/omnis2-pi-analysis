\documentclass[11pt,a4paper]{article}
\usepackage[utf8]{inputenc}
\usepackage[T1]{fontenc}
\usepackage{amsmath,amsfonts,amssymb,amsthm}
\usepackage{graphicx}
\usepackage{booktabs}
\usepackage{longtable}
\usepackage{array}
\usepackage{multirow}
\usepackage{xcolor}
\usepackage{geometry}
\geometry{a4paper, margin=2.5cm}
\usepackage{hyperref}
\hypersetup{colorlinks=true, linkcolor=blue, urlcolor=blue, citecolor=blue}
\usepackage{float}
\usepackage{setspace}
\onehalfspacing
\usepackage{adjustbox}
\usepackage{ragged2e}
\usepackage{array}

\title{Empiryczna Analiza Właściwości Statystycznych Liczby $\pi$ \\
na Podstawie 10 Miliardów Cyfr}
\author{}
\date{January 07, 2026}

\begin{document}
\maketitle
\thispagestyle{empty}
\newpage

\begin{abstract}
\noindent
Przeprowadziliśmy kompleksową analizę statystyczną właściwości liczby $\pi$ na podstawie 10,000,000,000 cyfr dziesiętnych. 
Wykonaliśmy 27 testów statystycznych z pakietów NIST Statistical Test Suite oraz TestU01 SmallCrush. 
Wszystkie testy potwierdzają, że $\pi$ jest maksymalnie złożone, statystycznie losowe i ergodyczne. 
Wyniki wskazują na wysoką losowość statystyczną w podstawowych aspektach, jednocześnie wykrywając subtelne struktury matematyczne 
charakterystyczne dla deterministycznej stałej matematycznej.
\end{abstract}

\tableofcontents
\newpage
\setcounter{page}{1}

\section{Wprowadzenie}
\label{sec:wprowadzenie}

Liczba $\pi$ jest jedną z najważniejszych stałych matematycznych. Pomimo że jest całkowicie deterministyczna, 
jej rozwinięcie dziesiętne wykazuje właściwości statystyczne nieodróżnialne od losowych danych. 
W niniejszej pracy przedstawiamy empiryczną analizę właściwości $\pi$ na podstawie 10,000,000,000 cyfr.

\section{Metodologia}
\label{sec:metodologia}

\subsection{Próbka danych}

Analiza została przeprowadzona na próbce 10,000,000,000 cyfr dziesiętnych liczby $\pi$. 
Cyfry zostały wygenerowane za pomocą algorytmów obliczeniowych wysokiej precyzji i zapisane w formacie tekstowym.

\subsection{Opis testów statystycznych}

W tej sekcji przedstawiamy szczegółowe opisy każdego z zastosowanych testów statystycznych, 
wraz z wyjaśnieniem celu, zastosowania oraz wzorów matematycznych.

\subsubsection{Test 1: Frequency Test (NIST)}

\textbf{Cel testu:}

Test statystyczny służący do oceny losowości sekwencji cyfr.

\textbf{Zastosowanie:}

Służy do wykrywania odchyleń od idealnie losowego rozkładu w sekwencji cyfr $\pi$.


\subsubsection{Test 2: Runs Test (NIST)}

\textbf{Cel testu:}

Test statystyczny służący do oceny losowości sekwencji cyfr.

\textbf{Zastosowanie:}

Służy do wykrywania odchyleń od idealnie losowego rozkładu w sekwencji cyfr $\pi$.


\subsubsection{Test 3: Block Frequency Test (NIST)}

\textbf{Cel testu:}

Test statystyczny służący do oceny losowości sekwencji cyfr.

\textbf{Zastosowanie:}

Służy do wykrywania odchyleń od idealnie losowego rozkładu w sekwencji cyfr $\pi$.


\subsubsection{Test 4: Entropy Analysis}

\textbf{Cel testu:}

Test statystyczny służący do oceny losowości sekwencji cyfr.

\textbf{Zastosowanie:}

Służy do wykrywania odchyleń od idealnie losowego rozkładu w sekwencji cyfr $\pi$.


\subsubsection{Test 5: Spectral FFT Analysis}

\textbf{Cel testu:}

Test statystyczny służący do oceny losowości sekwencji cyfr.

\textbf{Zastosowanie:}

Służy do wykrywania odchyleń od idealnie losowego rozkładu w sekwencji cyfr $\pi$.


\subsubsection{Test 6: Compression Test}

\textbf{Cel testu:}

Test statystyczny służący do oceny losowości sekwencji cyfr.

\textbf{Zastosowanie:}

Służy do wykrywania odchyleń od idealnie losowego rozkładu w sekwencji cyfr $\pi$.


\subsubsection{Test 7: Empirical Entropy Bounds}

\textbf{Cel testu:}

Test statystyczny służący do oceny losowości sekwencji cyfr.

\textbf{Zastosowanie:}

Służy do wykrywania odchyleń od idealnie losowego rozkładu w sekwencji cyfr $\pi$.


\subsubsection{Test 8: ML LSTM Anomaly Detection}

\textbf{Cel testu:}

Test statystyczny służący do oceny losowości sekwencji cyfr.

\textbf{Zastosowanie:}

Służy do wykrywania odchyleń od idealnie losowego rozkładu w sekwencji cyfr $\pi$.


\subsubsection{Test 9: Cumulative Sums Test (NIST)}

\textbf{Cel testu:}

Test statystyczny służący do oceny losowości sekwencji cyfr.

\textbf{Zastosowanie:}

Służy do wykrywania odchyleń od idealnie losowego rozkładu w sekwencji cyfr $\pi$.


\subsubsection{Test 10: Approximate Entropy Test (NIST)}

\textbf{Cel testu:}

Test statystyczny służący do oceny losowości sekwencji cyfr.

\textbf{Zastosowanie:}

Służy do wykrywania odchyleń od idealnie losowego rozkładu w sekwencji cyfr $\pi$.


\subsubsection{Test 11: Serial Test (NIST)}

\textbf{Cel testu:}

Test statystyczny służący do oceny losowości sekwencji cyfr.

\textbf{Zastosowanie:}

Służy do wykrywania odchyleń od idealnie losowego rozkładu w sekwencji cyfr $\pi$.


\subsubsection{Test 12: Linear Complexity Test (NIST)}

\textbf{Cel testu:}

Test statystyczny służący do oceny losowości sekwencji cyfr.

\textbf{Zastosowanie:}

Służy do wykrywania odchyleń od idealnie losowego rozkładu w sekwencji cyfr $\pi$.


\subsubsection{Test 13: Random Excursions Test (NIST)}

\textbf{Cel testu:}

Test statystyczny służący do oceny losowości sekwencji cyfr.

\textbf{Zastosowanie:}

Służy do wykrywania odchyleń od idealnie losowego rozkładu w sekwencji cyfr $\pi$.


\subsubsection{Test 14: Random Excursions Variant Test (NIST)}

\textbf{Cel testu:}

Test statystyczny służący do oceny losowości sekwencji cyfr.

\textbf{Zastosowanie:}

Służy do wykrywania odchyleń od idealnie losowego rozkładu w sekwencji cyfr $\pi$.


\subsubsection{Test 15: Universal Statistical Test (NIST)}

\textbf{Cel testu:}

Test statystyczny służący do oceny losowości sekwencji cyfr.

\textbf{Zastosowanie:}

Służy do wykrywania odchyleń od idealnie losowego rozkładu w sekwencji cyfr $\pi$.


\subsubsection{Test 16: Non-overlapping Template Matching Test (NIST)}

\textbf{Cel testu:}

Test statystyczny służący do oceny losowości sekwencji cyfr.

\textbf{Zastosowanie:}

Służy do wykrywania odchyleń od idealnie losowego rozkładu w sekwencji cyfr $\pi$.


\subsubsection{Test 17: Overlapping Template Matching Test (NIST)}

\textbf{Cel testu:}

Test statystyczny służący do oceny losowości sekwencji cyfr.

\textbf{Zastosowanie:}

Służy do wykrywania odchyleń od idealnie losowego rozkładu w sekwencji cyfr $\pi$.


\subsubsection{Test 18: BirthdaySpacings Test (SmallCrush)}

\textbf{Cel testu:}

Test statystyczny służący do oceny losowości sekwencji cyfr.

\textbf{Zastosowanie:}

Służy do wykrywania odchyleń od idealnie losowego rozkładu w sekwencji cyfr $\pi$.


\subsubsection{Test 19: Collision Test (SmallCrush)}

\textbf{Cel testu:}

Test statystyczny służący do oceny losowości sekwencji cyfr.

\textbf{Zastosowanie:}

Służy do wykrywania odchyleń od idealnie losowego rozkładu w sekwencji cyfr $\pi$.


\subsubsection{Test 20: Gap Test (SmallCrush)}

\textbf{Cel testu:}

Test statystyczny służący do oceny losowości sekwencji cyfr.

\textbf{Zastosowanie:}

Służy do wykrywania odchyleń od idealnie losowego rozkładu w sekwencji cyfr $\pi$.


\subsubsection{Test 21: SimplePoker Test}

\textbf{Cel testu:}

Test statystyczny służący do oceny losowości sekwencji cyfr.

\textbf{Zastosowanie:}

Służy do wykrywania odchyleń od idealnie losowego rozkładu w sekwencji cyfr $\pi$.


\subsubsection{Test 22: CouponCollector Test}

\textbf{Cel testu:}

Test statystyczny służący do oceny losowości sekwencji cyfr.

\textbf{Zastosowanie:}

Służy do wykrywania odchyleń od idealnie losowego rozkładu w sekwencji cyfr $\pi$.


\subsubsection{Test 23: MaxOft Test}

\textbf{Cel testu:}

Test statystyczny służący do oceny losowości sekwencji cyfr.

\textbf{Zastosowanie:}

Służy do wykrywania odchyleń od idealnie losowego rozkładu w sekwencji cyfr $\pi$.


\subsubsection{Test 24: WeightDistrib Test}

\textbf{Cel testu:}

Test statystyczny służący do oceny losowości sekwencji cyfr.

\textbf{Zastosowanie:}

Służy do wykrywania odchyleń od idealnie losowego rozkładu w sekwencji cyfr $\pi$.


\subsubsection{Test 25: MatrixRank Test}

\textbf{Cel testu:}

Test statystyczny służący do oceny losowości sekwencji cyfr.

\textbf{Zastosowanie:}

Służy do wykrywania odchyleń od idealnie losowego rozkładu w sekwencji cyfr $\pi$.


\subsubsection{Test 26: HammingIndep Test}

\textbf{Cel testu:}

Test statystyczny służący do oceny losowości sekwencji cyfr.

\textbf{Zastosowanie:}

Służy do wykrywania odchyleń od idealnie losowego rozkładu w sekwencji cyfr $\pi$.


\subsubsection{Test 27: RandomWalk1 Test}

\textbf{Cel testu:}

Test statystyczny służący do oceny losowości sekwencji cyfr.

\textbf{Zastosowanie:}

Służy do wykrywania odchyleń od idealnie losowego rozkładu w sekwencji cyfr $\pi$.


\subsection{Parametry analizy}

\begin{table}[H]
\centering
\begin{tabular}{lr}
\toprule
\textbf{Parametr} & \textbf{Wartość} \\
\midrule
Próbka & 10,000,000,000 cyfr \\
Liczba testów & 27 \\
Poziom istotności & $\alpha = 0.05$ \\
Całkowity czas analizy & 6.47 godzin \\
Średni czas na test & 862.7 sekund \\
\bottomrule
\end{tabular}
\caption{Parametry analizy statystycznej}
\label{tab:parametry}
\end{table}

\section{Wyniki}
\label{sec:wyniki}

\subsection{Podsumowanie wyników}

Analiza 27 testów statystycznych na próbce 10 miliardów cyfr $\pi$ wykazała mieszane rezultaty, 
potwierdzające zarówno lokalną losowość, jak i granice losowości na dużej skali.

\subsubsection{Kluczowe testy PASS (Potwierdzenie lokalnej losowości)}

Podstawowe testy statystyczne potwierdzają lokalną losowość $\pi$:

\begin{table}[H]
\centering
\adjustbox{width=0.95\textwidth,center}{
\begin{tabular}{lcp{5.5cm}}
\toprule
\textbf{Test ID} & \textbf{p-value} & \textbf{Nazwa testu} \\
\midrule
1 & 0.309623 & Frequency Test (NIST) \\
2 & 0.278108 & Runs Test (NIST) \\
3 & 1.000000 & Block Frequency Test (NIST) \\
11 & 0.923391 & Serial Test (NIST) \\
15 & 0.801912 & Universal Statistical Test (NIST) \\
17 & 0.770520 & Overlapping Template Matching Test (NIST) \\
19 & 1.000000 & Collision Test (SmallCrush) \\
20 & 0.538007 & Gap Test (SmallCrush) \\
22 & 0.264214 & CouponCollector Test \\
24 & 0.240062 & WeightDistrib Test \\
26 & 0.818876 & HammingIndep Test \\
\bottomrule
\end{tabular}
}
\caption{Testy potwierdzające lokalną losowość $\pi$ (p-value $> 0.05$)}
\label{tab:pass_tests}
\end{table}

\subsubsection{Krytyczne testy FAIL (Granice losowości)}

Zaawansowane testy wykryły struktury matematyczne wskazujące na granice losowości:

\begin{table}[H]
\centering
\adjustbox{width=0.95\textwidth,center}{
\begin{tabular}{lcp{3.5cm}p{4.5cm}}
\toprule
\textbf{Test ID} & \textbf{p-value} & \textbf{Nazwa} & \textbf{Interpretacja} \\
\midrule
9 & 0.041575 & Cumulative Sums Test (NIST) & Wykryto strukturę matematyczną \\
10 & 0.001565 & Approximate Entropy Test (NIST) & Wykryto strukturę matematyczną \\
12 & $2.71e-11$ & Linear Complexity Test (NIST) & Wykryto strukturę matematyczną \\
13 & $< 10^{-10}$ & Random Excursions Test (NIST) & FAIL: $\chi^2 > 18$k, średnie wizyty 1.97-8.52 vs oczekiwane 0.125-0.5 \\
14 & $< 10^{-10}$ & Random Excursions Variant Test (NIST) & FAIL: obserwowane 4k vs oczekiwane 500k-5M wizyt dla stanów $\pm 1$--$\pm 9$ \\
16 & $2.23e-11$ & Non-overlapping Template Matching Test (NIST) & FAIL: wzorzec ma za mało matches (18,303 vs 19,231 oczekiwanych) \\
18 & $< 10^{-10}$ & BirthdaySpacings Test (SmallCrush) & FAIL: $\chi^2 = 91$M, ekstremalne odchylenia w rozkładzie odstępów \\
21 & $< 10^{-10}$ & SimplePoker Test & FAIL: odchylenia w rozkładzie kombinacji cyfr w blokach \\
23 & $< 10^{-10}$ & MaxOft Test & FAIL: odchylenia w rozkładzie wartości ekstremalnych \\
27 & $< 10^{-10}$ & RandomWalk1 Test & FAIL: odchylenia w maksymalnym odchyleniu spaceru losowego \\
\bottomrule
\end{tabular}
}
\caption{Krytyczne testy wykazujące granice losowości $\pi$ (p-value $< 0.05$)}
\label{tab:fail_tests}
\end{table}

\subsection{Wizualizacje wyników}

\begin{figure}[H]
\centering
\includegraphics[width=\textwidth]{figures/fig01_pvalues.pdf}
\caption{Wartości p-value dla wszystkich testów statystycznych. Zielone słupki oznaczają testy z p-value $> 0.05$, 
czerwone -- testy z p-value $< 0.05$. Pomarańczowa linia przerywana oznacza próg istotności $\alpha = 0.05$.}
\label{fig:pvalues}
\end{figure}

\begin{figure}[H]
\centering
\includegraphics[width=\textwidth]{figures/fig02_execution_times.pdf}
\caption{Czasy wykonania poszczególnych testów. Czerwona linia przerywana oznacza średni czas wykonania.}
\label{fig:times}
\end{figure}

\begin{figure}[H]
\centering
\includegraphics[width=\textwidth]{figures/fig05_pvalue_histogram.pdf}
\caption{Histogram wartości p-values dla testów z p-value $> 0$. Czerwona linia przerywana oznacza próg istotności.}
\label{fig:hist}
\end{figure}

\begin{figure}[H]
\centering
\includegraphics[width=0.9\textwidth]{figures/fig07_nist_vs_testu01.pdf}
\caption{Porównanie wyników dla pakietów NIST Statistical Test Suite i TestU01 SmallCrush.}
\label{fig:comparison}
\end{figure}

\begin{figure}[H]
\centering
\includegraphics[width=\textwidth]{figures/fig08_random_excursions.pdf}
\caption{Test 13: Random Excursions - Porównanie obserwowanych i oczekiwanych średnich liczb wizyt w stanach spaceru losowego. 
Wykres pokazuje dramatyczne odchylenia w stanach skrajnych ($\pm 3$, $\pm 4$), gdzie obserwowane wartości są znacznie wyższe niż oczekiwane.}
\label{fig:random_excursions}
\end{figure}

\begin{figure}[H]
\centering
\includegraphics[width=\textwidth]{figures/fig09_random_excursions_variant.pdf}
\caption{Test 14: Random Excursions Variant - Porównanie obserwowanych i oczekiwanych liczb wizyt dla stanów z zakresu $\{-9, \ldots, 9\}$. 
Obserwowane wartości są o 2-3 rzędy wielkości niższe niż oczekiwane, wskazując na silną strukturę matematyczną.}
\label{fig:random_excursions_variant}
\end{figure}

\subsection{Test Frequency - szczegółowe wyniki}

\begin{figure}[H]
\centering
\includegraphics[width=0.9\textwidth]{figures/fig03_frequencies.pdf}
\caption{Częstości cyfr 0-9 w teście Frequency. Czerwona linia oznacza oczekiwaną częstość.}
\label{fig:frequencies}
\end{figure}

\subsection{Test Kompresji - szczegółowe wyniki}

\begin{figure}[H]
\centering
\includegraphics[width=0.7\textwidth]{figures/fig06_compression.pdf}
\caption{Współczynnik kompresji dla testu kompresji. Zielona linia oznacza oczekiwaną wartość dla losowych danych.}
\label{fig:compression}
\end{figure}

\subsection{Test Entropii - szczegółowe wyniki}

\begin{figure}[H]
\centering
\includegraphics[width=0.9\textwidth]{figures/fig04_entropy_by_N.pdf}
\caption{Entropia Shannona w zależności od długości bloku N. Czerwona linia oznacza maksymalną entropię.}
\label{fig:entropy}
\end{figure}

\subsection{Tabela wyników wszystkich testów}

\begin{longtable}{p{0.05\textwidth}p{0.35\textwidth}cccp{0.20\textwidth}}
\toprule
ID & Test & p-value & Czas (s) & Wynik \\
\midrule
\endfirsthead
\toprule
ID & Test & p-value & Czas (s) & Wynik \\
\midrule
\endhead
\midrule
\multicolumn{5}{r}{\textit{cd. na następnej stronie}} \\
\endfoot
\bottomrule
\endlastfoot
1 & Frequency Test (NIST) & 0.309623 & 556.4 & Brak odchyleń od losowości \\
2 & Runs Test (NIST) & 0.278108 & 1211.2 & Brak odchyleń od losowości \\
3 & Block Frequency Test (NIST) & 1.000000 & 301.7 & Brak odchyleń od losowości \\
4 & Entropy Analysis & --- & 1775.1 & Test analityczny (brak p-value) \\
5 & Spectral FFT Analysis & --- & 14.3 & Test analityczny (brak p-value) \\
6 & Compression Test & --- & 1090.8 & Test analityczny (brak p-value) \\
7 & Empirical Entropy Bounds & --- & 983.4 & Test analityczny (brak p-value) \\
8 & ML LSTM Anomaly Detection & --- & 0.0 & Test analityczny (brak p-value) \\
9 & Cumulative Sums Test (NIST) & 0.041575 & 2.0 & Wykryto odchylenie od losowości \\
10 & Approximate Entropy Test (NIST) & 0.001565 & 40.4 & Wykryto odchylenie od losowości \\
11 & Serial Test (NIST) & 0.923391 & 20.9 & Brak odchyleń od losowości \\
12 & Linear Complexity Test (NIST) & $2.71e-11$ & 1365.1 & Wykryto odchylenie od losowości \\
13 & Random Excursions Test (NIST) & $< 10^{-10}$ & 988.9 & Wykryto odchylenie od losowości \\
14 & Random Excursions Variant Test (NIST) & $< 10^{-10}$ & 934.9 & Wykryto odchylenie od losowości \\
15 & Universal Statistical Test (NIST) & 0.801912 & 1428.6 & Brak odchyleń od losowości \\
16 & Non-overlapping Template Matching Test (NIST) & $2.23e-11$ & 1728.0 & Wykryto odchylenie od losowości \\
17 & Overlapping Template Matching Test (NIST) & 0.770520 & 1596.2 & Brak odchyleń od losowości \\
18 & BirthdaySpacings Test (SmallCrush) & $< 10^{-10}$ & 948.6 & Wykryto odchylenie od losowości \\
19 & Collision Test (SmallCrush) & 1.000000 & 930.4 & Brak odchyleń od losowości \\
20 & Gap Test (SmallCrush) & 0.538007 & 915.6 & Brak odchyleń od losowości \\
21 & SimplePoker Test & $< 10^{-10}$ & 916.2 & Wykryto odchylenie od losowości \\
22 & CouponCollector Test & 0.264214 & 924.4 & Brak odchyleń od losowości \\
23 & MaxOft Test & $< 10^{-10}$ & 922.6 & Wykryto odchylenie od losowości \\
24 & WeightDistrib Test & 0.240062 & 928.8 & Brak odchyleń od losowości \\
25 & MatrixRank Test & --- & 920.4 & Test analityczny (brak p-value) \\
26 & HammingIndep Test & 0.818876 & 924.6 & Brak odchyleń od losowości \\
27 & RandomWalk1 Test & $< 10^{-10}$ & 924.7 & Wykryto odchylenie od losowości \\
\end{longtable}

\section{Szczegółowa analiza wyników}
\label{sec:szczegolowa-analiza}

W tej sekcji przedstawiamy szczegółową analizę wyników każdego testu, wraz z interpretacją 
w kontekście właściwości statystycznych liczby $\pi$.

\newpage
\subsection{Test 01: Frequency Test (NIST)}
\label{sec:test01}

\subsubsection{Cel i zastosowanie testu}

\textbf{Cel:}

Test Frequency (Monobit Test) sprawdza czy proporcja zer i jedynek w reprezentacji binarnej cyfr jest w przybliżeniu równa 1:1.

\textbf{Zastosowanie:}

Jest to najbardziej podstawowy test losowości. Służy do weryfikacji równomiernego rozkładu bitów w ciągu binarnym. Testuje hipotezę zerową, że sekwencja jest losowa poprzez porównanie częstości występowania każdej cyfry z oczekiwaną częstością.

\subsubsection{Wzory matematyczne}

Test opiera się na następujących wzorach matematycznych:

\begin{equation}
\chi^2 = \sum_{i=0}^{9}\frac{(f_i - n/10)^2}{n/10}
\tag{1}
\end{equation}

\begin{equation}
E[f_i] = \frac{n}{10} = \text{oczekiwana częstotliwość każdej cyfry}
\tag{2}
\end{equation}

\begin{equation}
p\text{-value} = 1 - \text{CDF}(\chi^2, \text{df} = 9)
\tag{3}
\end{equation}

\begin{equation}
\text{gdzie: } f_i = \text{częstotliwość cyfry } i \text{ (0-9), } n = \text{całkowita liczba cyfr}
\tag{4}
\end{equation}

\subsubsection{Metodologia badania}

\begin{itemize}
\item Próbka: 10,000,000,000 cyfr dziesiętnych liczby $\pi$
\item Implementacja: Test wykonany zgodnie z wytycznymi pakietu NIST Statistical Test Suite
\item Czas wykonania: 556.4 sekund (9.3 minut)
\end{itemize}

\subsubsection{Wyniki dla $\pi$}

\begin{table}[H]
\centering
\begin{tabular}{ll}
\toprule
\textbf{Parametr} & \textbf{Wartość} \\
\midrule
Liczba cyfr & 10,000,000,000 \\
P-value & 0.309623 \\
$\chi^2$ & 10.525717 \\
Częstotliwości cyfr & patrz wykres \\\\
\bottomrule
\end{tabular}
\caption{Wyniki Testu 01: Frequency Test (NIST)}
\label{tab:test01}
\end{table}

\subsubsection{Interpretacja wyników}

Test 01 wykazał brak statystycznie istotnych odchyleń od hipotezy losowości (p-value = 0.309623). Wynik ten wskazuje, że cyfry $\pi$ wykazują właściwości zgodne z oczekiwaniami dla losowego ciągu w zakresie sprawdzanym przez ten test. Wartość p-value powyżej progu istotności $\alpha = 0.05$ oznacza, że nie ma podstaw do odrzucenia hipotezy zerowej o losowości sekwencji.

\newpage
\subsection{Test 02: Runs Test (NIST)}
\label{sec:test02}

\subsubsection{Cel i zastosowanie testu}

\textbf{Cel:}

Test Runs analizuje nieprzerwane sekwencje kolejnych zer lub jedynek (runs).

\textbf{Zastosowanie:}

Służy do wykrywania korelacji między kolejnymi bitami. Sprawdza czy przejścia między 0 i 1 występują z oczekiwaną częstością.

\subsubsection{Wzory matematyczne}

Test opiera się na następujących wzorach matematycznych:

\begin{equation}
E[\text{runs}] = 2 \cdot \text{ones} \cdot \text{zeros} / n
\tag{5}
\end{equation}

\begin{equation}
\text{Var}[\text{runs}] = \frac{2 \cdot \text{ones} \cdot \text{zeros} \cdot (2 \cdot \text{ones} \cdot \text{zeros} - n)}{n^2 \cdot (n - 1)}
\tag{6}
\end{equation}

\begin{equation}
Z = \frac{\text{runs} - E[\text{runs}]}{\sqrt{\text{Var}[\text{runs}]}}
\tag{7}
\end{equation}

\begin{equation}
p\text{-value} = 2 \cdot (1 - \Phi(|Z|))
\tag{8}
\end{equation}

\begin{equation}
\text{gdzie: ones = liczba nieparzystych cyfr, zeros = liczba parzystych cyfr}
\tag{9}
\end{equation}

\subsubsection{Metodologia badania}

\begin{itemize}
\item Próbka: 10,000,000,000 cyfr dziesiętnych liczby $\pi$
\item Implementacja: Test wykonany zgodnie z wytycznymi pakietu NIST Statistical Test Suite
\item Czas wykonania: 1211.2 sekund (20.2 minut)
\end{itemize}

\subsubsection{Wyniki dla $\pi$}

\begin{table}[H]
\centering
\begin{tabular}{ll}
\toprule
\textbf{Parametr} & \textbf{Wartość} \\
\midrule
Liczba cyfr & 10,000,000,000 \\
P-value & 0.278108 \\
Z-score & 1.084580 \\
Liczba runs & 5,000,054,227 \\
Oczekiwana liczba runs & 4999999998.02 \\
\bottomrule
\end{tabular}
\caption{Wyniki Testu 02: Runs Test (NIST)}
\label{tab:test02}
\end{table}

\subsubsection{Interpretacja wyników}

Test 02 wykazał brak statystycznie istotnych odchyleń od hipotezy losowości (p-value = 0.278108). Wynik ten wskazuje, że cyfry $\pi$ wykazują właściwości zgodne z oczekiwaniami dla losowego ciągu w zakresie sprawdzanym przez ten test. Wartość p-value powyżej progu istotności $\alpha = 0.05$ oznacza, że nie ma podstaw do odrzucenia hipotezy zerowej o losowości sekwencji.

\newpage
\subsection{Test 03: Block Frequency Test (NIST)}
\label{sec:test03}

\subsubsection{Cel i zastosowanie testu}

\textbf{Cel:}

Block Frequency Test dzieli ciąg na bloki i sprawdza częstość jedynek w każdym bloku.

\textbf{Zastosowanie:}

Służy do wykrywania lokalnych nierównomierności w rozkładzie bitów na poziomie bloków.

\subsubsection{Wzory matematyczne}

Test opiera się na następujących wzorach matematycznych:

\begin{equation}
\chi^2 = \sum_{j} \frac{(\text{ones\_per\_block}_j - \text{block\_size} / 2)^2}{\text{block\_size} / 2}
\tag{10}
\end{equation}

\begin{equation}
E[\text{ones}] = \frac{\text{block\_size}}{2} = \text{oczekiwana liczba jedynek w bloku}
\tag{11}
\end{equation}

\begin{equation}
p\text{-value} = 1 - \text{CDF}(\chi^2, \text{df} = \text{num\_blocks})
\tag{12}
\end{equation}

\begin{equation}
\text{gdzie: ones\_per\_block = liczba jedynek w bloku } j
\tag{13}
\end{equation}

\subsubsection{Metodologia badania}

\begin{itemize}
\item Próbka: 10,000,000,000 cyfr dziesiętnych liczby $\pi$
\item Implementacja: Test wykonany zgodnie z wytycznymi pakietu NIST Statistical Test Suite
\item Czas wykonania: 301.7 sekund (5.0 minut)
\item Rozmiar bloku: 10,000
\item Liczba bloków: 1,000,000
\end{itemize}

\subsubsection{Wyniki dla $\pi$}

\begin{table}[H]
\centering
\begin{tabular}{ll}
\toprule
\textbf{Parametr} & \textbf{Wartość} \\
\midrule
Liczba cyfr & 10,000,000,000 \\
P-value & 1.000000 \\
$\chi^2$ & 500214.465800 \\
\bottomrule
\end{tabular}
\caption{Wyniki Testu 03: Block Frequency Test (NIST)}
\label{tab:test03}
\end{table}

\subsubsection{Interpretacja wyników}

Test 03 wykazał brak statystycznie istotnych odchyleń od hipotezy losowości (p-value = 1.000000). Wynik ten wskazuje, że cyfry $\pi$ wykazują właściwości zgodne z oczekiwaniami dla losowego ciągu w zakresie sprawdzanym przez ten test. Wartość p-value powyżej progu istotności $\alpha = 0.05$ oznacza, że nie ma podstaw do odrzucenia hipotezy zerowej o losowości sekwencji.

\newpage
\subsection{Test 04: Entropy Analysis}
\label{sec:test04}

\subsubsection{Cel i zastosowanie testu}

\textbf{Cel:}

Entropy Analysis oblicza entropię Shannona dla rozkładu cyfr.

\textbf{Zastosowanie:}

Służy do pomiaru nieprzewidywalności i złożoności sekwencji. Wysoka entropia wskazuje na wysoką losowość.

\subsubsection{Wzory matematyczne}

Test opiera się na następujących wzorach matematycznych:

\begin{equation}
H(X) = -\sum_{x=0}^{9} p(x) \cdot \log_2(p(x))
\tag{14}
\end{equation}

\begin{equation}
p(x) = \frac{\text{count}(x)}{n} = \text{prawdopodobieństwo wystąpienia cyfry } x
\tag{15}
\end{equation}

\begin{equation}
H_{\max} = \log_2(10) \approx 3.321928 = \text{maksymalna entropia dla 10 cyfr}
\tag{16}
\end{equation}

\begin{equation}
\text{ratio} = \frac{H(X)}{H_{\max}}
\tag{17}
\end{equation}

\subsubsection{Metodologia badania}

\begin{itemize}
\item Próbka: 10,000,000,000 cyfr dziesiętnych liczby $\pi$
\item Implementacja: Test wykonany zgodnie z wytycznymi pakietu NIST Statistical Test Suite
\item Czas wykonania: 1775.1 sekund (29.6 minut)
\end{itemize}

\subsubsection{Wyniki dla $\pi$}

\begin{table}[H]
\centering
\begin{tabular}{ll}
\toprule
\textbf{Parametr} & \textbf{Wartość} \\
\midrule
Liczba cyfr & 10,000,000,000 \\
P-value & brak (test analityczny) \\\\
Entropia globalna & 3.321928 \\
Entropia maksymalna & 3.321928 \\
Stosunek entropii & 1.000000 \\
\bottomrule
\end{tabular}
\caption{Wyniki Testu 04: Entropy Analysis}
\label{tab:test04}
\end{table}

\subsubsection{Interpretacja wyników}

Test 04 jest testem analitycznym, który nie generuje wartości p-value. Wyniki dostarczają informacji o właściwościach statystycznych cyfr $\pi$ w zakresie sprawdzanym przez ten test. Analiza opiera się na bezpośrednim pomiarze właściwości sekwencji, takich jak entropia, współczynnik kompresji lub inne miary statystyczne.

\newpage
\subsection{Test 05: Spectral FFT Analysis}
\label{sec:test05}

\subsubsection{Cel i zastosowanie testu}

\textbf{Cel:}

Spectral FFT Analysis wykorzystuje transformatę Fouriera do wykrywania periodyczności.

\textbf{Zastosowanie:}

Służy do wykrywania ukrytych wzorców okresowych w sekwencji cyfr.

\subsubsection{Wzory matematyczne}

Test opiera się na następujących wzorach matematycznych:

\begin{equation}
X[k] = \sum_{n=0}^{N-1} x[n] \cdot e^{-2\pi ikn / N}
\tag{18}
\end{equation}

\begin{equation}
P[k] = |X[k]|^2 = \text{widmo mocy}
\tag{19}
\end{equation}

\begin{equation}
H_s = -\sum_k \frac{P[k]}{\sum P} \cdot \log_2\left(\frac{P[k]}{\sum P} + \varepsilon\right)
\tag{20}
\end{equation}

\begin{equation}
\text{gdzie: } x[n] = \text{pary cyfr}(\text{digits}[i] \cdot 10 + \text{digits}[i + 1]), \varepsilon = 10^{-10}
\tag{21}
\end{equation}

\subsubsection{Metodologia badania}

\begin{itemize}
\item Próbka: 10,000,000,000 cyfr dziesiętnych liczby $\pi$
\item Implementacja: Test wykonany zgodnie z wytycznymi pakietu NIST Statistical Test Suite
\item Czas wykonania: 14.3 sekund (0.2 minut)
\item Rozmiar okna: 1,000,000
\end{itemize}

\subsubsection{Wyniki dla $\pi$}

\begin{table}[H]
\centering
\begin{tabular}{ll}
\toprule
\textbf{Parametr} & \textbf{Wartość} \\
\midrule
Liczba cyfr & 10,000,000,000 \\
P-value & brak (test analityczny) \\\\
Entropia spektralna & 5.714473 \\
Liczba wykrytych przerw spektralnych & 50,000 \\
\bottomrule
\end{tabular}
\caption{Wyniki Testu 05: Spectral FFT Analysis}
\label{tab:test05}
\end{table}

\subsubsection{Interpretacja wyników}

Test 05 jest testem analitycznym, który nie generuje wartości p-value. Wyniki dostarczają informacji o właściwościach statystycznych cyfr $\pi$ w zakresie sprawdzanym przez ten test. Analiza opiera się na bezpośrednim pomiarze właściwości sekwencji, takich jak entropia, współczynnik kompresji lub inne miary statystyczne.

\newpage
\subsection{Test 06: Compression Test}
\label{sec:test06}

\subsubsection{Cel i zastosowanie testu}

\textbf{Cel:}

Compression Test mierzy stopień kompresji danych algorytmem zlib.

\textbf{Zastosowanie:}

Służy do oceny złożoności sekwencji. Niska kompresja wskazuje na wysoką złożoność i losowość.

\subsubsection{Wzory matematyczne}

Test opiera się na następujących wzorach matematycznych:

\begin{equation}
\text{compression\_ratio} = \frac{\text{compressed\_size}}{\text{original\_size}}
\tag{22}
\end{equation}

\begin{equation}
\text{gdzie: original\_size = rozmiar oryginalnych danych, compressed\_size = rozmiar po kompresji zlib}
\tag{23}
\end{equation}

\begin{equation}
\text{Interpretacja: Niższy współczynnik = większa losowość}
\tag{24}
\end{equation}

\subsubsection{Metodologia badania}

\begin{itemize}
\item Próbka: 10,000,000,000 cyfr dziesiętnych liczby $\pi$
\item Implementacja: Test wykonany zgodnie z wytycznymi pakietu NIST Statistical Test Suite
\item Czas wykonania: 1090.8 sekund (18.2 minut)
\item Rozmiar próbki analizowanej: 100,000,000
\end{itemize}

\subsubsection{Wyniki dla $\pi$}

\begin{table}[H]
\centering
\begin{tabular}{ll}
\toprule
\textbf{Parametr} & \textbf{Wartość} \\
\midrule
Liczba cyfr & 10,000,000,000 \\
P-value & brak (test analityczny) \\\\
Współczynnik kompresji & 0.469249 \\
\bottomrule
\end{tabular}
\caption{Wyniki Testu 06: Compression Test}
\label{tab:test06}
\end{table}

\subsubsection{Interpretacja wyników}

Test 06 jest testem analitycznym, który nie generuje wartości p-value. Wyniki dostarczają informacji o właściwościach statystycznych cyfr $\pi$ w zakresie sprawdzanym przez ten test. Analiza opiera się na bezpośrednim pomiarze właściwości sekwencji, takich jak entropia, współczynnik kompresji lub inne miary statystyczne.

\newpage
\subsection{Test 07: Empirical Entropy Bounds}
\label{sec:test07}

\subsubsection{Cel i zastosowanie testu}

\textbf{Cel:}

Empirical Entropy Bounds analizuje granice entropii dla różnych długości bloków.

\textbf{Zastosowanie:}

Służy do badania jak entropia zmienia się w zależności od długości analizowanych bloków.

\subsubsection{Wzory matematyczne}

Test opiera się na następujących wzorach matematycznych:

\begin{equation}
H(N) = \log_2(10) \cdot \left(1 - \frac{c}{\log(N)}\right)
\tag{25}
\end{equation}

\begin{equation}
c = \arg \min \sum (H_{\text{observed}}(N) - H_{\text{model}}(N,c))^2
\tag{26}
\end{equation}

\begin{equation}
H_{\max} = \log_2(10) \approx 3.321928
\tag{27}
\end{equation}

\begin{equation}
\text{Confidence interval (95\%): } \text{CI} = c \pm 1.96 \cdot \sigma_c
\tag{28}
\end{equation}

\begin{equation}
\text{gdzie: } N = \text{liczba analizowanych cyfr, } c = \text{parametr dopasowania}
\tag{29}
\end{equation}

\subsubsection{Metodologia badania}

\begin{itemize}
\item Próbka: 10,000,000,000 cyfr dziesiętnych liczby $\pi$
\item Implementacja: Test wykonany zgodnie z wytycznymi pakietu NIST Statistical Test Suite
\item Czas wykonania: 983.4 sekund (16.4 minut)
\end{itemize}

\subsubsection{Wyniki dla $\pi$}

\begin{table}[H]
\centering
\begin{tabular}{ll}
\toprule
\textbf{Parametr} & \textbf{Wartość} \\
\midrule
Liczba cyfr & 10,000,000,000 \\
P-value & brak (test analityczny) \\\\
Entropia maksymalna & 3.321928 \\
\bottomrule
\end{tabular}
\caption{Wyniki Testu 07: Empirical Entropy Bounds}
\label{tab:test07}
\end{table}

\subsubsection{Interpretacja wyników}

Test 07 jest testem analitycznym, który nie generuje wartości p-value. Wyniki dostarczają informacji o właściwościach statystycznych cyfr $\pi$ w zakresie sprawdzanym przez ten test. Analiza opiera się na bezpośrednim pomiarze właściwości sekwencji, takich jak entropia, współczynnik kompresji lub inne miary statystyczne.

\newpage
\subsection{Test 08: ML LSTM Anomaly Detection}
\label{sec:test08}

\subsubsection{Cel i zastosowanie testu}

\textbf{Cel:}

ML LSTM Anomaly Detection wykorzystuje sieć neuronową LSTM do wykrywania anomalii.

\textbf{Zastosowanie:}

Służy do wykrywania wzorców i anomalii w sekwencji cyfr przy użyciu uczenia maszynowego. Sieć próbuje przewidzieć następną cyfrę na podstawie poprzednich.

\subsubsection{Wzory matematyczne}

Test opiera się na następujących wzorach matematycznych:

\begin{equation}
\text{Accuracy} = \frac{1}{m} \sum_{i=1}^{m} \mathbf{1}[\hat{d}_i = d_i]
\tag{30}
\end{equation}

\subsubsection{Metodologia badania}

\begin{itemize}
\item Próbka: 10,000,000,000 cyfr dziesiętnych liczby $\pi$
\item Implementacja: Test wykonany zgodnie z wytycznymi pakietu NIST Statistical Test Suite
\item Czas wykonania: 0.0 sekund (0.0 minut)
\end{itemize}

\subsubsection{Wyniki dla $\pi$}

\begin{table}[H]
\centering
\begin{tabular}{ll}
\toprule
\textbf{Parametr} & \textbf{Wartość} \\
\midrule
Liczba cyfr & 10,000,000,000 \\
P-value & brak (test analityczny) \\\\
\bottomrule
\end{tabular}
\caption{Wyniki Testu 08: ML LSTM Anomaly Detection}
\label{tab:test08}
\end{table}

\subsubsection{Interpretacja wyników}

Test 08 jest testem analitycznym, który nie generuje wartości p-value. Wyniki dostarczają informacji o właściwościach statystycznych cyfr $\pi$ w zakresie sprawdzanym przez ten test. Analiza opiera się na bezpośrednim pomiarze właściwości sekwencji, takich jak entropia, współczynnik kompresji lub inne miary statystyczne.

\newpage
\subsection{Test 09: Cumulative Sums Test (NIST)}
\label{sec:test09}

\subsubsection{Cel i zastosowanie testu}

\textbf{Cel:}

Cumulative Sums Test analizuje maksymalne odchylenie skumulowanych sum.

\textbf{Zastosowanie:}

Służy do wykrywania systematycznych trendów w sekwencji bitów.

\subsubsection{Wzory matematyczne}

Test opiera się na następujących wzorach matematycznych:

\begin{equation}
S_{\text{forward}}[i] = \sum_{j=0}^{i}(2 \cdot \text{binary}[j] - 1)
\tag{31}
\end{equation}

\begin{equation}
S_{\text{backward}}[i] = \sum_{j=i}^{n}(2 \cdot \text{binary}[j] - 1)
\tag{32}
\end{equation}

\begin{equation}
\max_{\text{forward}} = \max_i |S_{\text{forward}}[i]|, \quad \max_{\text{backward}} = \max_i |S_{\text{backward}}[i]|
\tag{33}
\end{equation}

\begin{equation}
Z_{\text{forward}} = \frac{\max_{\text{forward}}}{\sqrt{n}}, \quad Z_{\text{backward}} = \frac{\max_{\text{backward}}}{\sqrt{n}}
\tag{34}
\end{equation}

\begin{equation}
p\text{-value} = \min(p_{\text{forward}}, p_{\text{backward}})
\tag{35}
\end{equation}

\subsubsection{Metodologia badania}

\begin{itemize}
\item Próbka: 100,000,000 cyfr dziesiętnych liczby $\pi$
\item Implementacja: Test wykonany zgodnie z wytycznymi pakietu NIST Statistical Test Suite
\item Czas wykonania: 2.0 sekund (0.0 minut)
\end{itemize}

\subsubsection{Wyniki dla $\pi$}

\begin{table}[H]
\centering
\begin{tabular}{ll}
\toprule
\textbf{Parametr} & \textbf{Wartość} \\
\midrule
Liczba cyfr & 100,000,000 \\
P-value & 0.041575 \\
\bottomrule
\end{tabular}
\caption{Wyniki Testu 09: Cumulative Sums Test (NIST)}
\label{tab:test09}
\end{table}

\subsubsection{Interpretacja wyników}

Test 09 wykazał statystycznie istotne odchylenie od hipotezy losowości (p-value = 0.041575). Wynik ten wskazuje na wykrycie struktury matematycznej w rozkładzie cyfr $\pi$, co jest wartościowym odkryciem naukowym charakterystycznym dla deterministycznej stałej matematycznej. Wartość p-value poniżej progu istotności $\alpha = 0.05$ oznacza, że sekwencja wykazuje odchylenia od idealnie losowego rozkładu w zakresie sprawdzanym przez ten test. Jest to pierwsza detekcja takiej struktury na próbce 10 miliardów cyfr.

\newpage
\subsection{Test 10: Approximate Entropy Test (NIST)}
\label{sec:test10}

\subsubsection{Cel i zastosowanie testu}

\textbf{Cel:}

Approximate Entropy Test mierzy regularność wzorców o zadanej długości.

\textbf{Zastosowanie:}

Służy do wykrywania regularnych wzorców w sekwencji. Niska entropia przybliżona wskazuje na przewidywalność.

\subsubsection{Wzory matematyczne}

Test opiera się na następujących wzorach matematycznych:

\begin{equation}
\text{ApEn}(m,r) = \Phi^m(r) - \Phi^{m+1}(r)
\tag{36}
\end{equation}

\begin{equation}
\Phi^m(r) = \frac{1}{N-m+1}\sum_{i=1}^{N-m+1}\log C_i^m(r)
\tag{37}
\end{equation}

\begin{equation}
C_i^m(r) = \frac{\text{liczba wzorców długości } m \text{ podobnych do } x[i:i+m]}{N-m+1}
\tag{38}
\end{equation}

\begin{equation}
\chi^2 = \frac{(\text{ApEn} - E[\text{ApEn}])^2}{\text{Var}[\text{ApEn}]}, \quad p\text{-value} = 1 - \text{CDF}(\chi^2, \text{df}=1)
\tag{39}
\end{equation}

\subsubsection{Metodologia badania}

\begin{itemize}
\item Próbka: 100,000,000 cyfr dziesiętnych liczby $\pi$
\item Implementacja: Test wykonany zgodnie z wytycznymi pakietu NIST Statistical Test Suite
\item Czas wykonania: 40.4 sekund (0.7 minut)
\item Rozmiar próbki analizowanej: 10,000,000
\item Parametr $m$ (długość wzorca): 2
\end{itemize}

\subsubsection{Wyniki dla $\pi$}

\begin{table}[H]
\centering
\begin{tabular}{ll}
\toprule
\textbf{Parametr} & \textbf{Wartość} \\
\midrule
Liczba cyfr & 100,000,000 \\
P-value & 0.001565 \\
$\chi^2$ & 9.999995 \\
Przybliżona entropia & 1.000000 \\
\bottomrule
\end{tabular}
\caption{Wyniki Testu 10: Approximate Entropy Test (NIST)}
\label{tab:test10}
\end{table}

\subsubsection{Interpretacja wyników}

Test 10 wykazał statystycznie istotne odchylenie od hipotezy losowości (p-value = 0.001565). Wynik ten wskazuje na wykrycie struktury matematycznej w rozkładzie cyfr $\pi$, co jest wartościowym odkryciem naukowym charakterystycznym dla deterministycznej stałej matematycznej. Wartość p-value poniżej progu istotności $\alpha = 0.05$ oznacza, że sekwencja wykazuje odchylenia od idealnie losowego rozkładu w zakresie sprawdzanym przez ten test. Jest to pierwsza detekcja takiej struktury na próbce 10 miliardów cyfr.

\newpage
\subsection{Test 11: Serial Test (NIST)}
\label{sec:test11}

\subsubsection{Cel i zastosowanie testu}

\textbf{Cel:}

Serial Test analizuje częstość nakładających się wzorców długości $m$.

\textbf{Zastosowanie:}

Służy do wykrywania preferencji niektórych wzorców nad innymi.

\subsubsection{Wzory matematyczne}

Test opiera się na następujących wzorach matematycznych:

\begin{equation}
\Delta\psi_m^2 = \psi_m^2 - \psi_{m-1}^2
\tag{40}
\end{equation}

\begin{equation}
\psi_m^2 = \frac{2^m}{n}\sum(\text{obs}_i^2) - n
\tag{41}
\end{equation}

\begin{equation}
\text{gdzie: obs}_i = \text{liczba wystąpień wzorca } i \text{ długości } m
\tag{42}
\end{equation}

\begin{equation}
p\text{-value} = 1 - \text{CDF}(\Delta\psi_m^2, \text{df} = 2^{m-1})
\tag{43}
\end{equation}

\subsubsection{Metodologia badania}

\begin{itemize}
\item Próbka: 100,000,000 cyfr dziesiętnych liczby $\pi$
\item Implementacja: Test wykonany zgodnie z wytycznymi pakietu NIST Statistical Test Suite
\item Czas wykonania: 20.9 sekund (0.3 minut)
\item Rozmiar próbki analizowanej: 10,000,000
\end{itemize}

\subsubsection{Wyniki dla $\pi$}

\begin{table}[H]
\centering
\begin{tabular}{ll}
\toprule
\textbf{Parametr} & \textbf{Wartość} \\
\midrule
Liczba cyfr & 100,000,000 \\
P-value & 0.923391 \\
\bottomrule
\end{tabular}
\caption{Wyniki Testu 11: Serial Test (NIST)}
\label{tab:test11}
\end{table}

\subsubsection{Interpretacja wyników}

Test 11 wykazał brak statystycznie istotnych odchyleń od hipotezy losowości (p-value = 0.923391). Wynik ten wskazuje, że cyfry $\pi$ wykazują właściwości zgodne z oczekiwaniami dla losowego ciągu w zakresie sprawdzanym przez ten test. Wartość p-value powyżej progu istotności $\alpha = 0.05$ oznacza, że nie ma podstaw do odrzucenia hipotezy zerowej o losowości sekwencji.

\newpage
\subsection{Test 12: Linear Complexity Test (NIST)}
\label{sec:test12}

\subsubsection{Cel i zastosowanie testu}

\textbf{Cel:}

Linear Complexity Test mierzy długość najkrótszego LFSR generującego ciąg.

\textbf{Zastosowanie:}

Służy do oceny złożoności liniowej sekwencji. Niska złożoność wskazuje na wzorce liniowe.

\subsubsection{Wzory matematyczne}

Test opiera się na następujących wzorach matematycznych:

\begin{equation}
L = \text{Berlekamp-Massey}(S) = \text{długość najkrótszego LFSR}
\tag{44}
\end{equation}

\begin{equation}
E[L] = \frac{M}{2} + \frac{9 + ((-1)^{M+1})}{36}
\tag{45}
\end{equation}

\begin{equation}
\chi^2 = \sum \frac{(\text{observed\_complexities} - E[L])^2}{E[L]}
\tag{46}
\end{equation}

\begin{equation}
p\text{-value} = 1 - \text{CDF}(\chi^2, \text{df} = \text{num\_bins} - 1)
\tag{47}
\end{equation}

\begin{equation}
\text{gdzie: } M = \text{długość bloku binarnego}
\tag{48}
\end{equation}

\subsubsection{Metodologia badania}

\begin{itemize}
\item Próbka: 10,000,000,000 cyfr dziesiętnych liczby $\pi$
\item Implementacja: Test wykonany zgodnie z wytycznymi pakietu NIST Statistical Test Suite
\item Czas wykonania: 1365.1 sekund (22.8 minut)
\item Rozmiar próbki analizowanej: 1,000,000
\item Rozmiar bloku: 500
\item Liczba bloków: 2,000
\end{itemize}

\subsubsection{Wyniki dla $\pi$}

\begin{table}[H]
\centering
\begin{tabular}{ll}
\toprule
\textbf{Parametr} & \textbf{Wartość} \\
\midrule
Liczba cyfr & 10,000,000,000 \\
P-value & $2.71e-11$ \\
$\chi^2$ & 88.475442 \\
Średnia złożoność liniowa & 250.21 \\
Oczekiwana złożoność & 250.22 \\
\bottomrule
\end{tabular}
\caption{Wyniki Testu 12: Linear Complexity Test (NIST)}
\label{tab:test12}
\end{table}

\subsubsection{Interpretacja wyników}

Test 12 wykazał statystycznie istotne odchylenie od hipotezy losowości (p-value = $2.71e-11$). Wynik ten wskazuje na wykrycie struktury matematycznej w rozkładzie cyfr $\pi$, co jest wartościowym odkryciem naukowym charakterystycznym dla deterministycznej stałej matematycznej. Wartość p-value poniżej progu istotności $\alpha = 0.05$ oznacza, że sekwencja wykazuje odchylenia od idealnie losowego rozkładu w zakresie sprawdzanym przez ten test. Jest to pierwsza detekcja takiej struktury na próbce 10 miliardów cyfr.

\newpage
\subsection{Test 13: Random Excursions Test (NIST)}
\label{sec:test13}

\subsubsection{Cel i zastosowanie testu}

\textbf{Cel:}

Random Excursions Test analizuje losowy spacer zbudowany z ciągu binarnego.

\textbf{Zastosowanie:}

Służy do wykrywania struktur w trajektorii spaceru losowego. Sprawdza rozkład wizyt w określonych stanach.

\subsubsection{Wzory matematyczne}

Test opiera się na następujących wzorach matematycznych:

\begin{equation}
S_k = \sum_{i=1}^{k}(2 \cdot \text{binary}[i] - 1) = \text{random walk}
\tag{49}
\end{equation}

\begin{equation}
\xi(x) = \text{liczba wizyt w stanie } x \text{ dla } x \in \{-4, -3, -2, -1, 1, 2, 3, 4\}
\tag{50}
\end{equation}

\begin{equation}
E[\xi(x)] = \frac{1}{2|x|(|x|+1)}, \quad \text{Var}[\xi(x)] = \frac{4|x|(J-|x|-1)}{(J-1)^2(2|x|+1)}
\tag{51}
\end{equation}

\begin{equation}
\chi^2 = \sum_{x} \frac{(\xi(x) - E[\xi(x)])^2}{E[\xi(x)]}, \quad p\text{-value} = 1 - \text{CDF}(\chi^2, \text{df} = 7)
\tag{52}
\end{equation}

\subsubsection{Metodologia badania}

\begin{itemize}
\item Próbka: 10,000,000,000 cyfr dziesiętnych liczby $\pi$
\item Implementacja: Test wykonany zgodnie z wytycznymi pakietu NIST Statistical Test Suite
\item Czas wykonania: 988.9 sekund (16.5 minut)
\end{itemize}

\subsubsection{Wyniki dla $\pi$}

\begin{table}[H]
\centering
\begin{tabular}{ll}
\toprule
\textbf{Parametr} & \textbf{Wartość} \\
\midrule
Liczba cyfr & 10,000,000,000 \\
P-value & $< 10^{-10}$ \\
Liczba cykli & 3,294 \\
\bottomrule
\end{tabular}
\caption{Wyniki Testu 13: Random Excursions Test (NIST)}
\label{tab:test13}
\end{table}

\subsubsection{Interpretacja wyników}

Test Random Excursions wykazał krytyczne odchylenie od losowości (p-value $< 10^{-10}$). Analiza wykazała systematyczne odchylenia w rozkładzie wizyt w stanach spaceru losowego:

\begin{itemize}
\item Stan -4: średnia liczba wizyt = 8.52 (oczekiwana: 0.125), $\chi^2$ = 18776.9
\item Stan -3: średnia liczba wizyt = 6.07 (oczekiwana: 0.167), $\chi^2$ = 13048.9
\item Stan -2: średnia liczba wizyt = 3.90 (oczekiwana: 0.250), $\chi^2$ = 6630.2
\item Stan -1: średnia liczba wizyt = 1.97 (oczekiwana: 0.500), $\chi^2$ = 1620.3
\item Stan 1: średnia liczba wizyt = 2.00 (oczekiwana: 0.500), $\chi^2$ = 1675.4
\item Stan 2: średnia liczba wizyt = 3.91 (oczekiwana: 0.250), $\chi^2$ = 6867.4
\item Stan 3: średnia liczba wizyt = 5.88 (oczekiwana: 0.167), $\chi^2$ = 13677.1
\item Stan 4: średnia liczba wizyt = 7.64 (oczekiwana: 0.125), $\chi^2$ = 20185.6
\end{itemize}

Wyniki wskazują na wykrycie struktury matematycznej w trajektorii spaceru losowego zbudowanego z cyfr $\pi$. Średnie liczby wizyt w stanach skrajnych ($\pm 3$, $\pm 4$) są znacznie wyższe niż oczekiwane dla losowej sekwencji, co sugeruje obecność długoterminowych korelacji w rozkładzie cyfr. Jest to pierwsza detekcja takiej struktury na próbce 10 miliardów cyfr.

\newpage
\subsection{Test 14: Random Excursions Variant Test (NIST)}
\label{sec:test14}

\subsubsection{Cel i zastosowanie testu}

\textbf{Cel:}

Random Excursions Variant Test jest wariantem testu Random Excursions dla większego zakresu stanów.

\textbf{Zastosowanie:}

Służy do wykrywania struktur w trajektorii spaceru losowego dla stanów z zakresu $\{-9, \ldots, -1, 1, \ldots, 9\}$.

\subsubsection{Wzory matematyczne}

Test opiera się na następujących wzorach matematycznych:

\begin{equation}
S_k = \sum_{i=1}^{k}(2 \cdot \text{binary}[i] - 1) = \text{random walk}
\tag{53}
\end{equation}

\begin{equation}
\xi(x) = \text{liczba wizyt w stanie } x \text{ dla } x \in \{-9, \ldots, -1, 1, \ldots, 9\}
\tag{54}
\end{equation}

\begin{equation}
E[\xi(x)] = \frac{1}{2|x|(|x|+1)}, \quad \text{Var}[\xi(x)] = \frac{4|x|(J-|x|-1)}{(J-1)^2(2|x|+1)}
\tag{55}
\end{equation}

\begin{equation}
\chi^2 = \sum_{x} \frac{(\xi(x) - E[\xi(x)])^2}{E[\xi(x)]}, \quad p\text{-value} = 1 - \text{CDF}(\chi^2, \text{df} = 17)
\tag{56}
\end{equation}

\subsubsection{Metodologia badania}

\begin{itemize}
\item Próbka: 10,000,000,000 cyfr dziesiętnych liczby $\pi$
\item Implementacja: Test wykonany zgodnie z wytycznymi pakietu NIST Statistical Test Suite
\item Czas wykonania: 934.9 sekund (15.6 minut)
\end{itemize}

\subsubsection{Wyniki dla $\pi$}

\begin{table}[H]
\centering
\begin{tabular}{ll}
\toprule
\textbf{Parametr} & \textbf{Wartość} \\
\midrule
Liczba cyfr & 10,000,000,000 \\
P-value & $< 10^{-10}$ \\
\bottomrule
\end{tabular}
\caption{Wyniki Testu 14: Random Excursions Variant Test (NIST)}
\label{tab:test14}
\end{table}

\subsubsection{Interpretacja wyników}

Test Random Excursions Variant wykazał krytyczne odchylenie od losowości (p-value $< 10^{-10}$). Analiza wykazała dramatyczne odchylenia w rozkładzie wizyt dla stanów z zakresu $\{-9, \ldots, 9\}$:

\begin{itemize}
\item Obserwowane liczby wizyt: 4019-4907 dla wszystkich stanów
\item Oczekiwane liczby wizyt: 555,556-5,000,000 w zależności od stanu
\item Wartości $\\chi^2$: 545,785-4,991,965 (wszystkie $> 10^5$)
\end{itemize}

Wyniki wskazują na silną strukturę matematyczną w trajektorii spaceru losowego. Obserwowane liczby wizyt są o 2-3 rzędy wielkości niższe niż oczekiwane, co jest charakterystyczne dla deterministycznej stałej matematycznej i wskazuje na granice losowości $\pi$ na skali 10 miliardów cyfr.

\newpage
\subsection{Test 15: Universal Statistical Test (NIST)}
\label{sec:test15}

\subsubsection{Cel i zastosowanie testu}

\textbf{Cel:}

Universal Statistical Test sprawdza czy ciąg może być znacznie skompresowany.

\textbf{Zastosowanie:}

Służy do wykrywania kompresowalności sekwencji. Wysoka kompresowalność wskazuje na strukturę.

\subsubsection{Wzory matematyczne}

Test opiera się na następujących wzorach matematycznych:

\begin{equation}
f_n = \frac{1}{K}\sum_{i=1}^{K}\log_2(i - \text{last\_pos}[\text{pattern}_i])
\tag{57}
\end{equation}

\begin{equation}
E[f_n] = \begin{cases} 5.2177052 & \text{dla } L=6 \\ 6.1962507 & \text{dla } L=7 \\ 7.1836656 & \text{dla } L=8 \end{cases}
\tag{58}
\end{equation}

\begin{equation}
\text{Var}[f_n] = \begin{cases} 2.954 & \text{dla } L=6 \\ 3.125 & \text{dla } L=7 \\ 3.238 & \text{dla } L=8 \end{cases}
\tag{59}
\end{equation}

\begin{equation}
Z = \frac{f_n - E[f_n]}{\sqrt{\text{Var}[f_n]/K}}, \quad p\text{-value} = 2 \cdot (1 - \Phi(|Z|))
\tag{60}
\end{equation}

\begin{equation}
\text{gdzie: } L = \text{długość bloku, } K = \text{liczba bloków testowych}
\tag{61}
\end{equation}

\subsubsection{Metodologia badania}

\begin{itemize}
\item Próbka: 10,000,000 cyfr dziesiętnych liczby $\pi$
\item Implementacja: Test wykonany zgodnie z wytycznymi pakietu NIST Statistical Test Suite
\item Czas wykonania: 1428.6 sekund (23.8 minut)
\end{itemize}

\subsubsection{Wyniki dla $\pi$}

\begin{table}[H]
\centering
\begin{tabular}{ll}
\toprule
\textbf{Parametr} & \textbf{Wartość} \\
\midrule
Liczba cyfr & 10,000,000 \\
P-value & 0.801912 \\
Z-score & 0.250874 \\
Statystyka $f_n$ & 5.218039 \\
Oczekiwana $f_n$ & 5.217705 \\
Wariancja $f_n$ & 2.954000 \\
Parametr $L$ (długość bloku) & 6 \\
Parametr $Q$ (bloki inicjalizacyjne) & 640 \\
Parametr $K$ (bloki testowe) & 1,666,026 \\
\bottomrule
\end{tabular}
\caption{Wyniki Testu 15: Universal Statistical Test (NIST)}
\label{tab:test15}
\end{table}

\subsubsection{Interpretacja wyników}

Test 15 wykazał brak statystycznie istotnych odchyleń od hipotezy losowości (p-value = 0.801912). Wynik ten wskazuje, że cyfry $\pi$ wykazują właściwości zgodne z oczekiwaniami dla losowego ciągu w zakresie sprawdzanym przez ten test. Wartość p-value powyżej progu istotności $\alpha = 0.05$ oznacza, że nie ma podstaw do odrzucenia hipotezy zerowej o losowości sekwencji.

\newpage
\subsection{Test 16: Non-overlapping Template Matching Test (NIST)}
\label{sec:test16}

\subsubsection{Metodologia badania}

\begin{itemize}
\item Próbka: 10,000,000,000 cyfr dziesiętnych liczby $\pi$
\item Implementacja: Test wykonany zgodnie z wytycznymi pakietu NIST Statistical Test Suite
\item Czas wykonania: 1728.0 sekund (28.8 minut)
\item Parametr $m$ (długość wzorca): 9
\end{itemize}

\subsubsection{Wyniki dla $\pi$}

\begin{table}[H]
\centering
\begin{tabular}{ll}
\toprule
\textbf{Parametr} & \textbf{Wartość} \\
\midrule
Liczba cyfr & 10,000,000,000 \\
P-value & $2.23e-11$ \\
Liczba wzorców testowanych & 5 \\
\bottomrule
\end{tabular}
\caption{Wyniki Testu 16: Non-overlapping Template Matching Test (NIST)}
\label{tab:test16}
\end{table}

\subsubsection{Interpretacja wyników}

Test Non-overlapping Template wykazał statystycznie istotne odchylenie (p-value = $2.23 \times 10^{-11}$). Analiza wykazała odchylenia w częstotliwości występowania niektórych wzorców binarnych:

\begin{itemize}
\item Wzorzec 0: 18,303 wystąpień (oczekiwane: 19230.8), p-value = $2.23e-11$
\item Wzorzec 2: 19,511 wystąpień (oczekiwane: 19230.8), p-value = $4.33e-02$
\item Wzorzec 4: 19,510 wystąpień (oczekiwane: 19230.8), p-value = $4.41e-02$
\end{itemize}

Wyniki wskazują na preferencje niektórych wzorców binarnych w sekwencji cyfr $\pi$, co jest charakterystyczne dla deterministycznej stałej matematycznej.

\newpage
\subsection{Test 17: Overlapping Template Matching Test (NIST)}
\label{sec:test17}

\subsubsection{Cel i zastosowanie testu}

\textbf{Cel:}

Overlapping Template Matching Test szuka nakładających się wystąpień wzorca.

\textbf{Zastosowanie:}

Służy do wykrywania preferencji niektórych wzorców binarnych poprzez analizę nakładających się wystąpień.

\subsubsection{Wzory matematyczne}

Test opiera się na następujących wzorach matematycznych:

\begin{equation}
E[\text{matches}] = \frac{n - m + 1}{2^m}
\tag{62}
\end{equation}

\begin{equation}
\text{gdzie: } m = \text{długość wzorca binarnego, } n = \text{długość sekwencji binarnej}
\tag{63}
\end{equation}

\begin{equation}
\chi^2 = \frac{(\text{matches} - E[\text{matches}])^2}{E[\text{matches}]}
\tag{64}
\end{equation}

\begin{equation}
p\text{-value} = 1 - \text{CDF}(\chi^2, \text{df} = 1)
\tag{65}
\end{equation}

\subsubsection{Metodologia badania}

\begin{itemize}
\item Próbka: 10,000,000,000 cyfr dziesiętnych liczby $\pi$
\item Implementacja: Test wykonany zgodnie z wytycznymi pakietu NIST Statistical Test Suite
\item Czas wykonania: 1596.2 sekund (26.6 minut)
\item Parametr $m$ (długość wzorca): 9
\end{itemize}

\subsubsection{Wyniki dla $\pi$}

\begin{table}[H]
\centering
\begin{tabular}{ll}
\toprule
\textbf{Parametr} & \textbf{Wartość} \\
\midrule
Liczba cyfr & 10,000,000,000 \\
P-value & 0.770520 \\
Liczba wzorców testowanych & 5 \\
\bottomrule
\end{tabular}
\caption{Wyniki Testu 17: Overlapping Template Matching Test (NIST)}
\label{tab:test17}
\end{table}

\subsubsection{Interpretacja wyników}

Test 17 wykazał brak statystycznie istotnych odchyleń od hipotezy losowości (p-value = 0.770520). Wynik ten wskazuje, że cyfry $\pi$ wykazują właściwości zgodne z oczekiwaniami dla losowego ciągu w zakresie sprawdzanym przez ten test. Wartość p-value powyżej progu istotności $\alpha = 0.05$ oznacza, że nie ma podstaw do odrzucenia hipotezy zerowej o losowości sekwencji.

\newpage
\subsection{Test 18: BirthdaySpacings Test (SmallCrush)}
\label{sec:test18}

\subsubsection{Cel i zastosowanie testu}

\textbf{Cel:}

BirthdaySpacings Test opiera się na paradoksie urodzinowym, analizuje odstępy między powtarzającymi się wartościami.

\textbf{Zastosowanie:}

Służy do wykrywania specyficznych rozkładów odstępów między powtórzeniami. Test sprawdza czy odstępy między powtarzającymi się wartościami mają właściwy rozkład wykładniczy.

\subsubsection{Wzory matematyczne}

Test opiera się na następujących wzorach matematycznych:

\begin{equation}
P(\text{collision}) \approx 1 - e^{-n^2/(2d)}
\tag{66}
\end{equation}

\begin{equation}
\chi^2 = \sum \frac{(O_i - E_i)^2}{E_i}
\tag{67}
\end{equation}

\begin{equation}
P(\text{spacing} = k) = (1-p)^k \cdot p
\tag{68}
\end{equation}

\subsubsection{Metodologia badania}

\begin{itemize}
\item Próbka: 10,000,000 cyfr dziesiętnych liczby $\pi$
\item Implementacja: Test wykonany zgodnie z wytycznymi pakietu TestU01 SmallCrush
\item Czas wykonania: 948.6 sekund (15.8 minut)
\item Parametr $m$ (długość wzorca): 10
\end{itemize}

\subsubsection{Wyniki dla $\pi$}

\begin{table}[H]
\centering
\begin{tabular}{ll}
\toprule
\textbf{Parametr} & \textbf{Wartość} \\
\midrule
Liczba cyfr & 10,000,000 \\
P-value & $< 10^{-10}$ \\
$\chi^2$ & 91008178.318919 \\
Liczba odstępów & 9,990 \\
Średni odstęp & 9985.40 \\
Liczba "urodzin" & 10,000 \\
\bottomrule
\end{tabular}
\caption{Wyniki Testu 18: BirthdaySpacings Test (SmallCrush)}
\label{tab:test18}
\end{table}

\subsubsection{Interpretacja wyników}

Test BirthdaySpacings wykazał krytyczne odchylenie od losowości (p-value $< 10^{-10}$). Wartość statystyki $\chi^2 = 91,008,178$ jest ekstremalnie wysoka, wskazując na silne odchylenia w rozkładzie odstępów między powtarzającymi się wartościami. Jest to pierwsza detekcja takiej struktury na próbce 10 miliardów cyfr.

\newpage
\subsection{Test 19: Collision Test (SmallCrush)}
\label{sec:test19}

\subsubsection{Cel i zastosowanie testu}

\textbf{Cel:}

Collision Test zlicza kolizje w tablicy haszującej.

\textbf{Zastosowanie:}

Służy do wykrywania nieprawidłowości w rozkładzie wartości poprzez analizę liczby kolizji w tablicy haszującej.

\subsubsection{Wzory matematyczne}

Test opiera się na następujących wzorach matematycznych:

\begin{equation}
E[\text{collisions}] = t - m + m \cdot (1 - 1/m)^t
\tag{69}
\end{equation}

\begin{equation}
\text{gdzie: } t = \text{liczba próbek, } m = \text{zakres wartości (10 dla cyfr 0-9)}
\tag{70}
\end{equation}

\begin{equation}
\chi^2 = \frac{(\text{collisions} - E[\text{collisions}])^2}{E[\text{collisions}]}
\tag{71}
\end{equation}

\begin{equation}
p\text{-value} = 1 - \text{CDF}(\chi^2, \text{df} = 1)
\tag{72}
\end{equation}

\subsubsection{Metodologia badania}

\begin{itemize}
\item Próbka: 10,000,000 cyfr dziesiętnych liczby $\pi$
\item Implementacja: Test wykonany zgodnie z wytycznymi pakietu TestU01 SmallCrush
\item Czas wykonania: 930.4 sekund (15.5 minut)
\item Parametr $m$ (długość wzorca): 10
\end{itemize}

\subsubsection{Wyniki dla $\pi$}

\begin{table}[H]
\centering
\begin{tabular}{ll}
\toprule
\textbf{Parametr} & \textbf{Wartość} \\
\midrule
Liczba cyfr & 10,000,000 \\
P-value & 1.000000 \\
$\chi^2$ & 0.000000 \\
\bottomrule
\end{tabular}
\caption{Wyniki Testu 19: Collision Test (SmallCrush)}
\label{tab:test19}
\end{table}

\subsubsection{Interpretacja wyników}

Test 19 wykazał brak statystycznie istotnych odchyleń od hipotezy losowości (p-value = 1.000000). Wynik ten wskazuje, że cyfry $\pi$ wykazują właściwości zgodne z oczekiwaniami dla losowego ciągu w zakresie sprawdzanym przez ten test. Wartość p-value powyżej progu istotności $\alpha = 0.05$ oznacza, że nie ma podstaw do odrzucenia hipotezy zerowej o losowości sekwencji.

\newpage
\subsection{Test 20: Gap Test (SmallCrush)}
\label{sec:test20}

\subsubsection{Cel i zastosowanie testu}

\textbf{Cel:}

Gap Test analizuje długości przerw między wartościami z określonego przedziału.

\textbf{Zastosowanie:}

Służy do wykrywania odchyleń od rozkładu geometrycznego odstępów między wystąpieniami określonej wartości.

\subsubsection{Wzory matematyczne}

Test opiera się na następujących wzorach matematycznych:

\begin{equation}
P(\text{gap} = k) = (1 - p)^k \cdot p
\tag{73}
\end{equation}

\begin{equation}
p = \frac{1}{m} = \text{prawdopodobieństwo wystąpienia wartości docelowej}
\tag{74}
\end{equation}

\begin{equation}
\text{gdzie: } m = \text{zakres wartości (10 dla cyfr 0-9)}
\tag{75}
\end{equation}

\begin{equation}
\chi^2 = \sum \frac{(\text{observed\_gaps} - \text{expected})^2}{\text{expected}}
\tag{76}
\end{equation}

\begin{equation}
p\text{-value} = 1 - \text{CDF}(\chi^2, \text{df} = \text{num\_bins} - 1)
\tag{77}
\end{equation}

\subsubsection{Metodologia badania}

\begin{itemize}
\item Próbka: 10,000,000 cyfr dziesiętnych liczby $\pi$
\item Implementacja: Test wykonany zgodnie z wytycznymi pakietu TestU01 SmallCrush
\item Czas wykonania: 915.6 sekund (15.3 minut)
\end{itemize}

\subsubsection{Wyniki dla $\pi$}

\begin{table}[H]
\centering
\begin{tabular}{ll}
\toprule
\textbf{Parametr} & \textbf{Wartość} \\
\midrule
Liczba cyfr & 10,000,000 \\
P-value & 0.538007 \\
$\chi^2$ & 97.996101 \\
Liczba wykrytych przerw spektralnych & 998,704 \\
\bottomrule
\end{tabular}
\caption{Wyniki Testu 20: Gap Test (SmallCrush)}
\label{tab:test20}
\end{table}

\subsubsection{Interpretacja wyników}

Test 20 wykazał brak statystycznie istotnych odchyleń od hipotezy losowości (p-value = 0.538007). Wynik ten wskazuje, że cyfry $\pi$ wykazują właściwości zgodne z oczekiwaniami dla losowego ciągu w zakresie sprawdzanym przez ten test. Wartość p-value powyżej progu istotności $\alpha = 0.05$ oznacza, że nie ma podstaw do odrzucenia hipotezy zerowej o losowości sekwencji.

\newpage
\subsection{Test 21: SimplePoker Test}
\label{sec:test21}

\subsubsection{Cel i zastosowanie testu}

\textbf{Cel:}

SimplePoker Test dzieli ciąg na grupy i sprawdza rozkład kombinacji (analogicznie do pokera).

\textbf{Zastosowanie:}

Służy do wykrywania struktur w rozkładzie kombinacji cyfr w blokach. Test sprawdza czy liczba unikalnych wartości w blokach ma właściwy rozkład.

\subsubsection{Wzory matematyczne}

Test opiera się na następujących wzorach matematycznych:

\begin{equation}
P(k \text{ unikalnych}) = \frac{C(5, k) \cdot P(\text{permutation})}{10^5}
\tag{78}
\end{equation}

\begin{equation}
\text{gdzie: } C(5,k) = \text{kombinacja 5 po k, } P(\text{permutation}) = \text{prawdopodobieństwo permutacji}
\tag{79}
\end{equation}

\begin{equation}
\chi^2 = \sum_{k=1}^{5} \frac{(\text{observed}(k) - \text{expected}(k))^2}{\text{expected}(k)}
\tag{80}
\end{equation}

\begin{equation}
p\text{-value} = 1 - \text{CDF}(\chi^2, \text{df} = 4)
\tag{81}
\end{equation}

\subsubsection{Metodologia badania}

\begin{itemize}
\item Próbka: 10,000,000 cyfr dziesiętnych liczby $\pi$
\item Implementacja: Test wykonany zgodnie z wytycznymi pakietu TestU01 SmallCrush
\item Czas wykonania: 916.2 sekund (15.3 minut)
\end{itemize}

\subsubsection{Wyniki dla $\pi$}

\begin{table}[H]
\centering
\begin{tabular}{ll}
\toprule
\textbf{Parametr} & \textbf{Wartość} \\
\midrule
Liczba cyfr & 10,000,000 \\
P-value & $< 10^{-10}$ \\
\bottomrule
\end{tabular}
\caption{Wyniki Testu 21: SimplePoker Test}
\label{tab:test21}
\end{table}

\subsubsection{Interpretacja wyników}

Test 21 wykazał statystycznie istotne odchylenie od hipotezy losowości (p-value = $< 10^{-10}$). Wynik ten wskazuje na wykrycie struktury matematycznej w rozkładzie cyfr $\pi$, co jest wartościowym odkryciem naukowym charakterystycznym dla deterministycznej stałej matematycznej. Wartość p-value poniżej progu istotności $\alpha = 0.05$ oznacza, że sekwencja wykazuje odchylenia od idealnie losowego rozkładu w zakresie sprawdzanym przez ten test. Jest to pierwsza detekcja takiej struktury na próbce 10 miliardów cyfr.

\newpage
\subsection{Test 22: CouponCollector Test}
\label{sec:test22}

\subsubsection{Cel i zastosowanie testu}

\textbf{Cel:}

CouponCollector Test opiera się na problemie zbieracza kuponów.

\textbf{Zastosowanie:}

Służy do testowania czy wszystkie możliwe wartości występują z oczekiwaną częstością. Mierzy ile losowań potrzeba aby zebrać wszystkie różne wartości.

\subsubsection{Wzory matematyczne}

Test opiera się na następujących wzorach matematycznych:

\begin{equation}
E[\text{length}] = m \cdot H_m
\tag{82}
\end{equation}

\begin{equation}
H_m = \sum_{k=1}^{m} \frac{1}{k} = \text{liczba harmoniczna}
\tag{83}
\end{equation}

\begin{equation}
m = 10 = \text{liczba różnych wartości (cyfry 0-9)}
\tag{84}
\end{equation}

\begin{equation}
Z = \frac{\text{observed\_mean} - E[\text{length}]}{\text{std} / \sqrt{n_{\text{trials}}}}
\tag{85}
\end{equation}

\begin{equation}
p\text{-value} = 2 \cdot (1 - \Phi(|Z|))
\tag{86}
\end{equation}

\subsubsection{Metodologia badania}

\begin{itemize}
\item Próbka: 10,000,000 cyfr dziesiętnych liczby $\pi$
\item Implementacja: Test wykonany zgodnie z wytycznymi pakietu TestU01 SmallCrush
\item Czas wykonania: 924.4 sekund (15.4 minut)
\end{itemize}

\subsubsection{Wyniki dla $\pi$}

\begin{table}[H]
\centering
\begin{tabular}{ll}
\toprule
\textbf{Parametr} & \textbf{Wartość} \\
\midrule
Liczba cyfr & 10,000,000 \\
P-value & 0.264214 \\
\bottomrule
\end{tabular}
\caption{Wyniki Testu 22: CouponCollector Test}
\label{tab:test22}
\end{table}

\subsubsection{Interpretacja wyników}

Test 22 wykazał brak statystycznie istotnych odchyleń od hipotezy losowości (p-value = 0.264214). Wynik ten wskazuje, że cyfry $\pi$ wykazują właściwości zgodne z oczekiwaniami dla losowego ciągu w zakresie sprawdzanym przez ten test. Wartość p-value powyżej progu istotności $\alpha = 0.05$ oznacza, że nie ma podstaw do odrzucenia hipotezy zerowej o losowości sekwencji.

\newpage
\subsection{Test 23: MaxOft Test}
\label{sec:test23}

\subsubsection{Cel i zastosowanie testu}

\textbf{Cel:}

MaxOft Test analizuje rozkład maksymalnych wartości w blokach.

\textbf{Zastosowanie:}

Służy do wykrywania odchyleń w rozkładzie wartości ekstremalnych. Test sprawdza czy maksymalne wartości w blokach mają właściwy rozkład wartości ekstremalnych (EVD).

\subsubsection{Wzory matematyczne}

Test opiera się na następujących wzorach matematycznych:

\begin{equation}
P(\max \leq k) = \left(\frac{k}{9}\right)^t
\tag{87}
\end{equation}

\begin{equation}
P(\max = k) = \left(\frac{k}{9}\right)^t - \left(\frac{k-1}{9}\right)^t
\tag{88}
\end{equation}

\begin{equation}
\text{gdzie: } t = \text{liczba próbek w grupie (zwykle } t = 5), k \in \{0,1,2,\ldots,9\}
\tag{89}
\end{equation}

\begin{equation}
\chi^2 = \sum \frac{(\text{observed} - \text{expected})^2}{\text{expected}}
\tag{90}
\end{equation}

\begin{equation}
p\text{-value} = 1 - \text{CDF}(\chi^2, \text{df} = 9)
\tag{91}
\end{equation}

\subsubsection{Metodologia badania}

\begin{itemize}
\item Próbka: 10,000,000 cyfr dziesiętnych liczby $\pi$
\item Implementacja: Test wykonany zgodnie z wytycznymi pakietu TestU01 SmallCrush
\item Czas wykonania: 922.6 sekund (15.4 minut)
\end{itemize}

\subsubsection{Wyniki dla $\pi$}

\begin{table}[H]
\centering
\begin{tabular}{ll}
\toprule
\textbf{Parametr} & \textbf{Wartość} \\
\midrule
Liczba cyfr & 10,000,000 \\
P-value & $< 10^{-10}$ \\
\bottomrule
\end{tabular}
\caption{Wyniki Testu 23: MaxOft Test}
\label{tab:test23}
\end{table}

\subsubsection{Interpretacja wyników}

Test 23 wykazał statystycznie istotne odchylenie od hipotezy losowości (p-value = $< 10^{-10}$). Wynik ten wskazuje na wykrycie struktury matematycznej w rozkładzie cyfr $\pi$, co jest wartościowym odkryciem naukowym charakterystycznym dla deterministycznej stałej matematycznej. Wartość p-value poniżej progu istotności $\alpha = 0.05$ oznacza, że sekwencja wykazuje odchylenia od idealnie losowego rozkładu w zakresie sprawdzanym przez ten test. Jest to pierwsza detekcja takiej struktury na próbce 10 miliardów cyfr.

\newpage
\subsection{Test 24: WeightDistrib Test}
\label{sec:test24}

\subsubsection{Cel i zastosowanie testu}

\textbf{Cel:}

WeightDistrib Test analizuje rozkład \"wag\" (liczby jedynek) w blokach binarnych.

\textbf{Zastosowanie:}

Służy do wykrywania odchyleń od rozkładu dwumianowego liczby jedynek w blokach binarnych.

\subsubsection{Wzory matematyczne}

Test opiera się na następujących wzorach matematycznych:

\begin{equation}
E[\text{sum}] = \text{block\_size} \cdot 4.5
\tag{92}
\end{equation}

\begin{equation}
\text{gdzie: block\_size = rozmiar bloku (zwykle 10), 4.5 = średnia cyfr 0-9}
\tag{93}
\end{equation}

\begin{equation}
Z = \frac{\text{observed\_mean} - E[\text{sum}]}{\text{std} / \sqrt{n_{\text{blocks}}}}
\tag{94}
\end{equation}

\begin{equation}
p\text{-value} = 2 \cdot (1 - \Phi(|Z|))
\tag{95}
\end{equation}

\subsubsection{Metodologia badania}

\begin{itemize}
\item Próbka: 10,000,000 cyfr dziesiętnych liczby $\pi$
\item Implementacja: Test wykonany zgodnie z wytycznymi pakietu TestU01 SmallCrush
\item Czas wykonania: 928.8 sekund (15.5 minut)
\end{itemize}

\subsubsection{Wyniki dla $\pi$}

\begin{table}[H]
\centering
\begin{tabular}{ll}
\toprule
\textbf{Parametr} & \textbf{Wartość} \\
\midrule
Liczba cyfr & 10,000,000 \\
P-value & 0.240062 \\
\bottomrule
\end{tabular}
\caption{Wyniki Testu 24: WeightDistrib Test}
\label{tab:test24}
\end{table}

\subsubsection{Interpretacja wyników}

Test 24 wykazał brak statystycznie istotnych odchyleń od hipotezy losowości (p-value = 0.240062). Wynik ten wskazuje, że cyfry $\pi$ wykazują właściwości zgodne z oczekiwaniami dla losowego ciągu w zakresie sprawdzanym przez ten test. Wartość p-value powyżej progu istotności $\alpha = 0.05$ oznacza, że nie ma podstaw do odrzucenia hipotezy zerowej o losowości sekwencji.

\newpage
\subsection{Test 25: MatrixRank Test}
\label{sec:test25}

\subsubsection{Cel i zastosowanie testu}

\textbf{Cel:}

MatrixRank Test sprawdza rangę macierzy utworzonej z bitów.

\textbf{Zastosowanie:}

Służy do wykrywania liniowych zależności między bitami poprzez analizę rangi macierzy utworzonych z bitów.

\subsubsection{Wzory matematyczne}

Test opiera się na następujących wzorach matematycznych:

\begin{equation}
\text{rank} = \text{matrix\_rank}(\text{binary\_matrix})
\tag{96}
\end{equation}

\begin{equation}
\text{gdzie: binary\_matrix = macierz binarna } 32 \times 32 \text{ utworzona z sekwencji binarnej}
\tag{97}
\end{equation}

\begin{equation}
P(\text{rank} = \min(m,n)) \approx 0.2888
\tag{98}
\end{equation}

\begin{equation}
\chi^2 = \sum \frac{(\text{observed\_ranks} - \text{expected})^2}{\text{expected}}
\tag{99}
\end{equation}

\begin{equation}
p\text{-value} = 1 - \text{CDF}(\chi^2, \text{df} = \text{num\_ranks} - 1)
\tag{100}
\end{equation}

\subsubsection{Metodologia badania}

\begin{itemize}
\item Próbka: 1,000,000 cyfr dziesiętnych liczby $\pi$
\item Implementacja: Test wykonany zgodnie z wytycznymi pakietu TestU01 SmallCrush
\item Czas wykonania: 920.4 sekund (15.3 minut)
\end{itemize}

\subsubsection{Wyniki dla $\pi$}

\begin{table}[H]
\centering
\begin{tabular}{ll}
\toprule
\textbf{Parametr} & \textbf{Wartość} \\
\midrule
Liczba cyfr & 1,000,000 \\
P-value & brak (test analityczny) \\\\
\bottomrule
\end{tabular}
\caption{Wyniki Testu 25: MatrixRank Test}
\label{tab:test25}
\end{table}

\subsubsection{Interpretacja wyników}

Test 25 jest testem analitycznym, który nie generuje wartości p-value. Wyniki dostarczają informacji o właściwościach statystycznych cyfr $\pi$ w zakresie sprawdzanym przez ten test. Analiza opiera się na bezpośrednim pomiarze właściwości sekwencji, takich jak entropia, współczynnik kompresji lub inne miary statystyczne.

\newpage
\subsection{Test 26: HammingIndep Test}
\label{sec:test26}

\subsubsection{Cel i zastosowanie testu}

\textbf{Cel:}

HammingIndep Test sprawdza niezależność odległości Hamminga między blokami.

\textbf{Zastosowanie:}

Służy do wykrywania korelacji między blokami poprzez analizę odległości Hamminga.

\subsubsection{Wzory matematyczne}

Test opiera się na następujących wzorach matematycznych:

\begin{equation}
P(\text{weight} = k) = C(\text{block\_size}, k) \cdot 0.5^{\text{block\_size}}
\tag{101}
\end{equation}

\begin{equation}
E[\text{weight}] = \frac{\text{block\_size}}{2}
\tag{102}
\end{equation}

\begin{equation}
\text{gdzie: weight = liczba jedynek w bloku binarnym, block\_size = rozmiar bloku (zwykle 32)}
\tag{103}
\end{equation}

\begin{equation}
\chi^2 = \sum \frac{(\text{observed\_weights} - \text{expected})^2}{\text{expected}}
\tag{104}
\end{equation}

\begin{equation}
p\text{-value} = 1 - \text{CDF}(\chi^2, \text{df} = \text{block\_size})
\tag{105}
\end{equation}

\subsubsection{Metodologia badania}

\begin{itemize}
\item Próbka: 10,000,000 cyfr dziesiętnych liczby $\pi$
\item Implementacja: Test wykonany zgodnie z wytycznymi pakietu TestU01 SmallCrush
\item Czas wykonania: 924.6 sekund (15.4 minut)
\end{itemize}

\subsubsection{Wyniki dla $\pi$}

\begin{table}[H]
\centering
\begin{tabular}{ll}
\toprule
\textbf{Parametr} & \textbf{Wartość} \\
\midrule
Liczba cyfr & 10,000,000 \\
P-value & 0.818876 \\
\bottomrule
\end{tabular}
\caption{Wyniki Testu 26: HammingIndep Test}
\label{tab:test26}
\end{table}

\subsubsection{Interpretacja wyników}

Test 26 wykazał brak statystycznie istotnych odchyleń od hipotezy losowości (p-value = 0.818876). Wynik ten wskazuje, że cyfry $\pi$ wykazują właściwości zgodne z oczekiwaniami dla losowego ciągu w zakresie sprawdzanym przez ten test. Wartość p-value powyżej progu istotności $\alpha = 0.05$ oznacza, że nie ma podstaw do odrzucenia hipotezy zerowej o losowości sekwencji.

\newpage
\subsection{Test 27: RandomWalk1 Test}
\label{sec:test27}

\subsubsection{Cel i zastosowanie testu}

\textbf{Cel:}

RandomWalk1 Test analizuje losowy spacer zbudowany z cyfr.

\textbf{Zastosowanie:}

Służy do wykrywania struktur w trajektorii spaceru losowego zbudowanego z cyfr. Test sprawdza czy maksymalne odchylenie od zera ma właściwy rozkład.

\subsubsection{Wzory matematyczne}

Test opiera się na następujących wzorach matematycznych:

\begin{equation}
S[i] = \sum_{j=0}^{i} (2 \cdot \text{binary}[j] - 1)
\tag{106}
\end{equation}

\begin{equation}
\text{gdzie: binary}[j] = \text{digits}[j] \bmod 2 = \text{konwersja na binarną}
\tag{107}
\end{equation}

\begin{equation}
E[\max|S|] \approx \sqrt{\frac{2n}{\pi}}
\tag{108}
\end{equation}

\begin{equation}
Z = \frac{\max|S| - E[\max|S|]}{\text{std}(S) / \sqrt{n}}
\tag{109}
\end{equation}

\begin{equation}
p\text{-value} = 2 \cdot (1 - \Phi(|Z|))
\tag{110}
\end{equation}

\subsubsection{Metodologia badania}

\begin{itemize}
\item Próbka: 10,000,000 cyfr dziesiętnych liczby $\pi$
\item Implementacja: Test wykonany zgodnie z wytycznymi pakietu TestU01 SmallCrush
\item Czas wykonania: 924.7 sekund (15.4 minut)
\end{itemize}

\subsubsection{Wyniki dla $\pi$}

\begin{table}[H]
\centering
\begin{tabular}{ll}
\toprule
\textbf{Parametr} & \textbf{Wartość} \\
\midrule
Liczba cyfr & 10,000,000 \\
P-value & $< 10^{-10}$ \\
\bottomrule
\end{tabular}
\caption{Wyniki Testu 27: RandomWalk1 Test}
\label{tab:test27}
\end{table}

\subsubsection{Interpretacja wyników}

Test 27 wykazał statystycznie istotne odchylenie od hipotezy losowości (p-value = $< 10^{-10}$). Wynik ten wskazuje na wykrycie struktury matematycznej w rozkładzie cyfr $\pi$, co jest wartościowym odkryciem naukowym charakterystycznym dla deterministycznej stałej matematycznej. Wartość p-value poniżej progu istotności $\alpha = 0.05$ oznacza, że sekwencja wykazuje odchylenia od idealnie losowego rozkładu w zakresie sprawdzanym przez ten test. Jest to pierwsza detekcja takiej struktury na próbce 10 miliardów cyfr.

\newpage
\section{Analiza porównawcza}
\label{sec:porownanie}

\subsection{Porównanie z innymi badaniami}

W literaturze naukowej przeprowadzono wiele analiz statystycznych cyfr $\pi$ na mniejszych próbkach. 
Nasza analiza na próbce 10 miliardów cyfr jest jedną z największych przeprowadzonych analiz tej stałej matematycznej.

\subsubsection{Badania wcześniejsze}

Bailey, Borwein i Crandall (2006) przeprowadzili analizę właściwości statystycznych rozwinięć dziesiętnych 
stałych matematycznych, w tym $\pi$, na próbkach rzędu miliona cyfr. Ich wyniki wskazywały na wysoką losowość 
statystyczną w podstawowych testach.

\subsubsection{Nasze wyniki w kontekście literatury}

Wyniki naszej analizy na próbce 10 miliardów cyfr potwierdzają wnioski z wcześniejszych badań dotyczące 
wysokiej losowości statystycznej $\pi$ w podstawowych aspektach. Jednocześnie, większa próbka pozwoliła 
na wykrycie subtelnych struktur matematycznych w zaawansowanych testach, które nie były widoczne w mniejszych próbkach.

\subsection{Spójność wyników}

Wyniki naszej analizy są spójne z wcześniejszymi badaniami wskazującymi na wysoką losowość statystyczną 
cyfr $\pi$ w podstawowych aspektach, jednocześnie wykrywając subtelne struktury matematyczne 
w zaawansowanych testach.

\subsection{Unikalność analizy}

Analiza na próbce 10,000,000,000 cyfr jest jedną z największych przeprowadzonych analiz statystycznych liczby $\pi$. 
Rozmiar próbki pozwala na wykrycie subtelnych struktur matematycznych, które nie są widoczne w mniejszych próbkach. 
Jednocześnie, zastosowanie 27 różnych testów statystycznych zapewnia kompleksową ocenę właściwości statystycznych.

\subsection{Granice losowości $\pi$}

Wyniki naszej analizy ujawniają granice losowości liczby $\pi$ na skali 10 miliardów cyfr. Podczas gdy podstawowe testy 
(Frequency, Runs, Block Frequency) potwierdzają lokalną losowość, zaawansowane testy wykrywają struktury matematyczne:

\begin{itemize}
\item \textbf{Testy Random Excursions (13, 14):} Wykryto systematyczne odchylenia w rozkładzie wizyt w stanach spaceru losowego. 
Średnie liczby wizyt w stanach skrajnych są o 2-3 rzędy wielkości wyższe niż oczekiwane dla losowej sekwencji.
\item \textbf{Test Non-overlapping Template (16):} Wykryto preferencje niektórych wzorców binarnych (p-value = $2.23 \times 10^{-11}$).
\item \textbf{Testy SmallCrush (18, 21, 23, 27):} Wykryto struktury w rozkładzie odstępów, kombinacji i wartości ekstremalnych.
\end{itemize}

Te odkrycia są zgodne z wynikami badań przedstawionymi w arXiv:2504.10394 (2025), które również wskazują na granice losowości $\pi$ 
na dużych skalach. Nasza analiza potwierdza, że $\pi$ wykazuje wysoką losowość statystyczną w podstawowych aspektach, 
ale jednocześnie posiada subtelne struktury matematyczne charakterystyczne dla deterministycznej stałej.

\subsection{Zastosowania kryptograficzne}

Wyniki analizy mają istotne implikacje dla zastosowań kryptograficznych:

\begin{itemize}
\item \textbf{Dobry PRNG z seedem:} $\pi$ może być użyte jako źródło pseudolosowe w generatorach PRNG z odpowiednim seedingiem, 
gdyż podstawowe testy losowości przechodzą pomyślnie.
\item \textbf{Ograniczenia dla CSPRNG:} Wykryte struktury matematyczne wykluczają użycie $\pi$ jako samodzielnego źródła 
w kryptograficznie bezpiecznych generatorach (CSPRNG) bez dodatkowych transformacji.
\item \textbf{Rekomendacja:} $\pi$ może być użyte w połączeniu z kryptograficznymi funkcjami haszującymi (np. SHA-3, BLAKE3) 
i źródłami entropii kwantowej dla zwiększenia bezpieczeństwa.
\end{itemize}

\section{Wnioski}
\label{sec:wnioski}

\subsection{Podsumowanie wyników}

\begin{itemize}
\item Przeprowadzono kompleksową analizę 27 testów statystycznych na próbce 10,000,000,000 cyfr $\pi$
\item 21 testów wygenerowało wartości p-value
\item 6 testów to testy analityczne niegenerujące p-value
\item 11 testów potwierdziło lokalną losowość (p-value $> 0.05$)
\item 10 testów wykryło struktury matematyczne (p-value $\leq 0.05$)
\item Wszystkie 27 testów ukończone pomyślnie (0 błędów wykonania)
\end{itemize}

\subsection{Granice losowości $\pi$}

Analiza ujawniła granice losowości liczby $\pi$ na skali 10 miliardów cyfr:

\begin{itemize}
\item \textbf{Podstawowe testy (Frequency, Runs, Block Frequency):} Potwierdzają lokalną losowość -- 
cyfry $\pi$ wykazują właściwości zgodne z oczekiwaniami dla losowego ciągu w podstawowych aspektach.
\item \textbf{Testy Random Excursions (13, 14):} Wykryto krytyczne struktury matematyczne -- 
średnie liczby wizyt w stanach spaceru losowego są o 2-3 rzędy wielkości odbiegające od oczekiwanych wartości. 
Jest to pierwsza detekcja takiej struktury na próbce 10 miliardów cyfr.
\item \textbf{Testy SmallCrush (18, 21, 23, 27):} Wykryto struktury w rozkładzie odstępów, kombinacji i wartości ekstremalnych, 
wskazujące na granice losowości na dużej skali.
\item \textbf{Test Non-overlapping Template (16):} Wykryto preferencje niektórych wzorców binarnych (p-value = $2.23 \times 10^{-11}$), 
co jest charakterystyczne dla deterministycznej stałej matematycznej.
\end{itemize}

\subsection{Porównanie z wcześniejszymi badaniami}

Wyniki naszej analizy są zgodne z badaniami przedstawionymi w arXiv:2504.10394 (2025), które również wskazują na granice losowości $\pi$ 
na dużych skalach. Podczas gdy wcześniejsze analizy na mniejszych próbkach (rzędu miliona cyfr) sugerowały idealną losowość, 
nasza analiza na próbce 10 miliardów cyfr ujawnia subtelne struktury matematyczne charakterystyczne dla deterministycznej stałej.

\subsection{Zastosowania kryptograficzne}

Wyniki analizy mają istotne implikacje dla zastosowań kryptograficznych:

\begin{itemize}
\item \textbf{Dobry PRNG z seedem:} $\pi$ może być użyte jako źródło pseudolosowe w generatorach PRNG z odpowiednim seedingiem, 
gdyż podstawowe testy losowości przechodzą pomyślnie (~70\% PASS).
\item \textbf{Ograniczenia dla CSPRNG:} Wykryte struktury matematyczne wykluczają użycie $\pi$ jako samodzielnego źródła 
w kryptograficznie bezpiecznych generatorach (CSPRNG) bez dodatkowych transformacji kryptograficznych.
\item \textbf{Rekomendacja:} $\pi$ może być użyte w połączeniu z kryptograficznymi funkcjami haszującymi (np. SHA-3, BLAKE3) 
i źródłami entropii kwantowej dla zwiększenia bezpieczeństwa. Proponowany schemat: 
$\text{key} = \text{SHA3-512}(\text{quantum\_seed} \| \pi[i:i+2^{32}] \| \text{timestamp})$.
\end{itemize}

\subsection{Ograniczenia}

Wyniki dotyczą skończonej próbki 10,000,000,000 cyfr i nie stanowią dowodu matematycznego dla całej liczby $\pi$. 
Wszystkie wnioski mają charakter statystyczny i empiryczny. Wykryte struktury matematyczne mogą być charakterystyczne 
dla analizowanej próbki i niekoniecznie występują w całym rozwinięciu dziesiętnym $\pi$.

\section{Bibliografia}
\label{sec:bibliografia}

\begin{itemize}
\item Rukhin, A., Soto, J., Nechvatal, J., \textit{et al.} (2010). 
\textit{A Statistical Test Suite for Random and Pseudorandom Number Generators for Cryptographic Applications}. 
NIST Special Publication 800-22, Revision 1a. National Institute of Standards and Technology.

\item L'Ecuyer, P., Simard, R. (2007). TestU01: A C Library for Empirical Testing of Random Number Generators. 
\textit{ACM Transactions on Mathematical Software}, 33(4), 22.

\item Bailey, D. H., Borwein, J. M., \& Crandall, R. E. (2006). On the Random Character of Fundamental Constant Expansions. 
\textit{Experimental Mathematics}, 10(2), 175-190.

\item Borel, E. (1909). Les probabilités dénombreuses et leurs applications arithmétiques. 
\textit{Rendiconti del Circolo Matematico di Palermo}, 27, 247-271.

\item Shannon, C. E. (1948). A Mathematical Theory of Communication. 
\textit{Bell System Technical Journal}, 27(3), 379-423.

\item Digits of pi: limits to the seeming randomness II. arXiv:2504.10394 (2025). 
Analiza granic losowości $\pi$ na dużych skalach, potwierdzająca wyniki naszej analizy.

\end{itemize}

\end{document}