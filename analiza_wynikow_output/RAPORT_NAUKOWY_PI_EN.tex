\documentclass[11pt,a4paper]{article}
\usepackage[utf8]{inputenc}
\usepackage[T1]{fontenc}
\usepackage{amsmath,amsfonts,amssymb,amsthm}
\usepackage{graphicx}
\usepackage{booktabs}
\usepackage{longtable}
\usepackage{array}
\usepackage{multirow}
\usepackage{xcolor}
\usepackage{geometry}
\geometry{a4paper, margin=2.5cm}
\usepackage{hyperref}
\hypersetup{colorlinks=true, linkcolor=blue, urlcolor=blue, citecolor=blue}
\usepackage{float}
\usepackage{setspace}
\onehalfspacing
\usepackage{adjustbox}

\title{Empirical Analysis of Statistical Properties of $\pi$ \\
Based on 10 Billion Digits}
\author{}
\date{January 07, 2026}

\begin{document}
\maketitle
\thispagestyle{empty}
\newpage

\begin{abstract}
\noindent
We conducted a comprehensive statistical analysis of the properties of $\pi$ based on 10,000,000,000 decimal digits. 
We performed 27 statistical tests from the NIST Statistical Test Suite and TestU01 SmallCrush packages. 
All tests confirm that $\pi$ is maximally complex, statistically random, and ergodic. 
The results indicate high statistical randomness in basic aspects, while simultaneously detecting subtle mathematical structures 
characteristic of a deterministic mathematical constant.
\end{abstract}

\tableofcontents
\newpage
\setcounter{page}{1}

\section{Introduction}
\label{sec:introduction}

The number $\pi$ is one of the most important mathematical constants. Although it is completely deterministic, 
its decimal expansion exhibits statistical properties indistinguishable from random data. 
In this work, we present an empirical analysis of the properties of $\pi$ based on 10,000,000,000 digits.

\section{Methodology}
\label{sec:methodology}

\subsection{Data Sample}

The analysis was conducted on a sample of 10,000,000,000 decimal digits of $\pi$. 
The digits were generated using high-precision computational algorithms and saved in text format.

\subsection{Description of Statistical Tests}

In this section, we present detailed descriptions of each of the applied statistical tests, 
along with explanations of purpose, application, and mathematical formulas.

\subsubsection{Test 01: Frequency Test (NIST)}

\textbf{Purpose:}

Frequency Test (Monobit Test) checks whether the proportion of zeros and ones in the binary representation of digits is approximately 1:1.

\textbf{Application:}

This is the most basic randomness test. It serves to verify uniform distribution of bits in a binary sequence. It tests the null hypothesis that the sequence is random by comparing the frequency of occurrence of each digit with the expected frequency.

\textbf{Mathematical Formulas:}

\begin{equation}
\chi^2 = \sum_{i=0}^{9}\frac{(f_i - n/10)^2}{n/10}
\tag{1}
\end{equation}

\begin{equation}
E[f_i] = \frac{n}{10} = \text{expected frequency of each digit}
\tag{2}
\end{equation}

\begin{equation}
p\text{-value} = 1 - \text{CDF}(\chi^2, \text{df} = 9)
\tag{3}
\end{equation}

\begin{equation}
\text{where: } f_i = \text{frequency of digit } i \text{ (0-9), } n = \text{total number of digits}
\tag{4}
\end{equation}


\subsubsection{Test 02: Runs Test (NIST)}

\textbf{Purpose:}

Runs Test analyzes uninterrupted sequences of consecutive zeros or ones (runs).

\textbf{Application:}

Serves to detect correlations between consecutive bits. Checks whether transitions between 0 and 1 occur with expected frequency.

\textbf{Mathematical Formulas:}

\begin{equation}
E[\text{runs}] = 2 \cdot \text{ones} \cdot \text{zeros} / n
\tag{5}
\end{equation}

\begin{equation}
\text{Var}[\text{runs}] = \frac{2 \cdot \text{ones} \cdot \text{zeros} \cdot (2 \cdot \text{ones} \cdot \text{zeros} - n)}{n^2 \cdot (n - 1)}
\tag{6}
\end{equation}

\begin{equation}
Z = \frac{\text{runs} - E[\text{runs}]}{\sqrt{\text{Var}[\text{runs}]}}
\tag{7}
\end{equation}

\begin{equation}
p\text{-value} = 2 \cdot (1 - \Phi(|Z|))
\tag{8}
\end{equation}

\begin{equation}
\text{where: ones = number of odd digits, zeros = number of even digits}
\tag{9}
\end{equation}


\subsubsection{Test 03: Block Frequency Test (NIST)}

\textbf{Purpose:}

Block Frequency Test divides the sequence into blocks and checks the frequency of ones in each block.

\textbf{Application:}

Serves to detect local non-uniformities in bit distribution at the block level.

\textbf{Mathematical Formulas:}

\begin{equation}
\chi^2 = \sum_{j} \frac{(\text{ones\_per\_block}_j - \text{block\_size} / 2)^2}{\text{block\_size} / 2}
\tag{10}
\end{equation}

\begin{equation}
E[\text{ones}] = \frac{\text{block\_size}}{2} = \text{expected number of ones in block}
\tag{11}
\end{equation}

\begin{equation}
p\text{-value} = 1 - \text{CDF}(\chi^2, \text{df} = \text{num\_blocks})
\tag{12}
\end{equation}

\begin{equation}
\text{where: ones\_per\_block = number of ones in block } j
\tag{13}
\end{equation}


\subsubsection{Test 04: Entropy Analysis}

\textbf{Purpose:}

Entropy Analysis calculates Shannon entropy for the digit distribution.

\textbf{Application:}

Serves to measure unpredictability and complexity of the sequence. High entropy indicates high randomness.

\textbf{Mathematical Formulas:}

\begin{equation}
H(X) = -\sum_{x=0}^{9} p(x) \cdot \log_2(p(x))
\tag{14}
\end{equation}

\begin{equation}
p(x) = \frac{\text{count}(x)}{n} = \text{probability of digit } x
\tag{15}
\end{equation}

\begin{equation}
H_{\max} = \log_2(10) \approx 3.321928 = \text{maximum entropy for 10 digits}
\tag{16}
\end{equation}

\begin{equation}
\text{ratio} = \frac{H(X)}{H_{\max}}
\tag{17}
\end{equation}


\subsubsection{Test 05: Spectral FFT Analysis}

\textbf{Purpose:}

Spectral FFT Analysis uses Fourier transform to detect periodicity.

\textbf{Application:}

Serves to detect hidden periodic patterns in the digit sequence.

\textbf{Mathematical Formulas:}

\begin{equation}
X[k] = \sum_{n=0}^{N-1} x[n] \cdot e^{-2\pi ikn / N}
\tag{18}
\end{equation}

\begin{equation}
P[k] = |X[k]|^2 = \text{power spectrum}
\tag{19}
\end{equation}

\begin{equation}
H_s = -\sum_k \frac{P[k]}{\sum P} \cdot \log_2\left(\frac{P[k]}{\sum P} + \varepsilon\right)
\tag{20}
\end{equation}

\begin{equation}
\text{where: } x[n] = \text{digit pairs}(\text{digits}[i] \cdot 10 + \text{digits}[i + 1]), \varepsilon = 10^{-10}
\tag{21}
\end{equation}


\subsubsection{Test 06: Compression Test}

\textbf{Purpose:}

Compression Test measures the degree of data compression using zlib algorithm.

\textbf{Application:}

Serves to assess sequence complexity. Low compression indicates high complexity and randomness.

\textbf{Mathematical Formulas:}

\begin{equation}
\text{compression\_ratio} = \frac{\text{compressed\_size}}{\text{original\_size}}
\tag{22}
\end{equation}

\begin{equation}
\text{where: original\_size = size of original data, compressed\_size = size after zlib compression}
\tag{23}
\end{equation}

\begin{equation}
\text{Interpretation: Lower ratio = higher randomness}
\tag{24}
\end{equation}


\subsubsection{Test 07: Empirical Entropy Bounds}

\textbf{Purpose:}

Empirical Entropy Bounds analyzes entropy limits for different block lengths.

\textbf{Application:}

Serves to study how entropy changes depending on the length of analyzed blocks.

\textbf{Mathematical Formulas:}

\begin{equation}
H(N) = \log_2(10) \cdot \left(1 - \frac{c}{\log(N)}\right)
\tag{25}
\end{equation}

\begin{equation}
c = \arg \min \sum (H_{\text{observed}}(N) - H_{\text{model}}(N,c))^2
\tag{26}
\end{equation}

\begin{equation}
H_{\max} = \log_2(10) \approx 3.321928
\tag{27}
\end{equation}

\begin{equation}
\text{Confidence interval (95\%): } \text{CI} = c \pm 1.96 \cdot \sigma_c
\tag{28}
\end{equation}

\begin{equation}
\text{where: } N = \text{number of analyzed digits, } c = \text{fitting parameter}
\tag{29}
\end{equation}


\subsubsection{Test 08: ML LSTM Anomaly Detection}

\textbf{Purpose:}

ML LSTM Anomaly Detection uses an LSTM neural network to detect anomalies.

\textbf{Application:}

Serves to detect patterns and anomalies in the digit sequence using machine learning. The network attempts to predict the next digit based on previous ones.

\textbf{Mathematical Formulas:}

\begin{equation}
\text{Accuracy} = \frac{1}{m} \sum_{i=1}^{m} \mathbf{1}[\hat{d}_i = d_i]
\tag{30}
\end{equation}


\subsubsection{Test 09: Cumulative Sums Test (NIST)}

\textbf{Purpose:}

Cumulative Sums Test analyzes maximum deviation of cumulative sums.

\textbf{Application:}

Serves to detect systematic trends in the bit sequence.

\textbf{Mathematical Formulas:}

\begin{equation}
S_{\text{forward}}[i] = \sum_{j=0}^{i}(2 \cdot \text{binary}[j] - 1)
\tag{31}
\end{equation}

\begin{equation}
S_{\text{backward}}[i] = \sum_{j=i}^{n}(2 \cdot \text{binary}[j] - 1)
\tag{32}
\end{equation}

\begin{equation}
\max_{\text{forward}} = \max_i |S_{\text{forward}}[i]|, \quad \max_{\text{backward}} = \max_i |S_{\text{backward}}[i]|
\tag{33}
\end{equation}

\begin{equation}
Z_{\text{forward}} = \frac{\max_{\text{forward}}}{\sqrt{n}}, \quad Z_{\text{backward}} = \frac{\max_{\text{backward}}}{\sqrt{n}}
\tag{34}
\end{equation}

\begin{equation}
p\text{-value} = \min(p_{\text{forward}}, p_{\text{backward}})
\tag{35}
\end{equation}


\subsubsection{Test 10: Approximate Entropy Test (NIST)}

\textbf{Purpose:}

Approximate Entropy Test measures regularity of patterns of given length.

\textbf{Application:}

Serves to detect regular patterns in the sequence. Low approximate entropy indicates predictability.

\textbf{Mathematical Formulas:}

\begin{equation}
\text{ApEn}(m,r) = \Phi^m(r) - \Phi^{m+1}(r)
\tag{36}
\end{equation}

\begin{equation}
\Phi^m(r) = \frac{1}{N-m+1}\sum_{i=1}^{N-m+1}\log C_i^m(r)
\tag{37}
\end{equation}

\begin{equation}
C_i^m(r) = \frac{\text{number of patterns of length } m \text{ similar to } x[i:i+m]}{N-m+1}
\tag{38}
\end{equation}

\begin{equation}
\chi^2 = \frac{(\text{ApEn} - E[\text{ApEn}])^2}{\text{Var}[\text{ApEn}]}, \quad p\text{-value} = 1 - \text{CDF}(\chi^2, \text{df}=1)
\tag{39}
\end{equation}


\subsubsection{Test 11: Serial Test (NIST)}

\textbf{Purpose:}

Serial Test analyzes frequency of overlapping patterns of length $m$.

\textbf{Application:}

Serves to detect preferences for certain patterns over others.

\textbf{Mathematical Formulas:}

\begin{equation}
\Delta\psi_m^2 = \psi_m^2 - \psi_{m-1}^2
\tag{40}
\end{equation}

\begin{equation}
\psi_m^2 = \frac{2^m}{n}\sum(\text{obs}_i^2) - n
\tag{41}
\end{equation}

\begin{equation}
\text{where: obs}_i = \text{number of occurrences of pattern } i \text{ of length } m
\tag{42}
\end{equation}

\begin{equation}
p\text{-value} = 1 - \text{CDF}(\Delta\psi_m^2, \text{df} = 2^{m-1})
\tag{43}
\end{equation}


\subsubsection{Test 12: Linear Complexity Test (NIST)}

\textbf{Purpose:}

Linear Complexity Test measures the length of the shortest LFSR generating the sequence.

\textbf{Application:}

Serves to assess linear complexity of the sequence. Low complexity indicates linear patterns.

\textbf{Mathematical Formulas:}

\begin{equation}
L = \text{Berlekamp-Massey}(S) = \text{length of shortest LFSR}
\tag{44}
\end{equation}

\begin{equation}
E[L] = \frac{M}{2} + \frac{9 + ((-1)^{M+1})}{36}
\tag{45}
\end{equation}

\begin{equation}
\chi^2 = \sum \frac{(\text{observed\_complexities} - E[L])^2}{E[L]}
\tag{46}
\end{equation}

\begin{equation}
p\text{-value} = 1 - \text{CDF}(\chi^2, \text{df} = \text{num\_bins} - 1)
\tag{47}
\end{equation}

\begin{equation}
\text{where: } M = \text{binary block length}
\tag{48}
\end{equation}


\subsubsection{Test 13: Random Excursions Test (NIST)}

\textbf{Purpose:}

Random Excursions Test analyzes a random walk built from a binary sequence.

\textbf{Application:}

Serves to detect structures in the trajectory of a random walk. Checks the distribution of visits to specific states.

\textbf{Mathematical Formulas:}

\begin{equation}
S_k = \sum_{i=1}^{k}(2 \cdot \text{binary}[i] - 1) = \text{random walk}
\tag{49}
\end{equation}

\begin{equation}
\xi(x) = \text{number of visits to state } x \text{ for } x \in \{-4, -3, -2, -1, 1, 2, 3, 4\}
\tag{50}
\end{equation}

\begin{equation}
E[\xi(x)] = \frac{1}{2|x|(|x|+1)}, \quad \text{Var}[\xi(x)] = \frac{4|x|(J-|x|-1)}{(J-1)^2(2|x|+1)}
\tag{51}
\end{equation}

\begin{equation}
\chi^2 = \sum_{x} \frac{(\xi(x) - E[\xi(x)])^2}{E[\xi(x)]}, \quad p\text{-value} = 1 - \text{CDF}(\chi^2, \text{df} = 7)
\tag{52}
\end{equation}


\subsubsection{Test 14: Random Excursions Variant Test (NIST)}

\textbf{Purpose:}

Random Excursions Variant Test is a variant of Random Excursions test for a larger range of states.

\textbf{Application:}

Serves to detect structures in the trajectory of a random walk for states in the range $\{-9, \ldots, -1, 1, \ldots, 9\}$.

\textbf{Mathematical Formulas:}

\begin{equation}
S_k = \sum_{i=1}^{k}(2 \cdot \text{binary}[i] - 1) = \text{random walk}
\tag{53}
\end{equation}

\begin{equation}
\xi(x) = \text{number of visits to state } x \text{ for } x \in \{-9, \ldots, -1, 1, \ldots, 9\}
\tag{54}
\end{equation}

\begin{equation}
E[\xi(x)] = \frac{1}{2|x|(|x|+1)}, \quad \text{Var}[\xi(x)] = \frac{4|x|(J-|x|-1)}{(J-1)^2(2|x|+1)}
\tag{55}
\end{equation}

\begin{equation}
\chi^2 = \sum_{x} \frac{(\xi(x) - E[\xi(x)])^2}{E[\xi(x)]}, \quad p\text{-value} = 1 - \text{CDF}(\chi^2, \text{df} = 17)
\tag{56}
\end{equation}


\subsubsection{Test 15: Universal Statistical Test (NIST)}

\textbf{Purpose:}

Universal Statistical Test checks whether the sequence can be significantly compressed.

\textbf{Application:}

Serves to detect sequence compressibility. High compressibility indicates structure.

\textbf{Mathematical Formulas:}

\begin{equation}
f_n = \frac{1}{K}\sum_{i=1}^{K}\log_2(i - \text{last\_pos}[\text{pattern}_i])
\tag{57}
\end{equation}

\begin{equation}
E[f_n] = \begin{cases} 5.2177052 & \text{for } L=6 \\ 6.1962507 & \text{for } L=7 \\ 7.1836656 & \text{for } L=8 \end{cases}
\tag{58}
\end{equation}

\begin{equation}
\text{Var}[f_n] = \begin{cases} 2.954 & \text{for } L=6 \\ 3.125 & \text{for } L=7 \\ 3.238 & \text{for } L=8 \end{cases}
\tag{59}
\end{equation}

\begin{equation}
Z = \frac{f_n - E[f_n]}{\sqrt{\text{Var}[f_n]/K}}, \quad p\text{-value} = 2 \cdot (1 - \Phi(|Z|))
\tag{60}
\end{equation}

\begin{equation}
\text{where: } L = \text{block length, } K = \text{number of test blocks}
\tag{61}
\end{equation}


\subsubsection{Test 16: Non-overlapping Template Matching Test (NIST)}

\textbf{Purpose:}

Non-overlapping Template Matching Test searches for non-overlapping occurrences of a pattern.

\textbf{Application:}

Serves to detect preferences for certain binary patterns through analysis of non-overlapping occurrences.

\textbf{Mathematical Formulas:}

\begin{equation}
E[\text{matches}] = \frac{n - m + 1}{2^m} \cdot \frac{1}{2^m} = \frac{n - m + 1}{4^m}
\tag{62}
\end{equation}

\begin{equation}
\text{where: } m = \text{binary pattern length, } n = \text{sequence length}
\tag{63}
\end{equation}

\begin{equation}
\chi^2 = \frac{(\text{matches} - E[\text{matches}])^2}{E[\text{matches}]}
\tag{64}
\end{equation}

\begin{equation}
p\text{-value} = 1 - \text{CDF}(\chi^2, \text{df} = 1)
\tag{65}
\end{equation}


\subsubsection{Test 17: Overlapping Template Matching Test (NIST)}

\textbf{Purpose:}

Overlapping Template Matching Test searches for overlapping occurrences of a pattern.

\textbf{Application:}

Serves to detect preferences for certain binary patterns through analysis of overlapping occurrences.

\textbf{Mathematical Formulas:}

\begin{equation}
E[\text{matches}] = \frac{n - m + 1}{2^m}
\tag{66}
\end{equation}

\begin{equation}
\text{where: } m = \text{binary pattern length, } n = \text{binary sequence length}
\tag{67}
\end{equation}

\begin{equation}
\chi^2 = \frac{(\text{matches} - E[\text{matches}])^2}{E[\text{matches}]}
\tag{68}
\end{equation}

\begin{equation}
p\text{-value} = 1 - \text{CDF}(\chi^2, \text{df} = 1)
\tag{69}
\end{equation}


\subsubsection{Test 18: BirthdaySpacings Test (SmallCrush)}

\textbf{Purpose:}

BirthdaySpacings Test is based on the birthday paradox, analyzes spacings between repeating values.

\textbf{Application:}

Serves to detect specific distributions of spacings between repetitions. The test checks whether spacings between repeating values have the proper exponential distribution.

\textbf{Mathematical Formulas:}

\begin{equation}
P(\text{collision}) \approx 1 - e^{-n^2/(2d)}
\tag{70}
\end{equation}

\begin{equation}
\chi^2 = \sum \frac{(O_i - E_i)^2}{E_i}
\tag{71}
\end{equation}

\begin{equation}
P(\text{spacing} = k) = (1-p)^k \cdot p
\tag{72}
\end{equation}


\subsubsection{Test 19: Collision Test (SmallCrush)}

\textbf{Purpose:}

Collision Test counts collisions in a hash table.

\textbf{Application:}

Serves to detect irregularities in value distribution through analysis of the number of collisions in a hash table.

\textbf{Mathematical Formulas:}

\begin{equation}
E[\text{collisions}] = t - m + m \cdot (1 - 1/m)^t
\tag{73}
\end{equation}

\begin{equation}
\text{where: } t = \text{number of samples, } m = \text{value range (10 for digits 0-9)}
\tag{74}
\end{equation}

\begin{equation}
\chi^2 = \frac{(\text{collisions} - E[\text{collisions}])^2}{E[\text{collisions}]}
\tag{75}
\end{equation}

\begin{equation}
p\text{-value} = 1 - \text{CDF}(\chi^2, \text{df} = 1)
\tag{76}
\end{equation}


\subsubsection{Test 20: Gap Test (SmallCrush)}

\textbf{Purpose:}

Gap Test analyzes lengths of gaps between values from a specified range.

\textbf{Application:}

Serves to detect deviations from the geometric distribution of gaps between occurrences of a specified value.

\textbf{Mathematical Formulas:}

\begin{equation}
P(\text{gap} = k) = (1 - p)^k \cdot p
\tag{77}
\end{equation}

\begin{equation}
p = \frac{1}{m} = \text{probability of target value occurrence}
\tag{78}
\end{equation}

\begin{equation}
\text{where: } m = \text{value range (10 for digits 0-9)}
\tag{79}
\end{equation}

\begin{equation}
\chi^2 = \sum \frac{(\text{observed\_gaps} - \text{expected})^2}{\text{expected}}
\tag{80}
\end{equation}

\begin{equation}
p\text{-value} = 1 - \text{CDF}(\chi^2, \text{df} = \text{num\_bins} - 1)
\tag{81}
\end{equation}


\subsubsection{Test 21: SimplePoker Test}

\textbf{Purpose:}

SimplePoker Test divides the sequence into groups and checks the distribution of combinations (analogous to poker).

\textbf{Application:}

Serves to detect structures in the distribution of digit combinations in blocks. The test checks whether the number of unique values in blocks has the proper distribution.

\textbf{Mathematical Formulas:}

\begin{equation}
P(k \text{ unikalnych}) = \frac{C(5, k) \cdot P(\text{permutation})}{10^5}
\tag{82}
\end{equation}

\begin{equation}
\text{where: } C(5,k) = \text{combination 5 choose k, } P(\text{permutation}) = \text{permutation probability}
\tag{83}
\end{equation}

\begin{equation}
\chi^2 = \sum_{k=1}^{5} \frac{(\text{observed}(k) - \text{expected}(k))^2}{\text{expected}(k)}
\tag{84}
\end{equation}

\begin{equation}
p\text{-value} = 1 - \text{CDF}(\chi^2, \text{df} = 4)
\tag{85}
\end{equation}


\subsubsection{Test 22: CouponCollector Test}

\textbf{Purpose:}

CouponCollector Test is based on the coupon collector problem.

\textbf{Application:}

Serves to test whether all possible values occur with expected frequency. Measures how many draws are needed to collect all different values.

\textbf{Mathematical Formulas:}

\begin{equation}
E[\text{length}] = m \cdot H_m
\tag{86}
\end{equation}

\begin{equation}
H_m = \sum_{k=1}^{m} \frac{1}{k} = \text{harmonic number}
\tag{87}
\end{equation}

\begin{equation}
m = 10 = \text{number of different values (digits 0-9)}
\tag{88}
\end{equation}

\begin{equation}
Z = \frac{\text{observed\_mean} - E[\text{length}]}{\text{std} / \sqrt{n_{\text{trials}}}}
\tag{89}
\end{equation}

\begin{equation}
p\text{-value} = 2 \cdot (1 - \Phi(|Z|))
\tag{90}
\end{equation}


\subsubsection{Test 23: MaxOft Test}

\textbf{Purpose:}

MaxOft Test analyzes the distribution of maximum values in blocks.

\textbf{Application:}

Serves to detect deviations in the distribution of extreme values. The test checks whether maximum values in blocks have the proper extreme value distribution (EVD).

\textbf{Mathematical Formulas:}

\begin{equation}
P(\max \leq k) = \left(\frac{k}{9}\right)^t
\tag{91}
\end{equation}

\begin{equation}
P(\max = k) = \left(\frac{k}{9}\right)^t - \left(\frac{k-1}{9}\right)^t
\tag{92}
\end{equation}

\begin{equation}
\text{where: } t = \text{number of samples in group (usually } t = 5), k \in \{0,1,2,\ldots,9\}
\tag{93}
\end{equation}

\begin{equation}
\chi^2 = \sum \frac{(\text{observed} - \text{expected})^2}{\text{expected}}
\tag{94}
\end{equation}

\begin{equation}
p\text{-value} = 1 - \text{CDF}(\chi^2, \text{df} = 9)
\tag{95}
\end{equation}


\subsubsection{Test 24: WeightDistrib Test}

\textbf{Purpose:}

WeightDistrib Test analyzes the distribution of \"weights\" (number of ones) in binary blocks.

\textbf{Application:}

Serves to detect deviations from the binomial distribution of the number of ones in binary blocks.

\textbf{Mathematical Formulas:}

\begin{equation}
E[\text{sum}] = \text{block\_size} \cdot 4.5
\tag{96}
\end{equation}

\begin{equation}
\text{where: block\_size = block size (usually 10), 4.5 = mean of digits 0-9}
\tag{97}
\end{equation}

\begin{equation}
Z = \frac{\text{observed\_mean} - E[\text{sum}]}{\text{std} / \sqrt{n_{\text{blocks}}}}
\tag{98}
\end{equation}

\begin{equation}
p\text{-value} = 2 \cdot (1 - \Phi(|Z|))
\tag{99}
\end{equation}


\subsubsection{Test 25: MatrixRank Test}

\textbf{Purpose:}

MatrixRank Test checks the rank of a matrix formed from bits.

\textbf{Application:}

Serves to detect linear dependencies between bits through analysis of ranks of matrices formed from bits.

\textbf{Mathematical Formulas:}

\begin{equation}
\text{rank} = \text{matrix\_rank}(\text{binary\_matrix})
\tag{100}
\end{equation}

\begin{equation}
\text{where: binary\_matrix = binary matrix } 32 \times 32 \text{ formed from binary sequence}
\tag{101}
\end{equation}

\begin{equation}
P(\text{rank} = \min(m,n)) \approx 0.2888
\tag{102}
\end{equation}

\begin{equation}
\chi^2 = \sum \frac{(\text{observed\_ranks} - \text{expected})^2}{\text{expected}}
\tag{103}
\end{equation}

\begin{equation}
p\text{-value} = 1 - \text{CDF}(\chi^2, \text{df} = \text{num\_ranks} - 1)
\tag{104}
\end{equation}


\subsubsection{Test 26: HammingIndep Test}

\textbf{Purpose:}

HammingIndep Test checks independence of Hamming distances between blocks.

\textbf{Application:}

Serves to detect correlations between blocks through analysis of Hamming distance.

\textbf{Mathematical Formulas:}

\begin{equation}
P(\text{weight} = k) = C(\text{block\_size}, k) \cdot 0.5^{\text{block\_size}}
\tag{105}
\end{equation}

\begin{equation}
E[\text{weight}] = \frac{\text{block\_size}}{2}
\tag{106}
\end{equation}

\begin{equation}
\text{where: weight = number of ones in binary block, block\_size = block size (usually 32)}
\tag{107}
\end{equation}

\begin{equation}
\chi^2 = \sum \frac{(\text{observed\_weights} - \text{expected})^2}{\text{expected}}
\tag{108}
\end{equation}

\begin{equation}
p\text{-value} = 1 - \text{CDF}(\chi^2, \text{df} = \text{block\_size})
\tag{109}
\end{equation}


\subsubsection{Test 27: RandomWalk1 Test}

\textbf{Purpose:}

RandomWalk1 Test analyzes a random walk built from digits.

\textbf{Application:}

Serves to detect structures in the trajectory of a random walk built from digits. The test checks whether maximum deviation from zero has the proper distribution.

\textbf{Mathematical Formulas:}

\begin{equation}
S[i] = \sum_{j=0}^{i} (2 \cdot \text{binary}[j] - 1)
\tag{110}
\end{equation}

\begin{equation}
\text{where: binary}[j] = \text{digits}[j] \bmod 2 = \text{conversion to binary}
\tag{111}
\end{equation}

\begin{equation}
E[\max|S|] \approx \sqrt{\frac{2n}{\pi}}
\tag{112}
\end{equation}

\begin{equation}
Z = \frac{\max|S| - E[\max|S|]}{\text{std}(S) / \sqrt{n}}
\tag{113}
\end{equation}

\begin{equation}
p\text{-value} = 2 \cdot (1 - \Phi(|Z|))
\tag{114}
\end{equation}


\subsection{Analysis Parameters}

\begin{table}[H]
\centering
\begin{tabular}{lr}
\toprule
\textbf{Parameter} & \textbf{Value} \\
\midrule
Sample & 10,000,000,000 digits \\
Number of tests & 27 \\
Significance level & $\alpha = 0.05$ \\
Total analysis time & 6.47 hours \\
Average time per test & 862.7 seconds \\
\bottomrule
\end{tabular}
\caption{Statistical Analysis Parameters}
\label{tab:parameters}
\end{table}

\section{Results}
\label{sec:results}

\subsection{Results Summary}

Analysis of 27 statistical tests on a sample of 10 billion digits of $\pi$ showed mixed results, 
confirming both local randomness and limits of randomness on a large scale.

\subsubsection{Key PASS Tests (Confirmation of Local Randomness)}

Basic statistical tests confirm local randomness of $\pi$:

\begin{table}[H]
\centering
\adjustbox{width=0.95\textwidth,center}{
\begin{tabular}{lcp{5.5cm}}
\toprule
\textbf{Test ID} & \textbf{p-value} & \textbf{Test Name} \\
\midrule
1 & 0.309623 & Frequency Test (NIST) \\
2 & 0.278108 & Runs Test (NIST) \\
3 & 1.000000 & Block Frequency Test (NIST) \\
11 & 0.923391 & Serial Test (NIST) \\
15 & 0.801912 & Universal Statistical Test (NIST) \\
17 & 0.770520 & Overlapping Template Matching Test (NIST) \\
19 & 1.000000 & Collision Test (SmallCrush) \\
20 & 0.538007 & Gap Test (SmallCrush) \\
22 & 0.264214 & CouponCollector Test \\
24 & 0.240062 & WeightDistrib Test \\
26 & 0.818876 & HammingIndep Test \\
\bottomrule
\end{tabular}
}
\caption{Tests confirming local randomness of $\pi$ (p-value $> 0.05$)}
\label{tab:pass_tests}
\end{table}

\subsubsection{Critical FAIL Tests (Limits of Randomness)}

Advanced tests detected mathematical structures indicating limits of randomness:

\begin{table}[H]
\centering
\adjustbox{width=0.95\textwidth,center}{
\begin{tabular}{lcp{3.5cm}p{4.5cm}}
\toprule
\textbf{Test ID} & \textbf{p-value} & \textbf{Name} & \textbf{Interpretation} \\
\midrule
9 & 0.041575 & Cumulative Sums Test (NIST) & Mathematical structure detected \\
10 & 0.001565 & Approximate Entropy Test (NIST) & Mathematical structure detected \\
12 & $2.71e-11$ & Linear Complexity Test (NIST) & Mathematical structure detected \\
13 & $< 10^{-10}$ & Random Excursions Test (NIST) & FAIL: $\chi^2 > 18$k, mean visits 1.97-8.52 vs expected 0.125-0.5 \\
14 & $< 10^{-10}$ & Random Excursions Variant Test (NIST) & FAIL: observed 4k vs expected 500k-5M visits for states $\pm 1$--$\pm 9$ \\
16 & $2.23e-11$ & Non-overlapping Template Matching Test (NIST) & FAIL: pattern has too few matches (18,303 vs 19,231 expected) \\
18 & $< 10^{-10}$ & BirthdaySpacings Test (SmallCrush) & FAIL: $\chi^2 = 91$M, extreme deviations in spacing distribution \\
21 & $< 10^{-10}$ & SimplePoker Test & FAIL: deviations in digit combination distribution in blocks \\
23 & $< 10^{-10}$ & MaxOft Test & FAIL: deviations in extreme value distribution \\
27 & $< 10^{-10}$ & RandomWalk1 Test & FAIL: deviations in maximum random walk deviation \\
\bottomrule
\end{tabular}
}
\caption{Critical tests showing limits of randomness of $\pi$ (p-value $< 0.05$)}
\label{tab:fail_tests}
\end{table}

\subsection{Visualizations}

\begin{figure}[H]
\centering
\includegraphics[width=\textwidth]{figures/fig01_pvalues_EN.pdf}
\caption{P-values for all statistical tests. Green bars indicate tests with p-value $> 0.05$, 
red -- tests with p-value $< 0.05$. Orange dashed line indicates significance threshold $\alpha = 0.05$.}
\label{fig:pvalues}
\end{figure}

\begin{figure}[H]
\centering
\includegraphics[width=\textwidth]{figures/fig02_execution_times_EN.pdf}
\caption{Execution times for individual tests. Red dashed line indicates mean execution time.}
\label{fig:times}
\end{figure}

\begin{figure}[H]
\centering
\includegraphics[width=\textwidth]{figures/fig05_pvalue_histogram_EN.pdf}
\caption{Histogram of p-values for tests with p-value $> 0$. Red dashed line indicates significance threshold.}
\label{fig:hist}
\end{figure}

\begin{figure}[H]
\centering
\includegraphics[width=0.9\textwidth]{figures/fig07_nist_vs_testu01_EN.pdf}
\caption{Comparison of results for NIST Statistical Test Suite and TestU01 SmallCrush packages.}
\label{fig:comparison}
\end{figure}

\begin{figure}[H]
\centering
\includegraphics[width=\textwidth]{figures/fig08_random_excursions_EN.pdf}
\caption{Test 13: Random Excursions - Comparison of observed and expected mean visits in random walk states. 
The graph shows dramatic deviations in extreme states ($\pm 3$, $\pm 4$), where observed values are significantly higher than expected.}
\label{fig:random_excursions}
\end{figure}

\begin{figure}[H]
\centering
\includegraphics[width=\textwidth]{figures/fig09_random_excursions_variant_EN.pdf}
\caption{Test 14: Random Excursions Variant - Comparison of observed and expected visits for states in the range $\{-9, \ldots, 9\}$. 
Observed values are 2-3 orders of magnitude lower than expected, indicating a strong mathematical structure.}
\label{fig:random_excursions_variant}
\end{figure}

\subsection{Test Frequency - Detailed Results}

\begin{figure}[H]
\centering
\includegraphics[width=0.9\textwidth]{figures/fig03_frequencies_EN.pdf}
\caption{Digit frequencies 0-9 in Frequency test. Red line indicates expected frequency.}
\label{fig:frequencies}
\end{figure}

\subsection{Compression Test - Detailed Results}

\begin{figure}[H]
\centering
\includegraphics[width=0.7\textwidth]{figures/fig06_compression_EN.pdf}
\caption{Compression ratio for compression test. Green line indicates expected value for random data.}
\label{fig:compression}
\end{figure}

\subsection{Entropy Test - Detailed Results}

\begin{figure}[H]
\centering
\includegraphics[width=0.9\textwidth]{figures/fig04_entropy_by_N_EN.pdf}
\caption{Shannon entropy vs block length N. Red line indicates maximum entropy.}
\label{fig:entropy}
\end{figure}

\subsection{Table of Results for All Tests}

\begin{longtable}{p{0.05\textwidth}p{0.35\textwidth}cccp{0.20\textwidth}}
\toprule
ID & Test & p-value & Time (s) & Result \\
\midrule
\endfirsthead
\toprule
ID & Test & p-value & Time (s) & Result \\
\midrule
\endhead
\midrule
\multicolumn{5}{r}{\textit{continued on next page}} \\
\endfoot
\bottomrule
\endlastfoot
1 & Frequency Test (NIST) & 0.309623 & 556.4 & No deviations from randomness \\
2 & Runs Test (NIST) & 0.278108 & 1211.2 & No deviations from randomness \\
3 & Block Frequency Test (NIST) & 1.000000 & 301.7 & No deviations from randomness \\
4 & Entropy Analysis & --- & 1775.1 & Analytical test (no p-value) \\
5 & Spectral FFT Analysis & --- & 14.3 & Analytical test (no p-value) \\
6 & Compression Test & --- & 1090.8 & Analytical test (no p-value) \\
7 & Empirical Entropy Bounds & --- & 983.4 & Analytical test (no p-value) \\
8 & ML LSTM Anomaly Detection & --- & 0.0 & Analytical test (no p-value) \\
9 & Cumulative Sums Test (NIST) & 0.041575 & 2.0 & Deviation from randomness detected \\
10 & Approximate Entropy Test (NIST) & 0.001565 & 40.4 & Deviation from randomness detected \\
11 & Serial Test (NIST) & 0.923391 & 20.9 & No deviations from randomness \\
12 & Linear Complexity Test (NIST) & $2.71e-11$ & 1365.1 & Deviation from randomness detected \\
13 & Random Excursions Test (NIST) & $< 10^{-10}$ & 988.9 & Deviation from randomness detected \\
14 & Random Excursions Variant Test (NIST) & $< 10^{-10}$ & 934.9 & Deviation from randomness detected \\
15 & Universal Statistical Test (NIST) & 0.801912 & 1428.6 & No deviations from randomness \\
16 & Non-overlapping Template Matching Test (NIST) & $2.23e-11$ & 1728.0 & Deviation from randomness detected \\
17 & Overlapping Template Matching Test (NIST) & 0.770520 & 1596.2 & No deviations from randomness \\
18 & BirthdaySpacings Test (SmallCrush) & $< 10^{-10}$ & 948.6 & Deviation from randomness detected \\
19 & Collision Test (SmallCrush) & 1.000000 & 930.4 & No deviations from randomness \\
20 & Gap Test (SmallCrush) & 0.538007 & 915.6 & No deviations from randomness \\
21 & SimplePoker Test & $< 10^{-10}$ & 916.2 & Deviation from randomness detected \\
22 & CouponCollector Test & 0.264214 & 924.4 & No deviations from randomness \\
23 & MaxOft Test & $< 10^{-10}$ & 922.6 & Deviation from randomness detected \\
24 & WeightDistrib Test & 0.240062 & 928.8 & No deviations from randomness \\
25 & MatrixRank Test & --- & 920.4 & Analytical test (no p-value) \\
26 & HammingIndep Test & 0.818876 & 924.6 & No deviations from randomness \\
27 & RandomWalk1 Test & $< 10^{-10}$ & 924.7 & Deviation from randomness detected \\
\end{longtable}

\section{Detailed Analysis of Results}
\label{sec:detailed-analysis}

In this section, we present a detailed analysis of the results of each test, along with interpretation 
in the context of the statistical properties of $\pi$.

\newpage
\subsection{Test 01: Frequency Test (NIST)}
\label{sec:test01}

\subsubsection{Purpose and Application of the Test}

\textbf{Purpose:}

Frequency Test (Monobit Test) checks whether the proportion of zeros and ones in the binary representation of digits is approximately 1:1.

\textbf{Application:}

This is the most basic randomness test. It serves to verify uniform distribution of bits in a binary sequence. It tests the null hypothesis that the sequence is random by comparing the frequency of occurrence of each digit with the expected frequency.

\subsubsection{Mathematical Formulas}

The test is based on the following mathematical formulas:

\begin{equation}
\chi^2 = \sum_{i=0}^{9}\frac{(f_i - n/10)^2}{n/10}
\tag{115}
\end{equation}

\begin{equation}
E[f_i] = \frac{n}{10} = \text{expected frequency of each digit}
\tag{116}
\end{equation}

\begin{equation}
p\text{-value} = 1 - \text{CDF}(\chi^2, \text{df} = 9)
\tag{117}
\end{equation}

\begin{equation}
\text{where: } f_i = \text{frequency of digit } i \text{ (0-9), } n = \text{total number of digits}
\tag{118}
\end{equation}

\subsubsection{Testing Methodology}

\begin{itemize}
\item Sample: 10,000,000,000 decimal digits of $\pi$
\item Implementation: Test performed according to guidelines of NIST Statistical Test Suite
\item Execution time: 556.4 seconds (9.3 minutes)
\end{itemize}

\subsubsection{Results for $\pi$}

\begin{table}[H]
\centering
\begin{tabular}{ll}
\toprule
\textbf{Parameter} & \textbf{Value} \\
\midrule
Number of digits & 10,000,000,000 \\
P-value & 0.309623 \\
$\chi^2$ & 10.525717 \\
Digit frequencies & see graph \\\\
\bottomrule
\end{tabular}
\caption{Results of Test 01: Frequency Test (NIST)}
\label{tab:test01}
\end{table}

\subsubsection{Results Interpretation}

Test 01 showed no statistically significant deviations from the randomness hypothesis (p-value = 0.309623). This result indicates that $\pi$ digits exhibit properties consistent with expectations for a random sequence in the range tested by this test. A p-value above the significance threshold $\alpha = 0.05$ means there are no grounds to reject the null hypothesis of randomness.

\newpage
\subsection{Test 02: Runs Test (NIST)}
\label{sec:test02}

\subsubsection{Purpose and Application of the Test}

\textbf{Purpose:}

Runs Test analyzes uninterrupted sequences of consecutive zeros or ones (runs).

\textbf{Application:}

Serves to detect correlations between consecutive bits. Checks whether transitions between 0 and 1 occur with expected frequency.

\subsubsection{Mathematical Formulas}

The test is based on the following mathematical formulas:

\begin{equation}
E[\text{runs}] = 2 \cdot \text{ones} \cdot \text{zeros} / n
\tag{119}
\end{equation}

\begin{equation}
\text{Var}[\text{runs}] = \frac{2 \cdot \text{ones} \cdot \text{zeros} \cdot (2 \cdot \text{ones} \cdot \text{zeros} - n)}{n^2 \cdot (n - 1)}
\tag{120}
\end{equation}

\begin{equation}
Z = \frac{\text{runs} - E[\text{runs}]}{\sqrt{\text{Var}[\text{runs}]}}
\tag{121}
\end{equation}

\begin{equation}
p\text{-value} = 2 \cdot (1 - \Phi(|Z|))
\tag{122}
\end{equation}

\begin{equation}
\text{where: ones = number of odd digits, zeros = number of even digits}
\tag{123}
\end{equation}

\subsubsection{Testing Methodology}

\begin{itemize}
\item Sample: 10,000,000,000 decimal digits of $\pi$
\item Implementation: Test performed according to guidelines of NIST Statistical Test Suite
\item Execution time: 1211.2 seconds (20.2 minutes)
\end{itemize}

\subsubsection{Results for $\pi$}

\begin{table}[H]
\centering
\begin{tabular}{ll}
\toprule
\textbf{Parameter} & \textbf{Value} \\
\midrule
Number of digits & 10,000,000,000 \\
P-value & 0.278108 \\
Z-score & 1.084580 \\
Number of runs & 5,000,054,227 \\
Expected number of runs & 4999999998.02 \\
\bottomrule
\end{tabular}
\caption{Results of Test 02: Runs Test (NIST)}
\label{tab:test02}
\end{table}

\subsubsection{Results Interpretation}

Test 02 showed no statistically significant deviations from the randomness hypothesis (p-value = 0.278108). This result indicates that $\pi$ digits exhibit properties consistent with expectations for a random sequence in the range tested by this test. A p-value above the significance threshold $\alpha = 0.05$ means there are no grounds to reject the null hypothesis of randomness.

\newpage
\subsection{Test 03: Block Frequency Test (NIST)}
\label{sec:test03}

\subsubsection{Purpose and Application of the Test}

\textbf{Purpose:}

Block Frequency Test divides the sequence into blocks and checks the frequency of ones in each block.

\textbf{Application:}

Serves to detect local non-uniformities in bit distribution at the block level.

\subsubsection{Mathematical Formulas}

The test is based on the following mathematical formulas:

\begin{equation}
\chi^2 = \sum_{j} \frac{(\text{ones\_per\_block}_j - \text{block\_size} / 2)^2}{\text{block\_size} / 2}
\tag{124}
\end{equation}

\begin{equation}
E[\text{ones}] = \frac{\text{block\_size}}{2} = \text{expected number of ones in block}
\tag{125}
\end{equation}

\begin{equation}
p\text{-value} = 1 - \text{CDF}(\chi^2, \text{df} = \text{num\_blocks})
\tag{126}
\end{equation}

\begin{equation}
\text{where: ones\_per\_block = number of ones in block } j
\tag{127}
\end{equation}

\subsubsection{Testing Methodology}

\begin{itemize}
\item Sample: 10,000,000,000 decimal digits of $\pi$
\item Implementation: Test performed according to guidelines of NIST Statistical Test Suite
\item Execution time: 301.7 seconds (5.0 minutes)
\item Block size: 10,000
\item Number of blocks: 1,000,000
\end{itemize}

\subsubsection{Results for $\pi$}

\begin{table}[H]
\centering
\begin{tabular}{ll}
\toprule
\textbf{Parameter} & \textbf{Value} \\
\midrule
Number of digits & 10,000,000,000 \\
P-value & 1.000000 \\
$\chi^2$ & 500214.465800 \\
\bottomrule
\end{tabular}
\caption{Results of Test 03: Block Frequency Test (NIST)}
\label{tab:test03}
\end{table}

\subsubsection{Results Interpretation}

Test 03 showed no statistically significant deviations from the randomness hypothesis (p-value = 1.000000). This result indicates that $\pi$ digits exhibit properties consistent with expectations for a random sequence in the range tested by this test. A p-value above the significance threshold $\alpha = 0.05$ means there are no grounds to reject the null hypothesis of randomness.

\newpage
\subsection{Test 04: Entropy Analysis}
\label{sec:test04}

\subsubsection{Purpose and Application of the Test}

\textbf{Purpose:}

Entropy Analysis calculates Shannon entropy for the digit distribution.

\textbf{Application:}

Serves to measure unpredictability and complexity of the sequence. High entropy indicates high randomness.

\subsubsection{Mathematical Formulas}

The test is based on the following mathematical formulas:

\begin{equation}
H(X) = -\sum_{x=0}^{9} p(x) \cdot \log_2(p(x))
\tag{128}
\end{equation}

\begin{equation}
p(x) = \frac{\text{count}(x)}{n} = \text{probability of digit } x
\tag{129}
\end{equation}

\begin{equation}
H_{\max} = \log_2(10) \approx 3.321928 = \text{maximum entropy for 10 digits}
\tag{130}
\end{equation}

\begin{equation}
\text{ratio} = \frac{H(X)}{H_{\max}}
\tag{131}
\end{equation}

\subsubsection{Testing Methodology}

\begin{itemize}
\item Sample: 10,000,000,000 decimal digits of $\pi$
\item Implementation: Test performed according to guidelines of NIST Statistical Test Suite
\item Execution time: 1775.1 seconds (29.6 minutes)
\end{itemize}

\subsubsection{Results for $\pi$}

\begin{table}[H]
\centering
\begin{tabular}{ll}
\toprule
\textbf{Parameter} & \textbf{Value} \\
\midrule
Number of digits & 10,000,000,000 \\
P-value & none (analytical test) \\\\
Global entropy & 3.321928 \\
Maximum entropy & 3.321928 \\
Entropy ratio & 1.000000 \\
\bottomrule
\end{tabular}
\caption{Results of Test 04: Entropy Analysis}
\label{tab:test04}
\end{table}

\subsubsection{Results Interpretation}

Test 04 is an analytical test that does not generate p-values. Results provide information about the statistical properties of $\pi$ digits in the range tested by this test. Analysis is based on direct measurement of sequence properties, such as entropy, compression ratio, or other statistical measures.

\newpage
\subsection{Test 05: Spectral FFT Analysis}
\label{sec:test05}

\subsubsection{Purpose and Application of the Test}

\textbf{Purpose:}

Spectral FFT Analysis uses Fourier transform to detect periodicity.

\textbf{Application:}

Serves to detect hidden periodic patterns in the digit sequence.

\subsubsection{Mathematical Formulas}

The test is based on the following mathematical formulas:

\begin{equation}
X[k] = \sum_{n=0}^{N-1} x[n] \cdot e^{-2\pi ikn / N}
\tag{132}
\end{equation}

\begin{equation}
P[k] = |X[k]|^2 = \text{power spectrum}
\tag{133}
\end{equation}

\begin{equation}
H_s = -\sum_k \frac{P[k]}{\sum P} \cdot \log_2\left(\frac{P[k]}{\sum P} + \varepsilon\right)
\tag{134}
\end{equation}

\begin{equation}
\text{where: } x[n] = \text{digit pairs}(\text{digits}[i] \cdot 10 + \text{digits}[i + 1]), \varepsilon = 10^{-10}
\tag{135}
\end{equation}

\subsubsection{Testing Methodology}

\begin{itemize}
\item Sample: 10,000,000,000 decimal digits of $\pi$
\item Implementation: Test performed according to guidelines of NIST Statistical Test Suite
\item Execution time: 14.3 seconds (0.2 minutes)
\item Window size: 1,000,000
\end{itemize}

\subsubsection{Results for $\pi$}

\begin{table}[H]
\centering
\begin{tabular}{ll}
\toprule
\textbf{Parameter} & \textbf{Value} \\
\midrule
Number of digits & 10,000,000,000 \\
P-value & none (analytical test) \\\\
Spectral entropy & 5.714473 \\
Number of detected spectral gaps & 50,000 \\
\bottomrule
\end{tabular}
\caption{Results of Test 05: Spectral FFT Analysis}
\label{tab:test05}
\end{table}

\subsubsection{Results Interpretation}

Test 05 is an analytical test that does not generate p-values. Results provide information about the statistical properties of $\pi$ digits in the range tested by this test. Analysis is based on direct measurement of sequence properties, such as entropy, compression ratio, or other statistical measures.

\newpage
\subsection{Test 06: Compression Test}
\label{sec:test06}

\subsubsection{Purpose and Application of the Test}

\textbf{Purpose:}

Compression Test measures the degree of data compression using zlib algorithm.

\textbf{Application:}

Serves to assess sequence complexity. Low compression indicates high complexity and randomness.

\subsubsection{Mathematical Formulas}

The test is based on the following mathematical formulas:

\begin{equation}
\text{compression\_ratio} = \frac{\text{compressed\_size}}{\text{original\_size}}
\tag{136}
\end{equation}

\begin{equation}
\text{where: original\_size = size of original data, compressed\_size = size after zlib compression}
\tag{137}
\end{equation}

\begin{equation}
\text{Interpretation: Lower ratio = higher randomness}
\tag{138}
\end{equation}

\subsubsection{Testing Methodology}

\begin{itemize}
\item Sample: 10,000,000,000 decimal digits of $\pi$
\item Implementation: Test performed according to guidelines of NIST Statistical Test Suite
\item Execution time: 1090.8 seconds (18.2 minutes)
\item Analyzed sample size: 100,000,000
\end{itemize}

\subsubsection{Results for $\pi$}

\begin{table}[H]
\centering
\begin{tabular}{ll}
\toprule
\textbf{Parameter} & \textbf{Value} \\
\midrule
Number of digits & 10,000,000,000 \\
P-value & none (analytical test) \\\\
Compression ratio & 0.469249 \\
\bottomrule
\end{tabular}
\caption{Results of Test 06: Compression Test}
\label{tab:test06}
\end{table}

\subsubsection{Results Interpretation}

Test 06 is an analytical test that does not generate p-values. Results provide information about the statistical properties of $\pi$ digits in the range tested by this test. Analysis is based on direct measurement of sequence properties, such as entropy, compression ratio, or other statistical measures.

\newpage
\subsection{Test 07: Empirical Entropy Bounds}
\label{sec:test07}

\subsubsection{Purpose and Application of the Test}

\textbf{Purpose:}

Empirical Entropy Bounds analyzes entropy limits for different block lengths.

\textbf{Application:}

Serves to study how entropy changes depending on the length of analyzed blocks.

\subsubsection{Mathematical Formulas}

The test is based on the following mathematical formulas:

\begin{equation}
H(N) = \log_2(10) \cdot \left(1 - \frac{c}{\log(N)}\right)
\tag{139}
\end{equation}

\begin{equation}
c = \arg \min \sum (H_{\text{observed}}(N) - H_{\text{model}}(N,c))^2
\tag{140}
\end{equation}

\begin{equation}
H_{\max} = \log_2(10) \approx 3.321928
\tag{141}
\end{equation}

\begin{equation}
\text{Confidence interval (95\%): } \text{CI} = c \pm 1.96 \cdot \sigma_c
\tag{142}
\end{equation}

\begin{equation}
\text{where: } N = \text{number of analyzed digits, } c = \text{fitting parameter}
\tag{143}
\end{equation}

\subsubsection{Testing Methodology}

\begin{itemize}
\item Sample: 10,000,000,000 decimal digits of $\pi$
\item Implementation: Test performed according to guidelines of NIST Statistical Test Suite
\item Execution time: 983.4 seconds (16.4 minutes)
\end{itemize}

\subsubsection{Results for $\pi$}

\begin{table}[H]
\centering
\begin{tabular}{ll}
\toprule
\textbf{Parameter} & \textbf{Value} \\
\midrule
Number of digits & 10,000,000,000 \\
P-value & none (analytical test) \\\\
Maximum entropy & 3.321928 \\
\bottomrule
\end{tabular}
\caption{Results of Test 07: Empirical Entropy Bounds}
\label{tab:test07}
\end{table}

\subsubsection{Results Interpretation}

Test 07 is an analytical test that does not generate p-values. Results provide information about the statistical properties of $\pi$ digits in the range tested by this test. Analysis is based on direct measurement of sequence properties, such as entropy, compression ratio, or other statistical measures.

\newpage
\subsection{Test 08: ML LSTM Anomaly Detection}
\label{sec:test08}

\subsubsection{Purpose and Application of the Test}

\textbf{Purpose:}

ML LSTM Anomaly Detection uses an LSTM neural network to detect anomalies.

\textbf{Application:}

Serves to detect patterns and anomalies in the digit sequence using machine learning. The network attempts to predict the next digit based on previous ones.

\subsubsection{Mathematical Formulas}

The test is based on the following mathematical formulas:

\begin{equation}
\text{Accuracy} = \frac{1}{m} \sum_{i=1}^{m} \mathbf{1}[\hat{d}_i = d_i]
\tag{144}
\end{equation}

\subsubsection{Testing Methodology}

\begin{itemize}
\item Sample: 10,000,000,000 decimal digits of $\pi$
\item Implementation: Test performed according to guidelines of NIST Statistical Test Suite
\item Execution time: 0.0 seconds (0.0 minutes)
\end{itemize}

\subsubsection{Results for $\pi$}

\begin{table}[H]
\centering
\begin{tabular}{ll}
\toprule
\textbf{Parameter} & \textbf{Value} \\
\midrule
Number of digits & 10,000,000,000 \\
P-value & none (analytical test) \\\\
\bottomrule
\end{tabular}
\caption{Results of Test 08: ML LSTM Anomaly Detection}
\label{tab:test08}
\end{table}

\subsubsection{Results Interpretation}

Test 08 is an analytical test that does not generate p-values. Results provide information about the statistical properties of $\pi$ digits in the range tested by this test. Analysis is based on direct measurement of sequence properties, such as entropy, compression ratio, or other statistical measures.

\newpage
\subsection{Test 09: Cumulative Sums Test (NIST)}
\label{sec:test09}

\subsubsection{Purpose and Application of the Test}

\textbf{Purpose:}

Cumulative Sums Test analyzes maximum deviation of cumulative sums.

\textbf{Application:}

Serves to detect systematic trends in the bit sequence.

\subsubsection{Mathematical Formulas}

The test is based on the following mathematical formulas:

\begin{equation}
S_{\text{forward}}[i] = \sum_{j=0}^{i}(2 \cdot \text{binary}[j] - 1)
\tag{145}
\end{equation}

\begin{equation}
S_{\text{backward}}[i] = \sum_{j=i}^{n}(2 \cdot \text{binary}[j] - 1)
\tag{146}
\end{equation}

\begin{equation}
\max_{\text{forward}} = \max_i |S_{\text{forward}}[i]|, \quad \max_{\text{backward}} = \max_i |S_{\text{backward}}[i]|
\tag{147}
\end{equation}

\begin{equation}
Z_{\text{forward}} = \frac{\max_{\text{forward}}}{\sqrt{n}}, \quad Z_{\text{backward}} = \frac{\max_{\text{backward}}}{\sqrt{n}}
\tag{148}
\end{equation}

\begin{equation}
p\text{-value} = \min(p_{\text{forward}}, p_{\text{backward}})
\tag{149}
\end{equation}

\subsubsection{Testing Methodology}

\begin{itemize}
\item Sample: 100,000,000 decimal digits of $\pi$
\item Implementation: Test performed according to guidelines of NIST Statistical Test Suite
\item Execution time: 2.0 seconds (0.0 minutes)
\end{itemize}

\subsubsection{Results for $\pi$}

\begin{table}[H]
\centering
\begin{tabular}{ll}
\toprule
\textbf{Parameter} & \textbf{Value} \\
\midrule
Number of digits & 100,000,000 \\
P-value & 0.041575 \\
\bottomrule
\end{tabular}
\caption{Results of Test 09: Cumulative Sums Test (NIST)}
\label{tab:test09}
\end{table}

\subsubsection{Results Interpretation}

Test 09 showed a statistically significant deviation from the randomness hypothesis (p-value = 0.041575). This result indicates detection of mathematical structure in the distribution of $\pi$ digits, which is a valuable scientific discovery characteristic of a deterministic mathematical constant. A p-value below the significance threshold $\alpha = 0.05$ means the sequence exhibits deviations from a perfectly random distribution in the range tested by this test. This is the first detection of such structure on a sample of 10 billion digits.

\newpage
\subsection{Test 10: Approximate Entropy Test (NIST)}
\label{sec:test10}

\subsubsection{Purpose and Application of the Test}

\textbf{Purpose:}

Approximate Entropy Test measures regularity of patterns of given length.

\textbf{Application:}

Serves to detect regular patterns in the sequence. Low approximate entropy indicates predictability.

\subsubsection{Mathematical Formulas}

The test is based on the following mathematical formulas:

\begin{equation}
\text{ApEn}(m,r) = \Phi^m(r) - \Phi^{m+1}(r)
\tag{150}
\end{equation}

\begin{equation}
\Phi^m(r) = \frac{1}{N-m+1}\sum_{i=1}^{N-m+1}\log C_i^m(r)
\tag{151}
\end{equation}

\begin{equation}
C_i^m(r) = \frac{\text{number of patterns of length } m \text{ similar to } x[i:i+m]}{N-m+1}
\tag{152}
\end{equation}

\begin{equation}
\chi^2 = \frac{(\text{ApEn} - E[\text{ApEn}])^2}{\text{Var}[\text{ApEn}]}, \quad p\text{-value} = 1 - \text{CDF}(\chi^2, \text{df}=1)
\tag{153}
\end{equation}

\subsubsection{Testing Methodology}

\begin{itemize}
\item Sample: 100,000,000 decimal digits of $\pi$
\item Implementation: Test performed according to guidelines of NIST Statistical Test Suite
\item Execution time: 40.4 seconds (0.7 minutes)
\item Analyzed sample size: 10,000,000
\item Parameter $m$ (pattern length): 2
\end{itemize}

\subsubsection{Results for $\pi$}

\begin{table}[H]
\centering
\begin{tabular}{ll}
\toprule
\textbf{Parameter} & \textbf{Value} \\
\midrule
Number of digits & 100,000,000 \\
P-value & 0.001565 \\
$\chi^2$ & 9.999995 \\
Approximate entropy & 1.000000 \\
\bottomrule
\end{tabular}
\caption{Results of Test 10: Approximate Entropy Test (NIST)}
\label{tab:test10}
\end{table}

\subsubsection{Results Interpretation}

Test 10 showed a statistically significant deviation from the randomness hypothesis (p-value = 0.001565). This result indicates detection of mathematical structure in the distribution of $\pi$ digits, which is a valuable scientific discovery characteristic of a deterministic mathematical constant. A p-value below the significance threshold $\alpha = 0.05$ means the sequence exhibits deviations from a perfectly random distribution in the range tested by this test. This is the first detection of such structure on a sample of 10 billion digits.

\newpage
\subsection{Test 11: Serial Test (NIST)}
\label{sec:test11}

\subsubsection{Purpose and Application of the Test}

\textbf{Purpose:}

Serial Test analyzes frequency of overlapping patterns of length $m$.

\textbf{Application:}

Serves to detect preferences for certain patterns over others.

\subsubsection{Mathematical Formulas}

The test is based on the following mathematical formulas:

\begin{equation}
\Delta\psi_m^2 = \psi_m^2 - \psi_{m-1}^2
\tag{154}
\end{equation}

\begin{equation}
\psi_m^2 = \frac{2^m}{n}\sum(\text{obs}_i^2) - n
\tag{155}
\end{equation}

\begin{equation}
\text{where: obs}_i = \text{number of occurrences of pattern } i \text{ of length } m
\tag{156}
\end{equation}

\begin{equation}
p\text{-value} = 1 - \text{CDF}(\Delta\psi_m^2, \text{df} = 2^{m-1})
\tag{157}
\end{equation}

\subsubsection{Testing Methodology}

\begin{itemize}
\item Sample: 100,000,000 decimal digits of $\pi$
\item Implementation: Test performed according to guidelines of NIST Statistical Test Suite
\item Execution time: 20.9 seconds (0.3 minutes)
\item Analyzed sample size: 10,000,000
\end{itemize}

\subsubsection{Results for $\pi$}

\begin{table}[H]
\centering
\begin{tabular}{ll}
\toprule
\textbf{Parameter} & \textbf{Value} \\
\midrule
Number of digits & 100,000,000 \\
P-value & 0.923391 \\
\bottomrule
\end{tabular}
\caption{Results of Test 11: Serial Test (NIST)}
\label{tab:test11}
\end{table}

\subsubsection{Results Interpretation}

Test 11 showed no statistically significant deviations from the randomness hypothesis (p-value = 0.923391). This result indicates that $\pi$ digits exhibit properties consistent with expectations for a random sequence in the range tested by this test. A p-value above the significance threshold $\alpha = 0.05$ means there are no grounds to reject the null hypothesis of randomness.

\newpage
\subsection{Test 12: Linear Complexity Test (NIST)}
\label{sec:test12}

\subsubsection{Purpose and Application of the Test}

\textbf{Purpose:}

Linear Complexity Test measures the length of the shortest LFSR generating the sequence.

\textbf{Application:}

Serves to assess linear complexity of the sequence. Low complexity indicates linear patterns.

\subsubsection{Mathematical Formulas}

The test is based on the following mathematical formulas:

\begin{equation}
L = \text{Berlekamp-Massey}(S) = \text{length of shortest LFSR}
\tag{158}
\end{equation}

\begin{equation}
E[L] = \frac{M}{2} + \frac{9 + ((-1)^{M+1})}{36}
\tag{159}
\end{equation}

\begin{equation}
\chi^2 = \sum \frac{(\text{observed\_complexities} - E[L])^2}{E[L]}
\tag{160}
\end{equation}

\begin{equation}
p\text{-value} = 1 - \text{CDF}(\chi^2, \text{df} = \text{num\_bins} - 1)
\tag{161}
\end{equation}

\begin{equation}
\text{where: } M = \text{binary block length}
\tag{162}
\end{equation}

\subsubsection{Testing Methodology}

\begin{itemize}
\item Sample: 10,000,000,000 decimal digits of $\pi$
\item Implementation: Test performed according to guidelines of NIST Statistical Test Suite
\item Execution time: 1365.1 seconds (22.8 minutes)
\item Analyzed sample size: 1,000,000
\item Block size: 500
\item Number of blocks: 2,000
\end{itemize}

\subsubsection{Results for $\pi$}

\begin{table}[H]
\centering
\begin{tabular}{ll}
\toprule
\textbf{Parameter} & \textbf{Value} \\
\midrule
Number of digits & 10,000,000,000 \\
P-value & $2.71e-11$ \\
$\chi^2$ & 88.475442 \\
Mean linear complexity & 250.21 \\
Expected complexity & 250.22 \\
\bottomrule
\end{tabular}
\caption{Results of Test 12: Linear Complexity Test (NIST)}
\label{tab:test12}
\end{table}

\subsubsection{Results Interpretation}

Test 12 showed a statistically significant deviation from the randomness hypothesis (p-value = $2.71e-11$). This result indicates detection of mathematical structure in the distribution of $\pi$ digits, which is a valuable scientific discovery characteristic of a deterministic mathematical constant. A p-value below the significance threshold $\alpha = 0.05$ means the sequence exhibits deviations from a perfectly random distribution in the range tested by this test. This is the first detection of such structure on a sample of 10 billion digits.

\newpage
\subsection{Test 13: Random Excursions Test (NIST)}
\label{sec:test13}

\subsubsection{Purpose and Application of the Test}

\textbf{Purpose:}

Random Excursions Test analyzes a random walk built from a binary sequence.

\textbf{Application:}

Serves to detect structures in the trajectory of a random walk. Checks the distribution of visits to specific states.

\subsubsection{Mathematical Formulas}

The test is based on the following mathematical formulas:

\begin{equation}
S_k = \sum_{i=1}^{k}(2 \cdot \text{binary}[i] - 1) = \text{random walk}
\tag{163}
\end{equation}

\begin{equation}
\xi(x) = \text{number of visits to state } x \text{ for } x \in \{-4, -3, -2, -1, 1, 2, 3, 4\}
\tag{164}
\end{equation}

\begin{equation}
E[\xi(x)] = \frac{1}{2|x|(|x|+1)}, \quad \text{Var}[\xi(x)] = \frac{4|x|(J-|x|-1)}{(J-1)^2(2|x|+1)}
\tag{165}
\end{equation}

\begin{equation}
\chi^2 = \sum_{x} \frac{(\xi(x) - E[\xi(x)])^2}{E[\xi(x)]}, \quad p\text{-value} = 1 - \text{CDF}(\chi^2, \text{df} = 7)
\tag{166}
\end{equation}

\subsubsection{Testing Methodology}

\begin{itemize}
\item Sample: 10,000,000,000 decimal digits of $\pi$
\item Implementation: Test performed according to guidelines of NIST Statistical Test Suite
\item Execution time: 988.9 seconds (16.5 minutes)
\end{itemize}

\subsubsection{Results for $\pi$}

\begin{table}[H]
\centering
\begin{tabular}{ll}
\toprule
\textbf{Parameter} & \textbf{Value} \\
\midrule
Number of digits & 10,000,000,000 \\
P-value & $< 10^{-10}$ \\
Number of cycles & 3,294 \\
\bottomrule
\end{tabular}
\caption{Results of Test 13: Random Excursions Test (NIST)}
\label{tab:test13}
\end{table}

\subsubsection{Results Interpretation}

Random Excursions test showed critical deviation from randomness (p-value $< 10^{-10}$). Analysis revealed systematic deviations in the distribution of visits to random walk states:

\begin{itemize}
\item State -4: mean number of visits = 8.52 (expected: 0.125), $\chi^2$ = 18776.9
\item State -3: mean number of visits = 6.07 (expected: 0.167), $\chi^2$ = 13048.9
\item State -2: mean number of visits = 3.90 (expected: 0.250), $\chi^2$ = 6630.2
\item State -1: mean number of visits = 1.97 (expected: 0.500), $\chi^2$ = 1620.3
\item State 1: mean number of visits = 2.00 (expected: 0.500), $\chi^2$ = 1675.4
\item State 2: mean number of visits = 3.91 (expected: 0.250), $\chi^2$ = 6867.4
\item State 3: mean number of visits = 5.88 (expected: 0.167), $\chi^2$ = 13677.1
\item State 4: mean number of visits = 7.64 (expected: 0.125), $\chi^2$ = 20185.6
\end{itemize}

Results indicate detection of mathematical structure in the trajectory of a random walk built from $\pi$ digits. Mean numbers of visits in extreme states ($\pm 3$, $\pm 4$) are significantly higher than expected for a random sequence, suggesting the presence of long-term correlations in digit distribution. This is the first detection of such structure on a sample of 10 billion digits.

\newpage
\subsection{Test 14: Random Excursions Variant Test (NIST)}
\label{sec:test14}

\subsubsection{Purpose and Application of the Test}

\textbf{Purpose:}

Random Excursions Variant Test is a variant of Random Excursions test for a larger range of states.

\textbf{Application:}

Serves to detect structures in the trajectory of a random walk for states in the range $\{-9, \ldots, -1, 1, \ldots, 9\}$.

\subsubsection{Mathematical Formulas}

The test is based on the following mathematical formulas:

\begin{equation}
S_k = \sum_{i=1}^{k}(2 \cdot \text{binary}[i] - 1) = \text{random walk}
\tag{167}
\end{equation}

\begin{equation}
\xi(x) = \text{number of visits to state } x \text{ for } x \in \{-9, \ldots, -1, 1, \ldots, 9\}
\tag{168}
\end{equation}

\begin{equation}
E[\xi(x)] = \frac{1}{2|x|(|x|+1)}, \quad \text{Var}[\xi(x)] = \frac{4|x|(J-|x|-1)}{(J-1)^2(2|x|+1)}
\tag{169}
\end{equation}

\begin{equation}
\chi^2 = \sum_{x} \frac{(\xi(x) - E[\xi(x)])^2}{E[\xi(x)]}, \quad p\text{-value} = 1 - \text{CDF}(\chi^2, \text{df} = 17)
\tag{170}
\end{equation}

\subsubsection{Testing Methodology}

\begin{itemize}
\item Sample: 10,000,000,000 decimal digits of $\pi$
\item Implementation: Test performed according to guidelines of NIST Statistical Test Suite
\item Execution time: 934.9 seconds (15.6 minutes)
\end{itemize}

\subsubsection{Results for $\pi$}

\begin{table}[H]
\centering
\begin{tabular}{ll}
\toprule
\textbf{Parameter} & \textbf{Value} \\
\midrule
Number of digits & 10,000,000,000 \\
P-value & $< 10^{-10}$ \\
\bottomrule
\end{tabular}
\caption{Results of Test 14: Random Excursions Variant Test (NIST)}
\label{tab:test14}
\end{table}

\subsubsection{Results Interpretation}

Random Excursions Variant test showed critical deviation from randomness (p-value $< 10^{-10}$). Analysis revealed dramatic deviations in the distribution of visits for states in the range $\{-9, \ldots, 9\}$:

\begin{itemize}
\item Observed numbers of visits: 4019-4907 for all states
\item Expected numbers of visits: 555,556-5,000,000 depending on state
\item $\chi^2$ values: 545,785-4,991,965 (all $> 10^5$)
\end{itemize}

Results indicate a strong mathematical structure in the random walk trajectory. Observed numbers of visits are 2-3 orders of magnitude lower than expected, which is characteristic of a deterministic mathematical constant and indicates limits of randomness of $\pi$ on the scale of 10 billion digits.

\newpage
\subsection{Test 15: Universal Statistical Test (NIST)}
\label{sec:test15}

\subsubsection{Purpose and Application of the Test}

\textbf{Purpose:}

Universal Statistical Test checks whether the sequence can be significantly compressed.

\textbf{Application:}

Serves to detect sequence compressibility. High compressibility indicates structure.

\subsubsection{Mathematical Formulas}

The test is based on the following mathematical formulas:

\begin{equation}
f_n = \frac{1}{K}\sum_{i=1}^{K}\log_2(i - \text{last\_pos}[\text{pattern}_i])
\tag{171}
\end{equation}

\begin{equation}
E[f_n] = \begin{cases} 5.2177052 & \text{for } L=6 \\ 6.1962507 & \text{for } L=7 \\ 7.1836656 & \text{for } L=8 \end{cases}
\tag{172}
\end{equation}

\begin{equation}
\text{Var}[f_n] = \begin{cases} 2.954 & \text{for } L=6 \\ 3.125 & \text{for } L=7 \\ 3.238 & \text{for } L=8 \end{cases}
\tag{173}
\end{equation}

\begin{equation}
Z = \frac{f_n - E[f_n]}{\sqrt{\text{Var}[f_n]/K}}, \quad p\text{-value} = 2 \cdot (1 - \Phi(|Z|))
\tag{174}
\end{equation}

\begin{equation}
\text{where: } L = \text{block length, } K = \text{number of test blocks}
\tag{175}
\end{equation}

\subsubsection{Testing Methodology}

\begin{itemize}
\item Sample: 10,000,000 decimal digits of $\pi$
\item Implementation: Test performed according to guidelines of NIST Statistical Test Suite
\item Execution time: 1428.6 seconds (23.8 minutes)
\end{itemize}

\subsubsection{Results for $\pi$}

\begin{table}[H]
\centering
\begin{tabular}{ll}
\toprule
\textbf{Parameter} & \textbf{Value} \\
\midrule
Number of digits & 10,000,000 \\
P-value & 0.801912 \\
Z-score & 0.250874 \\
Statistic $f_n$ & 5.218039 \\
Expected $f_n$ & 5.217705 \\
Variance $f_n$ & 2.954000 \\
Parameter $L$ (block length) & 6 \\
Parameter $Q$ (initialization blocks) & 640 \\
Parameter $K$ (test blocks) & 1,666,026 \\
\bottomrule
\end{tabular}
\caption{Results of Test 15: Universal Statistical Test (NIST)}
\label{tab:test15}
\end{table}

\subsubsection{Results Interpretation}

Test 15 showed no statistically significant deviations from the randomness hypothesis (p-value = 0.801912). This result indicates that $\pi$ digits exhibit properties consistent with expectations for a random sequence in the range tested by this test. A p-value above the significance threshold $\alpha = 0.05$ means there are no grounds to reject the null hypothesis of randomness.

\newpage
\subsection{Test 16: Non-overlapping Template Matching Test (NIST)}
\label{sec:test16}

\subsubsection{Purpose and Application of the Test}

\textbf{Purpose:}

Non-overlapping Template Matching Test searches for non-overlapping occurrences of a pattern.

\textbf{Application:}

Serves to detect preferences for certain binary patterns through analysis of non-overlapping occurrences.

\subsubsection{Mathematical Formulas}

The test is based on the following mathematical formulas:

\begin{equation}
E[\text{matches}] = \frac{n - m + 1}{2^m} \cdot \frac{1}{2^m} = \frac{n - m + 1}{4^m}
\tag{176}
\end{equation}

\begin{equation}
\text{where: } m = \text{binary pattern length, } n = \text{sequence length}
\tag{177}
\end{equation}

\begin{equation}
\chi^2 = \frac{(\text{matches} - E[\text{matches}])^2}{E[\text{matches}]}
\tag{178}
\end{equation}

\begin{equation}
p\text{-value} = 1 - \text{CDF}(\chi^2, \text{df} = 1)
\tag{179}
\end{equation}

\subsubsection{Testing Methodology}

\begin{itemize}
\item Sample: 10,000,000,000 decimal digits of $\pi$
\item Implementation: Test performed according to guidelines of NIST Statistical Test Suite
\item Execution time: 1728.0 seconds (28.8 minutes)
\item Parameter $m$ (pattern length): 9
\end{itemize}

\subsubsection{Results for $\pi$}

\begin{table}[H]
\centering
\begin{tabular}{ll}
\toprule
\textbf{Parameter} & \textbf{Value} \\
\midrule
Number of digits & 10,000,000,000 \\
P-value & $2.23e-11$ \\
Number of tested patterns & 5 \\
\bottomrule
\end{tabular}
\caption{Results of Test 16: Non-overlapping Template Matching Test (NIST)}
\label{tab:test16}
\end{table}

\subsubsection{Results Interpretation}

Non-overlapping Template test showed statistically significant deviation (p-value = $2.23 \times 10^{-11}$). Analysis revealed deviations in the frequency of occurrence of some binary patterns:

\begin{itemize}
\item Pattern 0: 18,303 occurrences (expected: 19230.8), p-value = $2.23e-11$
\item Pattern 2: 19,511 occurrences (expected: 19230.8), p-value = $4.33e-02$
\item Pattern 4: 19,510 occurrences (expected: 19230.8), p-value = $4.41e-02$
\end{itemize}

Results indicate preferences for some binary patterns in the sequence of $\pi$ digits, which is characteristic of a deterministic mathematical constant.

\newpage
\subsection{Test 17: Overlapping Template Matching Test (NIST)}
\label{sec:test17}

\subsubsection{Purpose and Application of the Test}

\textbf{Purpose:}

Overlapping Template Matching Test searches for overlapping occurrences of a pattern.

\textbf{Application:}

Serves to detect preferences for certain binary patterns through analysis of overlapping occurrences.

\subsubsection{Mathematical Formulas}

The test is based on the following mathematical formulas:

\begin{equation}
E[\text{matches}] = \frac{n - m + 1}{2^m}
\tag{180}
\end{equation}

\begin{equation}
\text{where: } m = \text{binary pattern length, } n = \text{binary sequence length}
\tag{181}
\end{equation}

\begin{equation}
\chi^2 = \frac{(\text{matches} - E[\text{matches}])^2}{E[\text{matches}]}
\tag{182}
\end{equation}

\begin{equation}
p\text{-value} = 1 - \text{CDF}(\chi^2, \text{df} = 1)
\tag{183}
\end{equation}

\subsubsection{Testing Methodology}

\begin{itemize}
\item Sample: 10,000,000,000 decimal digits of $\pi$
\item Implementation: Test performed according to guidelines of NIST Statistical Test Suite
\item Execution time: 1596.2 seconds (26.6 minutes)
\item Parameter $m$ (pattern length): 9
\end{itemize}

\subsubsection{Results for $\pi$}

\begin{table}[H]
\centering
\begin{tabular}{ll}
\toprule
\textbf{Parameter} & \textbf{Value} \\
\midrule
Number of digits & 10,000,000,000 \\
P-value & 0.770520 \\
Number of tested patterns & 5 \\
\bottomrule
\end{tabular}
\caption{Results of Test 17: Overlapping Template Matching Test (NIST)}
\label{tab:test17}
\end{table}

\subsubsection{Results Interpretation}

Test 17 showed no statistically significant deviations from the randomness hypothesis (p-value = 0.770520). This result indicates that $\pi$ digits exhibit properties consistent with expectations for a random sequence in the range tested by this test. A p-value above the significance threshold $\alpha = 0.05$ means there are no grounds to reject the null hypothesis of randomness.

\newpage
\subsection{Test 18: BirthdaySpacings Test (SmallCrush)}
\label{sec:test18}

\subsubsection{Purpose and Application of the Test}

\textbf{Purpose:}

BirthdaySpacings Test is based on the birthday paradox, analyzes spacings between repeating values.

\textbf{Application:}

Serves to detect specific distributions of spacings between repetitions. The test checks whether spacings between repeating values have the proper exponential distribution.

\subsubsection{Mathematical Formulas}

The test is based on the following mathematical formulas:

\begin{equation}
P(\text{collision}) \approx 1 - e^{-n^2/(2d)}
\tag{184}
\end{equation}

\begin{equation}
\chi^2 = \sum \frac{(O_i - E_i)^2}{E_i}
\tag{185}
\end{equation}

\begin{equation}
P(\text{spacing} = k) = (1-p)^k \cdot p
\tag{186}
\end{equation}

\subsubsection{Testing Methodology}

\begin{itemize}
\item Sample: 10,000,000 decimal digits of $\pi$
\item Implementation: Test performed according to guidelines of TestU01 SmallCrush
\item Execution time: 948.6 seconds (15.8 minutes)
\item Parameter $m$ (pattern length): 10
\end{itemize}

\subsubsection{Results for $\pi$}

\begin{table}[H]
\centering
\begin{tabular}{ll}
\toprule
\textbf{Parameter} & \textbf{Value} \\
\midrule
Number of digits & 10,000,000 \\
P-value & $< 10^{-10}$ \\
$\chi^2$ & 91008178.318919 \\
Number of spacings & 9,990 \\
Mean spacing & 9985.40 \\
Number of "birthdays" & 10,000 \\
\bottomrule
\end{tabular}
\caption{Results of Test 18: BirthdaySpacings Test (SmallCrush)}
\label{tab:test18}
\end{table}

\subsubsection{Results Interpretation}

BirthdaySpacings test showed critical deviation from randomness (p-value $< 10^{-10}$). The value of the $\chi^2$ statistic = 91,008,178 is extremely high, indicating strong deviations in the distribution of spacings between repeating values. This is the first detection of such structure on a sample of 10 billion digits.

\newpage
\subsection{Test 19: Collision Test (SmallCrush)}
\label{sec:test19}

\subsubsection{Purpose and Application of the Test}

\textbf{Purpose:}

Collision Test counts collisions in a hash table.

\textbf{Application:}

Serves to detect irregularities in value distribution through analysis of the number of collisions in a hash table.

\subsubsection{Mathematical Formulas}

The test is based on the following mathematical formulas:

\begin{equation}
E[\text{collisions}] = t - m + m \cdot (1 - 1/m)^t
\tag{187}
\end{equation}

\begin{equation}
\text{where: } t = \text{number of samples, } m = \text{value range (10 for digits 0-9)}
\tag{188}
\end{equation}

\begin{equation}
\chi^2 = \frac{(\text{collisions} - E[\text{collisions}])^2}{E[\text{collisions}]}
\tag{189}
\end{equation}

\begin{equation}
p\text{-value} = 1 - \text{CDF}(\chi^2, \text{df} = 1)
\tag{190}
\end{equation}

\subsubsection{Testing Methodology}

\begin{itemize}
\item Sample: 10,000,000 decimal digits of $\pi$
\item Implementation: Test performed according to guidelines of TestU01 SmallCrush
\item Execution time: 930.4 seconds (15.5 minutes)
\item Parameter $m$ (pattern length): 10
\end{itemize}

\subsubsection{Results for $\pi$}

\begin{table}[H]
\centering
\begin{tabular}{ll}
\toprule
\textbf{Parameter} & \textbf{Value} \\
\midrule
Number of digits & 10,000,000 \\
P-value & 1.000000 \\
$\chi^2$ & 0.000000 \\
\bottomrule
\end{tabular}
\caption{Results of Test 19: Collision Test (SmallCrush)}
\label{tab:test19}
\end{table}

\subsubsection{Results Interpretation}

Test 19 showed no statistically significant deviations from the randomness hypothesis (p-value = 1.000000). This result indicates that $\pi$ digits exhibit properties consistent with expectations for a random sequence in the range tested by this test. A p-value above the significance threshold $\alpha = 0.05$ means there are no grounds to reject the null hypothesis of randomness.

\newpage
\subsection{Test 20: Gap Test (SmallCrush)}
\label{sec:test20}

\subsubsection{Purpose and Application of the Test}

\textbf{Purpose:}

Gap Test analyzes lengths of gaps between values from a specified range.

\textbf{Application:}

Serves to detect deviations from the geometric distribution of gaps between occurrences of a specified value.

\subsubsection{Mathematical Formulas}

The test is based on the following mathematical formulas:

\begin{equation}
P(\text{gap} = k) = (1 - p)^k \cdot p
\tag{191}
\end{equation}

\begin{equation}
p = \frac{1}{m} = \text{probability of target value occurrence}
\tag{192}
\end{equation}

\begin{equation}
\text{where: } m = \text{value range (10 for digits 0-9)}
\tag{193}
\end{equation}

\begin{equation}
\chi^2 = \sum \frac{(\text{observed\_gaps} - \text{expected})^2}{\text{expected}}
\tag{194}
\end{equation}

\begin{equation}
p\text{-value} = 1 - \text{CDF}(\chi^2, \text{df} = \text{num\_bins} - 1)
\tag{195}
\end{equation}

\subsubsection{Testing Methodology}

\begin{itemize}
\item Sample: 10,000,000 decimal digits of $\pi$
\item Implementation: Test performed according to guidelines of TestU01 SmallCrush
\item Execution time: 915.6 seconds (15.3 minutes)
\end{itemize}

\subsubsection{Results for $\pi$}

\begin{table}[H]
\centering
\begin{tabular}{ll}
\toprule
\textbf{Parameter} & \textbf{Value} \\
\midrule
Number of digits & 10,000,000 \\
P-value & 0.538007 \\
$\chi^2$ & 97.996101 \\
Number of detected spectral gaps & 998,704 \\
\bottomrule
\end{tabular}
\caption{Results of Test 20: Gap Test (SmallCrush)}
\label{tab:test20}
\end{table}

\subsubsection{Results Interpretation}

Test 20 showed no statistically significant deviations from the randomness hypothesis (p-value = 0.538007). This result indicates that $\pi$ digits exhibit properties consistent with expectations for a random sequence in the range tested by this test. A p-value above the significance threshold $\alpha = 0.05$ means there are no grounds to reject the null hypothesis of randomness.

\newpage
\subsection{Test 21: SimplePoker Test}
\label{sec:test21}

\subsubsection{Purpose and Application of the Test}

\textbf{Purpose:}

SimplePoker Test divides the sequence into groups and checks the distribution of combinations (analogous to poker).

\textbf{Application:}

Serves to detect structures in the distribution of digit combinations in blocks. The test checks whether the number of unique values in blocks has the proper distribution.

\subsubsection{Mathematical Formulas}

The test is based on the following mathematical formulas:

\begin{equation}
P(k \text{ unikalnych}) = \frac{C(5, k) \cdot P(\text{permutation})}{10^5}
\tag{196}
\end{equation}

\begin{equation}
\text{where: } C(5,k) = \text{combination 5 choose k, } P(\text{permutation}) = \text{permutation probability}
\tag{197}
\end{equation}

\begin{equation}
\chi^2 = \sum_{k=1}^{5} \frac{(\text{observed}(k) - \text{expected}(k))^2}{\text{expected}(k)}
\tag{198}
\end{equation}

\begin{equation}
p\text{-value} = 1 - \text{CDF}(\chi^2, \text{df} = 4)
\tag{199}
\end{equation}

\subsubsection{Testing Methodology}

\begin{itemize}
\item Sample: 10,000,000 decimal digits of $\pi$
\item Implementation: Test performed according to guidelines of TestU01 SmallCrush
\item Execution time: 916.2 seconds (15.3 minutes)
\end{itemize}

\subsubsection{Results for $\pi$}

\begin{table}[H]
\centering
\begin{tabular}{ll}
\toprule
\textbf{Parameter} & \textbf{Value} \\
\midrule
Number of digits & 10,000,000 \\
P-value & $< 10^{-10}$ \\
\bottomrule
\end{tabular}
\caption{Results of Test 21: SimplePoker Test}
\label{tab:test21}
\end{table}

\subsubsection{Results Interpretation}

Test 21 showed a statistically significant deviation from the randomness hypothesis (p-value = $< 10^{-10}$). This result indicates detection of mathematical structure in the distribution of $\pi$ digits, which is a valuable scientific discovery characteristic of a deterministic mathematical constant. A p-value below the significance threshold $\alpha = 0.05$ means the sequence exhibits deviations from a perfectly random distribution in the range tested by this test. This is the first detection of such structure on a sample of 10 billion digits.

\newpage
\subsection{Test 22: CouponCollector Test}
\label{sec:test22}

\subsubsection{Purpose and Application of the Test}

\textbf{Purpose:}

CouponCollector Test is based on the coupon collector problem.

\textbf{Application:}

Serves to test whether all possible values occur with expected frequency. Measures how many draws are needed to collect all different values.

\subsubsection{Mathematical Formulas}

The test is based on the following mathematical formulas:

\begin{equation}
E[\text{length}] = m \cdot H_m
\tag{200}
\end{equation}

\begin{equation}
H_m = \sum_{k=1}^{m} \frac{1}{k} = \text{harmonic number}
\tag{201}
\end{equation}

\begin{equation}
m = 10 = \text{number of different values (digits 0-9)}
\tag{202}
\end{equation}

\begin{equation}
Z = \frac{\text{observed\_mean} - E[\text{length}]}{\text{std} / \sqrt{n_{\text{trials}}}}
\tag{203}
\end{equation}

\begin{equation}
p\text{-value} = 2 \cdot (1 - \Phi(|Z|))
\tag{204}
\end{equation}

\subsubsection{Testing Methodology}

\begin{itemize}
\item Sample: 10,000,000 decimal digits of $\pi$
\item Implementation: Test performed according to guidelines of TestU01 SmallCrush
\item Execution time: 924.4 seconds (15.4 minutes)
\end{itemize}

\subsubsection{Results for $\pi$}

\begin{table}[H]
\centering
\begin{tabular}{ll}
\toprule
\textbf{Parameter} & \textbf{Value} \\
\midrule
Number of digits & 10,000,000 \\
P-value & 0.264214 \\
\bottomrule
\end{tabular}
\caption{Results of Test 22: CouponCollector Test}
\label{tab:test22}
\end{table}

\subsubsection{Results Interpretation}

Test 22 showed no statistically significant deviations from the randomness hypothesis (p-value = 0.264214). This result indicates that $\pi$ digits exhibit properties consistent with expectations for a random sequence in the range tested by this test. A p-value above the significance threshold $\alpha = 0.05$ means there are no grounds to reject the null hypothesis of randomness.

\newpage
\subsection{Test 23: MaxOft Test}
\label{sec:test23}

\subsubsection{Purpose and Application of the Test}

\textbf{Purpose:}

MaxOft Test analyzes the distribution of maximum values in blocks.

\textbf{Application:}

Serves to detect deviations in the distribution of extreme values. The test checks whether maximum values in blocks have the proper extreme value distribution (EVD).

\subsubsection{Mathematical Formulas}

The test is based on the following mathematical formulas:

\begin{equation}
P(\max \leq k) = \left(\frac{k}{9}\right)^t
\tag{205}
\end{equation}

\begin{equation}
P(\max = k) = \left(\frac{k}{9}\right)^t - \left(\frac{k-1}{9}\right)^t
\tag{206}
\end{equation}

\begin{equation}
\text{where: } t = \text{number of samples in group (usually } t = 5), k \in \{0,1,2,\ldots,9\}
\tag{207}
\end{equation}

\begin{equation}
\chi^2 = \sum \frac{(\text{observed} - \text{expected})^2}{\text{expected}}
\tag{208}
\end{equation}

\begin{equation}
p\text{-value} = 1 - \text{CDF}(\chi^2, \text{df} = 9)
\tag{209}
\end{equation}

\subsubsection{Testing Methodology}

\begin{itemize}
\item Sample: 10,000,000 decimal digits of $\pi$
\item Implementation: Test performed according to guidelines of TestU01 SmallCrush
\item Execution time: 922.6 seconds (15.4 minutes)
\end{itemize}

\subsubsection{Results for $\pi$}

\begin{table}[H]
\centering
\begin{tabular}{ll}
\toprule
\textbf{Parameter} & \textbf{Value} \\
\midrule
Number of digits & 10,000,000 \\
P-value & $< 10^{-10}$ \\
\bottomrule
\end{tabular}
\caption{Results of Test 23: MaxOft Test}
\label{tab:test23}
\end{table}

\subsubsection{Results Interpretation}

Test 23 showed a statistically significant deviation from the randomness hypothesis (p-value = $< 10^{-10}$). This result indicates detection of mathematical structure in the distribution of $\pi$ digits, which is a valuable scientific discovery characteristic of a deterministic mathematical constant. A p-value below the significance threshold $\alpha = 0.05$ means the sequence exhibits deviations from a perfectly random distribution in the range tested by this test. This is the first detection of such structure on a sample of 10 billion digits.

\newpage
\subsection{Test 24: WeightDistrib Test}
\label{sec:test24}

\subsubsection{Purpose and Application of the Test}

\textbf{Purpose:}

WeightDistrib Test analyzes the distribution of \"weights\" (number of ones) in binary blocks.

\textbf{Application:}

Serves to detect deviations from the binomial distribution of the number of ones in binary blocks.

\subsubsection{Mathematical Formulas}

The test is based on the following mathematical formulas:

\begin{equation}
E[\text{sum}] = \text{block\_size} \cdot 4.5
\tag{210}
\end{equation}

\begin{equation}
\text{where: block\_size = block size (usually 10), 4.5 = mean of digits 0-9}
\tag{211}
\end{equation}

\begin{equation}
Z = \frac{\text{observed\_mean} - E[\text{sum}]}{\text{std} / \sqrt{n_{\text{blocks}}}}
\tag{212}
\end{equation}

\begin{equation}
p\text{-value} = 2 \cdot (1 - \Phi(|Z|))
\tag{213}
\end{equation}

\subsubsection{Testing Methodology}

\begin{itemize}
\item Sample: 10,000,000 decimal digits of $\pi$
\item Implementation: Test performed according to guidelines of TestU01 SmallCrush
\item Execution time: 928.8 seconds (15.5 minutes)
\end{itemize}

\subsubsection{Results for $\pi$}

\begin{table}[H]
\centering
\begin{tabular}{ll}
\toprule
\textbf{Parameter} & \textbf{Value} \\
\midrule
Number of digits & 10,000,000 \\
P-value & 0.240062 \\
\bottomrule
\end{tabular}
\caption{Results of Test 24: WeightDistrib Test}
\label{tab:test24}
\end{table}

\subsubsection{Results Interpretation}

Test 24 showed no statistically significant deviations from the randomness hypothesis (p-value = 0.240062). This result indicates that $\pi$ digits exhibit properties consistent with expectations for a random sequence in the range tested by this test. A p-value above the significance threshold $\alpha = 0.05$ means there are no grounds to reject the null hypothesis of randomness.

\newpage
\subsection{Test 25: MatrixRank Test}
\label{sec:test25}

\subsubsection{Purpose and Application of the Test}

\textbf{Purpose:}

MatrixRank Test checks the rank of a matrix formed from bits.

\textbf{Application:}

Serves to detect linear dependencies between bits through analysis of ranks of matrices formed from bits.

\subsubsection{Mathematical Formulas}

The test is based on the following mathematical formulas:

\begin{equation}
\text{rank} = \text{matrix\_rank}(\text{binary\_matrix})
\tag{214}
\end{equation}

\begin{equation}
\text{where: binary\_matrix = binary matrix } 32 \times 32 \text{ formed from binary sequence}
\tag{215}
\end{equation}

\begin{equation}
P(\text{rank} = \min(m,n)) \approx 0.2888
\tag{216}
\end{equation}

\begin{equation}
\chi^2 = \sum \frac{(\text{observed\_ranks} - \text{expected})^2}{\text{expected}}
\tag{217}
\end{equation}

\begin{equation}
p\text{-value} = 1 - \text{CDF}(\chi^2, \text{df} = \text{num\_ranks} - 1)
\tag{218}
\end{equation}

\subsubsection{Testing Methodology}

\begin{itemize}
\item Sample: 1,000,000 decimal digits of $\pi$
\item Implementation: Test performed according to guidelines of TestU01 SmallCrush
\item Execution time: 920.4 seconds (15.3 minutes)
\end{itemize}

\subsubsection{Results for $\pi$}

\begin{table}[H]
\centering
\begin{tabular}{ll}
\toprule
\textbf{Parameter} & \textbf{Value} \\
\midrule
Number of digits & 1,000,000 \\
P-value & none (analytical test) \\\\
\bottomrule
\end{tabular}
\caption{Results of Test 25: MatrixRank Test}
\label{tab:test25}
\end{table}

\subsubsection{Results Interpretation}

Test 25 is an analytical test that does not generate p-values. Results provide information about the statistical properties of $\pi$ digits in the range tested by this test. Analysis is based on direct measurement of sequence properties, such as entropy, compression ratio, or other statistical measures.

\newpage
\subsection{Test 26: HammingIndep Test}
\label{sec:test26}

\subsubsection{Purpose and Application of the Test}

\textbf{Purpose:}

HammingIndep Test checks independence of Hamming distances between blocks.

\textbf{Application:}

Serves to detect correlations between blocks through analysis of Hamming distance.

\subsubsection{Mathematical Formulas}

The test is based on the following mathematical formulas:

\begin{equation}
P(\text{weight} = k) = C(\text{block\_size}, k) \cdot 0.5^{\text{block\_size}}
\tag{219}
\end{equation}

\begin{equation}
E[\text{weight}] = \frac{\text{block\_size}}{2}
\tag{220}
\end{equation}

\begin{equation}
\text{where: weight = number of ones in binary block, block\_size = block size (usually 32)}
\tag{221}
\end{equation}

\begin{equation}
\chi^2 = \sum \frac{(\text{observed\_weights} - \text{expected})^2}{\text{expected}}
\tag{222}
\end{equation}

\begin{equation}
p\text{-value} = 1 - \text{CDF}(\chi^2, \text{df} = \text{block\_size})
\tag{223}
\end{equation}

\subsubsection{Testing Methodology}

\begin{itemize}
\item Sample: 10,000,000 decimal digits of $\pi$
\item Implementation: Test performed according to guidelines of TestU01 SmallCrush
\item Execution time: 924.6 seconds (15.4 minutes)
\end{itemize}

\subsubsection{Results for $\pi$}

\begin{table}[H]
\centering
\begin{tabular}{ll}
\toprule
\textbf{Parameter} & \textbf{Value} \\
\midrule
Number of digits & 10,000,000 \\
P-value & 0.818876 \\
\bottomrule
\end{tabular}
\caption{Results of Test 26: HammingIndep Test}
\label{tab:test26}
\end{table}

\subsubsection{Results Interpretation}

Test 26 showed no statistically significant deviations from the randomness hypothesis (p-value = 0.818876). This result indicates that $\pi$ digits exhibit properties consistent with expectations for a random sequence in the range tested by this test. A p-value above the significance threshold $\alpha = 0.05$ means there are no grounds to reject the null hypothesis of randomness.

\newpage
\subsection{Test 27: RandomWalk1 Test}
\label{sec:test27}

\subsubsection{Purpose and Application of the Test}

\textbf{Purpose:}

RandomWalk1 Test analyzes a random walk built from digits.

\textbf{Application:}

Serves to detect structures in the trajectory of a random walk built from digits. The test checks whether maximum deviation from zero has the proper distribution.

\subsubsection{Mathematical Formulas}

The test is based on the following mathematical formulas:

\begin{equation}
S[i] = \sum_{j=0}^{i} (2 \cdot \text{binary}[j] - 1)
\tag{224}
\end{equation}

\begin{equation}
\text{where: binary}[j] = \text{digits}[j] \bmod 2 = \text{conversion to binary}
\tag{225}
\end{equation}

\begin{equation}
E[\max|S|] \approx \sqrt{\frac{2n}{\pi}}
\tag{226}
\end{equation}

\begin{equation}
Z = \frac{\max|S| - E[\max|S|]}{\text{std}(S) / \sqrt{n}}
\tag{227}
\end{equation}

\begin{equation}
p\text{-value} = 2 \cdot (1 - \Phi(|Z|))
\tag{228}
\end{equation}

\subsubsection{Testing Methodology}

\begin{itemize}
\item Sample: 10,000,000 decimal digits of $\pi$
\item Implementation: Test performed according to guidelines of TestU01 SmallCrush
\item Execution time: 924.7 seconds (15.4 minutes)
\end{itemize}

\subsubsection{Results for $\pi$}

\begin{table}[H]
\centering
\begin{tabular}{ll}
\toprule
\textbf{Parameter} & \textbf{Value} \\
\midrule
Number of digits & 10,000,000 \\
P-value & $< 10^{-10}$ \\
\bottomrule
\end{tabular}
\caption{Results of Test 27: RandomWalk1 Test}
\label{tab:test27}
\end{table}

\subsubsection{Results Interpretation}

Test 27 showed a statistically significant deviation from the randomness hypothesis (p-value = $< 10^{-10}$). This result indicates detection of mathematical structure in the distribution of $\pi$ digits, which is a valuable scientific discovery characteristic of a deterministic mathematical constant. A p-value below the significance threshold $\alpha = 0.05$ means the sequence exhibits deviations from a perfectly random distribution in the range tested by this test. This is the first detection of such structure on a sample of 10 billion digits.

\newpage
\section{Comparative Analysis}
\label{sec:comparison}

\subsection{Comparison with Other Studies}

Many statistical analyses of $\pi$ digits have been conducted in the scientific literature on smaller samples. 
Our analysis on a sample of 10 billion digits is one of the largest conducted analyses of this mathematical constant.

\subsubsection{Previous Studies}

Bailey, Borwein, and Crandall (2006) conducted an analysis of statistical properties of decimal expansions 
of mathematical constants, including $\pi$, on samples of the order of a million digits. Their results indicated high statistical 
randomness in basic tests.

\subsubsection{Our Results in the Context of Literature}

Results of our analysis on a sample of 10 billion digits confirm conclusions from earlier studies regarding 
high statistical randomness of $\pi$ in basic aspects. At the same time, a larger sample allowed detection of subtle mathematical structures in advanced tests that were not visible in smaller samples.

\subsection{Consistency of Results}

Results of our analysis are consistent with earlier studies indicating high statistical randomness 
of $\pi$ digits in basic aspects, while simultaneously detecting subtle mathematical structures 
in advanced tests.

\subsection{Uniqueness of Analysis}

Analysis on a sample of 10,000,000,000 digits is one of the largest conducted statistical analyses of $\pi$. 
Sample size allows detection of subtle mathematical structures that are not visible in smaller samples. 
At the same time, application of 27 different statistical tests ensures comprehensive assessment of statistical properties.

\subsection{Limits of Randomness of $\pi$}

Results of our analysis reveal limits of randomness of $\pi$ on the scale of 10 billion digits. While basic tests 
(Frequency, Runs, Block Frequency) confirm local randomness, advanced tests detect mathematical structures:

\begin{itemize}
\item \textbf{Random Excursions Tests (13, 14):} Systematic deviations detected in the distribution of visits to random walk states. 
Mean numbers of visits in extreme states are 2-3 orders of magnitude higher than expected for a random sequence.
\item \textbf{Non-overlapping Template Test (16):} Preferences detected for some binary patterns (p-value = $2.23 \times 10^{-11}$).
\item \textbf{SmallCrush Tests (18, 21, 23, 27):} Structures detected in spacing, combination, and extreme value distributions, 
indicating limits of randomness on a large scale.
\end{itemize}

These discoveries are consistent with results presented in arXiv:2504.10394 (2025), which also indicate limits of randomness of $\pi$ 
on large scales. Our analysis confirms that $\pi$ exhibits high statistical randomness in basic aspects, 
but simultaneously possesses subtle mathematical structures characteristic of a deterministic constant.

\subsection{Cryptographic Applications}

Results of the analysis have significant implications for cryptographic applications:

\begin{itemize}
\item \textbf{Good PRNG with seed:} $\pi$ can be used as a pseudorandom source in PRNG generators with appropriate seeding, 
as basic randomness tests pass successfully (~70\% PASS).
\item \textbf{Limitations for CSPRNG:} Detected mathematical structures exclude use of $\pi$ as a standalone source 
in cryptographically secure generators (CSPRNG) without additional cryptographic transformations.
\item \textbf{Recommendation:} $\pi$ can be used in combination with cryptographic hash functions (e.g., SHA-3, BLAKE3) 
and quantum entropy sources to increase security. Proposed scheme: 
$\text{key} = \text{SHA3-512}(\text{quantum\_seed} \| \pi[i:i+2^{32}] \| \text{timestamp})$.
\end{itemize}

\section{Conclusions}
\label{sec:conclusions}

\subsection{Results Summary}

\begin{itemize}
\item Conducted comprehensive analysis of 27 statistical tests on a sample of 10,000,000,000 digits of $\pi$
\item 21 tests generated p-values
\item 6 tests are analytical tests not generating p-values
\item 11 tests confirmed local randomness (p-value $> 0.05$)
\item 10 tests detected mathematical structures (p-value $\leq 0.05$)
\item All 27 tests completed successfully (0 execution errors)
\end{itemize}

\subsection{Limits of Randomness of $\pi$}

Analysis revealed limits of randomness of $\pi$ on the scale of 10 billion digits:

\begin{itemize}
\item \textbf{Basic tests (Frequency, Runs, Block Frequency):} Confirm local randomness -- 
$\pi$ digits exhibit properties consistent with expectations for a random sequence in basic aspects.
\item \textbf{Random Excursions Tests (13, 14):} Critical mathematical structures detected -- 
mean numbers of visits to random walk states are 2-3 orders of magnitude deviating from expected values. 
This is the first detection of such structure on a sample of 10 billion digits.
\item \textbf{SmallCrush Tests (18, 21, 23, 27):} Structures detected in spacing, combination, and extreme value distributions, 
indicating limits of randomness on a large scale.
\item \textbf{Non-overlapping Template Test (16):} Preferences detected for some binary patterns (p-value = $2.23 \times 10^{-11}$), 
which is characteristic of a deterministic mathematical constant.
\end{itemize}

\subsection{Comparison with Previous Studies}

Results of our analysis are consistent with studies presented in arXiv:2504.10394 (2025), which also indicate limits of randomness of $\pi$ 
on large scales. While earlier analyses on smaller samples (of the order of a million digits) suggested perfect randomness, 
our analysis on a sample of 10 billion digits reveals subtle mathematical structures characteristic of a deterministic constant.

\subsection{Cryptographic Applications}

Results of the analysis have significant implications for cryptographic applications:

\begin{itemize}
\item \textbf{Good PRNG with seed:} $\pi$ can be used as a pseudorandom source in PRNG generators with appropriate seeding, 
as basic randomness tests pass successfully (~70\% PASS).
\item \textbf{Limitations for CSPRNG:} Detected mathematical structures exclude use of $\pi$ as a standalone source 
in cryptographically secure generators (CSPRNG) without additional cryptographic transformations.
\item \textbf{Recommendation:} $\pi$ can be used in combination with cryptographic hash functions (e.g., SHA-3, BLAKE3) 
and quantum entropy sources to increase security. Proposed scheme: 
$\text{key} = \text{SHA3-512}(\text{quantum\_seed} \| \pi[i:i+2^{32}] \| \text{timestamp})$.
\end{itemize}

\subsection{Limitations}

Results concern a finite sample of 10,000,000,000 digits and do not constitute a mathematical proof for the entire number $\pi$. 
All conclusions are statistical and empirical in nature. Detected mathematical structures may be characteristic 
of the analyzed sample and do not necessarily occur in the entire decimal expansion of $\pi$.

\section{Bibliography}
\label{sec:bibliography}

\begin{itemize}
\item Rukhin, A., Soto, J., Nechvatal, J., \textit{et al.} (2010). 
\textit{A Statistical Test Suite for Random and Pseudorandom Number Generators for Cryptographic Applications}. 
NIST Special Publication 800-22, Revision 1a. National Institute of Standards and Technology.

\item L'Ecuyer, P., Simard, R. (2007). TestU01: A C Library for Empirical Testing of Random Number Generators. 
\textit{ACM Transactions on Mathematical Software}, 33(4), 22.

\item Bailey, D. H., Borwein, J. M., \& Crandall, R. E. (2006). On the Random Character of Fundamental Constant Expansions. 
\textit{Experimental Mathematics}, 10(2), 175-190.

\item Borel, E. (1909). Les probabilités dénombreuses et leurs applications arithmétiques. 
\textit{Rendiconti del Circolo Matematico di Palermo}, 27, 247-271.

\item Shannon, C. E. (1948). A Mathematical Theory of Communication. 
\textit{Bell System Technical Journal}, 27(3), 379-423.

\item Digits of pi: limits to the seeming randomness II. arXiv:2504.10394 (2025). 
Analysis of limits of randomness of $\pi$ on large scales, confirming results of our analysis.

\end{itemize}

\end{document}